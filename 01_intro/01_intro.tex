\documentclass[aspectratio=169,12pt,t]{beamer}
\usepackage{graphicx}
\setbeameroption{hide notes}
\setbeamertemplate{note page}[plain]
\usepackage{listings}

% header.tex: boring LaTeX/Beamer details + macros

% get rid of junk
\usetheme{default}
\beamertemplatenavigationsymbolsempty
\hypersetup{pdfpagemode=UseNone} % don't show bookmarks on initial view


% font
\usepackage{fontspec}
\setsansfont
  [ ExternalLocation = ../fonts/ ,
    UprightFont = *-regular ,
    BoldFont = *-bold ,
    ItalicFont = *-italic ,
    BoldItalicFont = *-bolditalic ]{texgyreheros}
\setbeamerfont{note page}{family*=pplx,size=\footnotesize} % Palatino for notes
% "TeX Gyre Heros can be used as a replacement for Helvetica"
% I've placed them in fonts/; alternatively you can install them
% permanently on your system as follows:
%     Download http://www.gust.org.pl/projects/e-foundry/tex-gyre/heros/qhv2.004otf.zip
%     In Unix, unzip it into ~/.fonts
%     In Mac, unzip it, double-click the .otf files, and install using "FontBook"

% named colors
\definecolor{offwhite}{RGB}{255,250,240}
\definecolor{gray}{RGB}{155,155,155}
\definecolor{purple}{RGB}{177,13,201}
\definecolor{green}{RGB}{46,204,64}

\definecolor{background}{RGB}{255,255,255}
\definecolor{foreground}{RGB}{24,24,24}
\definecolor{title}{RGB}{27,94,134}
\definecolor{subtitle}{RGB}{22,175,124}
\definecolor{hilit}{RGB}{122,0,128}
\definecolor{vhilit}{RGB}{255,0,128}
\definecolor{codehilit}{RGB}{255,0,128}
\definecolor{lolit}{RGB}{95,95,95}
\definecolor{myyellow}{rgb}{1,1,0.7}
\definecolor{nhilit}{RGB}{128,0,128}  % hilit color in notes
\definecolor{nvhilit}{RGB}{255,0,128} % vhilit for notes

\newcommand{\hilit}{\color{hilit}}
\newcommand{\vhilit}{\color{vhilit}}
\newcommand{\nhilit}{\color{nhilit}}
\newcommand{\nvhilit}{\color{nvhilit}}
\newcommand{\lolit}{\color{lolit}}

% use those colors
\setbeamercolor{titlelike}{fg=title}
\setbeamercolor{subtitle}{fg=subtitle}
\setbeamercolor{institute}{fg=lolit}
\setbeamercolor{normal text}{fg=foreground,bg=background}
\setbeamercolor{item}{fg=foreground} % color of bullets
\setbeamercolor{subitem}{fg=lolit}
\setbeamercolor{itemize/enumerate subbody}{fg=lolit}
\setbeamertemplate{itemize subitem}{{\textendash}}
\setbeamerfont{itemize/enumerate subbody}{size=\footnotesize}
\setbeamerfont{itemize/enumerate subitem}{size=\footnotesize}

% page number
\setbeamertemplate{footline}{%
    \raisebox{5pt}{\makebox[\paperwidth]{\hfill\makebox[20pt]{\lolit
          \scriptsize\insertframenumber}}}\hspace*{5pt}}

% add a bit of space at the top of the notes page
\addtobeamertemplate{note page}{\setlength{\parskip}{12pt}}

% default link color
\hypersetup{colorlinks, urlcolor={hilit}}

\lstset{language=bash,
        basicstyle=\ttfamily\scriptsize,
        frame=single,
        commentstyle=,
        backgroundcolor=\color{offwhite},
        showspaces=false,
        showstringspaces=false
        }


% a few macros
\newcommand{\bi}{\begin{itemize}}
\newcommand{\bbi}{\vspace{24pt} \begin{itemize} \itemsep8pt}
\newcommand{\ei}{\end{itemize}}
\newcommand{\be}{\begin{enumerate}}
\newcommand{\bbe}{\vspace{24pt} \begin{enumerate} \itemsep8pt}
\newcommand{\ee}{\end{enumerate}}
\newcommand{\ig}{\includegraphics}
\newcommand{\subt}[1]{{\footnotesize \color{subtitle} {#1}}}
\newcommand{\ttsm}{\tt \small}
\newcommand{\ttfn}{\tt \footnotesize}
\newcommand{\figh}[2]{\centerline{\includegraphics[height=#2\textheight]{#1}}}
\newcommand{\figw}[2]{\centerline{\includegraphics[width=#2\textwidth]{#1}}}


%%%%%%%%%%%%%%%%%%%%%%%%%%%%%%%%%%%%%%%%%%%%%%%%%%%%%%%%%%%%%%%%%%%%%%
% end of header
%%%%%%%%%%%%%%%%%%%%%%%%%%%%%%%%%%%%%%%%%%%%%%%%%%%%%%%%%%%%%%%%%%%%%%

% title info
\title{BMI 826}
\subtitle{Advanced Data Analysis}
\author{\href{https://kbroman.org}{Karl Broman}}
\institute{Biostatistics \& Medical Informatics, UW{\textendash}Madison}
\date{\href{https://kbroman.org}{\tt \scriptsize \color{foreground} kbroman.org}
\\[-4pt]
\href{https://github.com/kbroman}{\tt \scriptsize \color{foreground} github.com/kbroman}
\\[-4pt]
\href{https://twitter.com/kwbroman}{\tt \scriptsize \color{foreground} @kwbroman}
\\[-4pt]
{\scriptsize Course web: \href{https://kbroman.org/AdvData}{\tt kbroman.org/AdvData}}
}


\begin{document}

% title slide
{
\setbeamertemplate{footline}{} % no page number here
\frame{
  \titlepage

  \note{This is the introductory lecture for a special topics course at
    UW{\textendash}Madison on advanced data analysis. It might be
    better called ``The craft of data analysis.''

    Data analysis involves both a set of skills and a way of thinking
    about and looking at data. It is critical to consider the
    scientific context of the data, and to focus on the scientific
    questions for which data were gathered.

    And so the course has three streams: first, a set of case studies
    chosen to illustrate important lessons; second, a set of
    tutorials to introduce certain skills useful for developing and
    organizing reproducible data analyses; and third, homework
    assignments that involve direct, guided analyses of data.
}
} }





\begin{frame}{What is data analysis?}

\vspace*{-5mm}

\onslide<2->{\bbi}
\onslide<2->{\item Answer questions with data}
\onslide<3->{\item Identify/develop appropriate methods to do so}
\onslide<4->{\item Quantify uncertainty}
\onslide<4->{\item Assess appropriateness of the method}
\onslide<4->{\item Identify problems in the data}
\onslide<4->{\item Understand where the data came from and possible
  biases or other limitations}
\onslide<4->{\item Manage and organize data}
\onslide<4->{\item Manage/organize/develop/test software and analyses
  so they are reproducible and correct}
\onslide<4->{\item Communicate/present the results}
\onslide<2->{\ei}

\note{
  What is data analysis? It is a lot of things. First, of course, it
  is an effort to use data to answer questions. But also it involves
  identifying appropriate methods to answer questions with data, or
  developing new methods if such methods don't exist. And I would
  assert that it is always important to attempt to quantify the
  uncertainty in our answers.

  We also need to be able to assess the appropriateness of
  methods, to identify problems in the data, and to understand where
  the data come from and any biases and other limitations in the data.

  Further, good data analysis includes methods for managing and
  organizing data, and for managing, organizing, developing, and
  testing software and data analyses so that they are reproducible and
  correct.

  Finally, data analysis includes the communication and presentation
  of the results, in a form appropriate to the audience of the work.
}

\end{frame}


\begin{frame}{Important principles}

\bbe
\item You'll never know all the methods

\item Focus on the question and data, not the method

\item ``{\hilit Because you can}'' is not a good reason to do something
\ee

\note{
}

\end{frame}



\begin{frame}{This course}

\bbi
\item Data analysis projects

\item Tools for organizing analyses so that they are reproducible

\item Stories of data analysis projects, with lessons
\ei

\note{
}

\end{frame}



\begin{frame}[c]{Lesson 1}


\centerline{\Large Follow up artifacts}

\bigskip \bigskip

\centerline{\large \hilit They might be the most interesting results}


\note{
}

\end{frame}


\begin{frame}[c]{}

\figh{Figs/mfdmaps_paper.png}{0.9}

\note{
}

\end{frame}



\begin{frame}[c]{Eucalypt genetic map}

\figh{Figs/eucalypt_map.pdf}{0.75}

\vspace{5mm}

\hfill {\lolit \scriptsize
Byrne et al., Theor Appl Genet 91:869--875, 1995}

\note{
}

\end{frame}



\begin{frame}[c]{Meiosis}

\figh{Figs/meiosis.png}{0.95}

\note{
}

\end{frame}





\begin{frame}{Ordering markers}

\vspace{5mm}

$$
\begin{array}{c|} \mathsf{A} \\ \mathsf{B} \\ \mathsf{C} \end{array}
\left. \enspace
\begin{array}{|c} \mathsf{a} \\ \mathsf{b} \\ \mathsf{c} \end{array}
\right.
\hspace*{1cm} \longrightarrow \hspace*{1cm}
\begin{array}{ccc} \mathsf{ABC} & \hspace*{3mm} & \mathsf{abc} \\
\only<3>{\vhilit} \mathsf{ABc} & \hspace*{3mm} & \only<3>{\vhilit} \mathsf{abC} \\
\only<4>{\vhilit} \mathsf{Abc} & \hspace*{3mm} & \only<4>{\vhilit} \mathsf{aBC} \\
\only<2>{\vhilit} \mathsf{AbC} & \hspace*{3mm} & \only<2>{\vhilit}\mathsf{aBc} \end{array}
$$

\vspace{5mm}
\begin{center}

{\hilit Marker orders: \hspace{5mm}
{ \only<2>{ \vhilit } A--B--C } \hspace{5mm} { \only<3>{ \vhilit }
  A--C--B } \hspace{5mm} { \only<4>{ \vhilit } B--A--C }}

\vspace{15mm}

\onslide<5>{
With {\hilit $\mathsf{M}$} markers, there are {\hilit
  $\mathsf{M}!/\mathsf{2}$} possible orderings.

\vspace{3mm}

For {\hilit $\mathsf{M = {\vhilit 100}}$}, {\hilit $\mathsf{M!/2
    \approx {\vhilit 10^{157}}}$}
}

\end{center}

\note{
}

\end{frame}



\begin{frame}[c]{CEPH pedigrees}

\figw{Figs/ceph_pedigrees.pdf}{1.0}

\note{
}

\end{frame}




\begin{frame}{Marshfield genetic maps: Tasks}

\bbi
\item Assemble data

\item Understand marker names

  \bi
  \item[] AFM, UT, CHLC (GATA etc.), Mfd, D*S*
  \ei

\item Identify cryptic duplicates

\item Order markers and identify genotyping errors

  \bi
  \item[] Removed 764 / 969,425 genotypes
  \ei
\ei

\note{
}

\end{frame}






\begin{frame}[c]{CRIMAP chrompic}

\begin{center}

{\fontsize{6.9pt}{7.6}\selectfont \tt

1332-03 ma {\color{purple} -11-i--11--111-i111-11-1111i--1111i-1111-i--11---1--11-1111-1-1i1---1}... \\
1332-03 pa {\color{green} 0000----0000000o00o00-000-000-0000o00-000-00000-0000}{\color{purple} 1}---{\color{green} 000-00-o000-0}... \\[6pt]
1332-04 ma {\color{purple} -11-i--11--111-1111-11-i111i--i1111-1111-i--11---1--11-1111-1-11i--11}... \\
1332-04 pa {\color{purple} 1111----1111111111i11-1i1-111-i111i11-111-11111-11111---111-11-1i1111}... \\[6pt]
1332-05 ma {\color{purple} -11-i--11--111-i111-11-1111}{\color{green} o--0000o-0000-o--00---0--00-0000-0-0o0--00}... \\
1332-05 pa {\color{green} 0000----0000000o00o00-000}-{\color{purple} 111-1111i11-111-1111--11111---111-11-i11111}... \\[6pt]
1332-06 ma {\color{green} -00-o--00--000-o000-00-0000o--0000o-0000-o--00---0--00-0000}-{\color{purple} 1-11i--11}... \\
1332-06 pa {\color{purple} 1111----1111111i11i11-111-111-1111i11-111-11111-11111---111-11-1i1111}... \\[6pt]
1332-07 ma {\color{green} -00-o--00--000-o000-00-0000o--0000o-0000-o--00---0--00-0000-0-0o0--00}... \\
1332-07 pa {\color{purple} 1111----1111111i11i11-111-111-1111i11-111-1111--11111---111-11-i11111}... \\[6pt]
1332-08 ma {\color{purple} -1}{\color{green} 0-o--00--000-00-0-00-0000o--o0000-0000-o--00---0}--{\color{purple} 11-1111-1-1i1--11}... \\
1332-08 pa {\color{green} 0000----000000000-o00-0}{\color{purple} 1}{\color{green} 0-000-o000o00-000-00000-00000---000-00-o00000}... \\[6pt]
1332-10 ma {\color{purple} -11-i--1---111-i111-11-1111i--1111i-1111-i--11---1--11-1111-1-1i1--11}... \\
1332-10 pa {\color{purple} 1}{\color{green} 000-----000000o00o00-000-000-0000o00-000-00000-00000---000-00-o00000}... \\[6pt]
1332-11 ma {\color{purple} -11}-{\color{green} o--00--000-o000-00-0000o--0000o-0000-o--00---0--00-0000-0-0o0--00}... \\
1332-11 pa {\color{purple} 1111----1111111i11i11-111-111-1111i11-111-11111-11111---111-11-i11111}... \\[6pt]
1332-12 ma {\color{green} -00}-{\color{purple} i--11--111-i111-11---11i--1111i-1111-i--11---1--11-1111-1-1i1---1}... \\
1332-12 pa {\color{green} 0000----0000000o00o00-0---000-0000o00-000-00000-00000---000-00-o000-0}... \\[6pt]
1332-17 ma {\color{purple} -11-i--1---11--i111-1--1111i--1111i-1111-i--11---1--11-11}{\color{green} 00-0-00o--00}... \\
1332-17 pa {\color{green} 0000-----0000--o00o00-000-000-0000o-0-000-0000--00000---000-00-0o0000}...

}
\end{center}

\note{
}

\end{frame}

\begin{frame}[c]{CRIMAP chrompic}

\begin{center}

{\fontsize{6.9pt}{7.6}\selectfont \tt

1332-03 ma {\lolit -11-i--11--111-i111-11-1111i--1111i-1111-i--11---1--11-1111-1-1i1---1}... \\
1332-03 pa {\lolit 0000----0000000o00o00-000-000-0000o00-000-00000-0000}{\vhilit 1}{\lolit ---000-00-o000-0}... \\[6pt]
1332-04 ma {\lolit -11-i--11--111-1111-11-i111i--i1111-1111-i--11---1--11-1111-1-11i--11}... \\
1332-04 pa {\lolit 1111----1111111111i11-1i1-111-i111i11-111-11111-11111---111-11-1i1111}... \\[6pt]
1332-05 ma {\lolit -11-i--11--111-i111-11-1111}{\lolit o--0000o-0000-o--00---0--00-0000-0-0o0--00}... \\
1332-05 pa {\lolit 0000----0000000o00o00-000-}{\lolit 111-1111i11-111-1111--11111---111-11-i11111}... \\[6pt]
1332-06 ma {\lolit -00-o--00--000-o000-00-0000o--0000o-0000-o--00---0--00-0000}{\lolit -}{\lolit 1-11i--11}... \\
1332-06 pa {\lolit 1111----1111111i11i11-111-111-1111i11-111-11111-11111---111-11-1i1111}... \\[6pt]
1332-07 ma {\lolit -00-o--00--000-o000-00-0000o--0000o-0000-o--00---0--00-0000-0-0o0--00}... \\
1332-07 pa {\lolit 1111----1111111i11i11-111-111-1111i11-111-1111--11111---111-11-i11111}... \\[6pt]
1332-08 ma {\lolit -}{\vhilit 1}{\lolit 0-o--00--000-00-0-00-0000o--o0000-0000-o--00---0--}{\lolit 11-1111-1-1i1--11}... \\
1332-08 pa {\lolit 0000----000000000-o00-0}{\vhilit 1}{\lolit 0-000-o000o00-000-00000-00000---000-00-o00000}... \\[6pt]
1332-10 ma {\lolit -11-i--1---111-i111-11-1111i--1111i-1111-i--11---1--11-1111-1-1i1--11}... \\
1332-10 pa {\vhilit 1}{\lolit 000-----000000o00o00-000-000-0000o00-000-00000-00000---000-00-o00000}... \\[6pt]
1332-11 ma {\lolit -11-}{\lolit o--00--000-o000-00-0000o--0000o-0000-o--00---0--00-0000-0-0o0--00}... \\
1332-11 pa {\lolit 1111----1111111i11i11-111-111-1111i11-111-11111-11111---111-11-i11111}... \\[6pt]
1332-12 ma {\lolit -00-}{\lolit i--11--111-i111-11---11i--1111i-1111-i--11---1--11-1111-1-1i1---1}... \\
1332-12 pa {\lolit 0000----0000000o00o00-0---000-0000o00-000-00000-00000---000-00-o000-0}... \\[6pt]
1332-17 ma {\lolit -11-i--1---11--i111-1--1111i--1111i-1111-i--11---1--11-11}{\lolit 00-0-00o--00}... \\
1332-17 pa {\lolit 0000-----0000--o00o00-000-000-0000o-0-000-0000--00000---000-00-0o0000}...

}

\end{center}

\note{
}

\end{frame}


\begin{frame}[fragile]{Top of chr 22}

\begin{center}

{\fontsize{8pt}{9.5}\selectfont

\begin{verbatim}

 Marker       Dnumber     sex-ave(cM)       female(cM)        male(cM)

1 ATA2G02     Unknown             0.00             0.00             0.00
                          1.79             0.00             2.60
2 GATA198B05  Unknown             1.79             0.00             2.60
                          2.27             3.32             0.00
3 AFM217xf4   D22S420             4.06             3.32             2.60
                          4.26             4.51             5.42
4 AFM288we5   D22S427             8.32             7.83             8.02
                          5.25             7.52             3.00
5 265yf5      D22S425            13.57            15.35            11.02
                          0.03             0.00             0.65
6 GGAA10F06   D22S686            13.60            15.35            11.67
                          0.84             0.00             0.82
7 AFMa037zd1  D22S539            14.44            15.35            12.49
                          0.00             0.00             0.00
8 AFM292va9   D22S446            14.44            15.35            12.49
                          3.27             5.91             0.00
9 Mfd51       D22S257            17.71            21.26            12.49
\end{verbatim}

}
\end{center}

\note{
}

\end{frame}


\begin{frame}[c]{Marker search}

\figh{Figs/markersearch.jpg}{0.8}

\note{
}

\end{frame}






\begin{frame}{10th worst graph}


\figh{Figs/recrate.jpg}{0.85}

\vspace{0mm}

\hfill {\scriptsize \lolit
Broman et al., Am J Hum Genet 63:861--869, 1998}

\note{
}

\end{frame}



\begin{frame}{Total no. crossovers}

\vspace{6mm}

\figw{Figs/totalXO.jpg}{1.0}

\vspace{10mm}


\hfill {\scriptsize \lolit
Broman et al., Am J Hum Genet 63:861--869, 1998}

\note{
}

\end{frame}



\begin{frame}{Crossover locations}

\vspace{2mm}

\figh{Figs/xoloc.jpg}{0.75}

\vspace{4mm}

\hfill {\scriptsize \lolit
Broman et al., Am J Hum Genet 63:861--869, 1998}

\note{
}

\end{frame}




\begin{frame}[c]{Family 884, chr 6}


\figw{Figs/autozygosity.pdf}{1.0}

\note{
}

\end{frame}



\begin{frame}[c]{}

\figh{Figs/autozyg_paper.png}{0.9}

\note{
}

\end{frame}


\begin{frame}{Autozygosity}

  \vspace{4mm}

\figw{Figs/autozyg_table.jpg}{1.0}

\vspace{8mm}

\hfill {\scriptsize \lolit Broman and Weber, Am J Hum Genet 65:1493--1500, 1999}

\note{
}

\end{frame}



\begin{frame}[c]{}

\figh{Figs/xoi_paper.png}{0.9}

\note{
}

\end{frame}






\begin{frame}{Crossover interference}

\figh{Figs/xodist.jpg}{0.8}

\vspace{3mm}

\hfill {\scriptsize \lolit Broman and Weber, Am J Hum Genet 66:1911--1926, 2000}

\note{
}

\end{frame}



\begin{frame}[c]{Maternal chr 8}
\figw{Figs/chr8m.png}{1.0}

\note{
}

\end{frame}



\begin{frame}{Apparent triple XOs}

\figh{Figs/inversion_genotypes.jpg}{0.83}

\vspace{2mm}

\hfill {\scriptsize \lolit Broman et al., In: \emph{Science and Statistics: A Festschrift for Terry Speed}, 2003}

\note{
}

\end{frame}


\begin{frame}{Chr 8p inversion}

\figh{Figs/inversion_fish.jpg}{0.83}

\vspace{2mm}

\hfill {\scriptsize \lolit Broman et al., In: \emph{Science and Statistics: A Festschrift for Terry Speed}, 2003}

\note{
}

\end{frame}



\begin{frame}{Comparison to sequence}


\vspace{4mm}

\figh{Figs/matise_fig.jpg}{0.7}

\vspace{6mm}

\hfill {\scriptsize \lolit Matise et al., Am J Hum Genet 70:1398--1410, 2002}

\note{
}

\end{frame}


\begin{frame}{Comparison to sequence}

\vspace{5mm}

\figh{Figs/chr22rev.pdf}{0.7}

\note{
}

\end{frame}








\begin{frame}[c]{Lesson 1}


\centerline{\Large Follow up artifacts}

\bigskip \bigskip

\centerline{\large \hilit They might be the most interesting results}

\note{
}

\end{frame}




\begin{frame}[c]{Lesson 2}


  \centering

\Large The simplest things \\[4pt]
    can be the most important

\note{
}

\end{frame}



\end{document}
