\documentclass[aspectratio=169,12pt,t]{beamer}
\usepackage{graphicx}
\setbeameroption{hide notes}
\setbeamertemplate{note page}[plain]
\usepackage{listings}

% header.tex: boring LaTeX/Beamer details + macros

% get rid of junk
\usetheme{default}
\beamertemplatenavigationsymbolsempty
\hypersetup{pdfpagemode=UseNone} % don't show bookmarks on initial view


% font
\usepackage{fontspec}
\setsansfont
  [ ExternalLocation = ../fonts/ ,
    UprightFont = *-regular ,
    BoldFont = *-bold ,
    ItalicFont = *-italic ,
    BoldItalicFont = *-bolditalic ]{texgyreheros}
\setbeamerfont{note page}{family*=pplx,size=\footnotesize} % Palatino for notes
% "TeX Gyre Heros can be used as a replacement for Helvetica"
% I've placed them in fonts/; alternatively you can install them
% permanently on your system as follows:
%     Download http://www.gust.org.pl/projects/e-foundry/tex-gyre/heros/qhv2.004otf.zip
%     In Unix, unzip it into ~/.fonts
%     In Mac, unzip it, double-click the .otf files, and install using "FontBook"

% named colors
\definecolor{offwhite}{RGB}{255,250,240}
\definecolor{gray}{RGB}{155,155,155}
\definecolor{purple}{RGB}{177,13,201}
\definecolor{green}{RGB}{46,204,64}

\definecolor{background}{RGB}{255,255,255}
\definecolor{foreground}{RGB}{24,24,24}
\definecolor{title}{RGB}{27,94,134}
\definecolor{subtitle}{RGB}{22,175,124}
\definecolor{hilit}{RGB}{122,0,128}
\definecolor{vhilit}{RGB}{255,0,128}
\definecolor{codehilit}{RGB}{255,0,128}
\definecolor{lolit}{RGB}{95,95,95}
\definecolor{myyellow}{rgb}{1,1,0.7}
\definecolor{nhilit}{RGB}{128,0,128}  % hilit color in notes
\definecolor{nvhilit}{RGB}{255,0,128} % vhilit for notes

\newcommand{\hilit}{\color{hilit}}
\newcommand{\vhilit}{\color{vhilit}}
\newcommand{\nhilit}{\color{nhilit}}
\newcommand{\nvhilit}{\color{nvhilit}}
\newcommand{\lolit}{\color{lolit}}

% use those colors
\setbeamercolor{titlelike}{fg=title}
\setbeamercolor{subtitle}{fg=subtitle}
\setbeamercolor{institute}{fg=lolit}
\setbeamercolor{normal text}{fg=foreground,bg=background}
\setbeamercolor{item}{fg=foreground} % color of bullets
\setbeamercolor{subitem}{fg=lolit}
\setbeamercolor{itemize/enumerate subbody}{fg=lolit}
\setbeamertemplate{itemize subitem}{{\textendash}}
\setbeamerfont{itemize/enumerate subbody}{size=\footnotesize}
\setbeamerfont{itemize/enumerate subitem}{size=\footnotesize}

% page number
\setbeamertemplate{footline}{%
    \raisebox{5pt}{\makebox[\paperwidth]{\hfill\makebox[20pt]{\lolit
          \scriptsize\insertframenumber}}}\hspace*{5pt}}

% add a bit of space at the top of the notes page
\addtobeamertemplate{note page}{\setlength{\parskip}{12pt}}

% default link color
\hypersetup{colorlinks, urlcolor={hilit}}

\lstset{language=bash,
        basicstyle=\ttfamily\scriptsize,
        frame=single,
        commentstyle=,
        backgroundcolor=\color{offwhite},
        showspaces=false,
        showstringspaces=false
        }


% a few macros
\newcommand{\bi}{\begin{itemize}}
\newcommand{\bbi}{\vspace{24pt} \begin{itemize} \itemsep8pt}
\newcommand{\ei}{\end{itemize}}
\newcommand{\be}{\begin{enumerate}}
\newcommand{\bbe}{\vspace{24pt} \begin{enumerate} \itemsep8pt}
\newcommand{\ee}{\end{enumerate}}
\newcommand{\ig}{\includegraphics}
\newcommand{\subt}[1]{{\footnotesize \color{subtitle} {#1}}}
\newcommand{\ttsm}{\tt \small}
\newcommand{\ttfn}{\tt \footnotesize}
\newcommand{\figh}[2]{\centerline{\includegraphics[height=#2\textheight]{#1}}}
\newcommand{\figw}[2]{\centerline{\includegraphics[width=#2\textwidth]{#1}}}


%%%%%%%%%%%%%%%%%%%%%%%%%%%%%%%%%%%%%%%%%%%%%%%%%%%%%%%%%%%%%%%%%%%%%%
% end of header
%%%%%%%%%%%%%%%%%%%%%%%%%%%%%%%%%%%%%%%%%%%%%%%%%%%%%%%%%%%%%%%%%%%%%%

% title info
\title{BMI 826}
\subtitle{Advanced Data Analysis}
\author{\href{https://kbroman.org}{Karl Broman}}
\institute{Biostatistics \& Medical Informatics, UW{\textendash}Madison}
\date{\href{https://kbroman.org}{\tt \scriptsize \color{foreground} kbroman.org}
\\[-4pt]
\href{https://github.com/kbroman}{\tt \scriptsize \color{foreground} github.com/kbroman}
\\[-4pt]
\href{https://twitter.com/kwbroman}{\tt \scriptsize \color{foreground} @kwbroman}
\\[-4pt]
{\scriptsize Course web: \href{https://kbroman.org/AdvData}{\tt kbroman.org/AdvData}}
}


\begin{document}

% title slide
{
\setbeamertemplate{footline}{} % no page number here
\frame{
  \titlepage

  \note{This is the introductory lecture for a special topics course at
    UW{\textendash}Madison on advanced data analysis. It might be
    better called ``The craft of data analysis.''

    Data analysis involves both a set of skills and a way of thinking
    about and looking at data. It is critical to consider the
    scientific context of the data, and to focus on the scientific
    questions for which data were gathered.

    And so the course has three streams: first, a set of case studies
    chosen to illustrate important lessons; second, a set of
    tutorials to introduce certain skills useful for developing and
    organizing reproducible data analyses; and third, homework
    assignments that involve direct, guided analyses of data.
}
} }





\begin{frame}{What is data analysis?}

\vspace*{-5mm}

\onslide<2->{\bbi}
\onslide<2->{\item Answer questions with data}
\onslide<3->{\item Identify/develop appropriate methods to do so}
\onslide<4->{\item Quantify uncertainty}
\onslide<4->{\item Assess appropriateness of the method}
\onslide<4->{\item Identify problems in the data}
\onslide<4->{\item Understand where the data came from and possible
  biases or other limitations}
\onslide<4->{\item Manage and organize data}
\onslide<4->{\item Manage/organize/develop/test software and analyses
  so they are reproducible and correct}
\onslide<4->{\item Communicate/present the results}
\onslide<2->{\ei}

\note{
  What is data analysis? It is a lot of things. First, of course, it
  is an effort to use data to answer questions. But also it involves
  identifying appropriate methods to answer questions with data, or
  developing new methods if such methods don't exist. And I would
  assert that it is always important to attempt to quantify the
  uncertainty in our answers.

  We also need to be able to assess the appropriateness of
  methods, to identify problems in the data, and to understand where
  the data come from and any biases and other limitations in the data.

  Further, good data analysis includes methods for managing and
  organizing data, and for managing, organizing, developing, and
  testing software and data analyses so that they are reproducible and
  correct.

  Finally, data analysis includes the communication and presentation
  of the results, in a form appropriate to the audience of the work.
}

\end{frame}


\begin{frame}{Important principles}

\bbe
\item You'll never know all the methods

\item Focus on the question and data, not the method

\item ``{\hilit Because you can}'' is not a good reason to do something
\ee

\note{
  Here are some key principles that I live by.

  Courses focused on methods will always be incomplete and can quickly
  become outdated. And so that's led me, in this course, to focus on
  my general approach data analysis, and the various lessons and
  principles I've acquired over time.
}

\end{frame}



\begin{frame}{This course}

\bbi
\item Data analysis projects

\item Tools for organizing analyses so that they are reproducible

\item Stories of data analysis projects, with lessons
\ei

\note{
  Learning in this course will come from three streams of effort.

  First, homework assignments that give direct experience in data
  analysis.

  Second, explicit direct instruction in tools for organizing data
  analyses so that they are reproducible.

  And finally, stories of past data analysis projects, with lessons
  I've learned.
}

\end{frame}



\begin{frame}[c]{Lesson 1}


\centerline{\Large Follow up artifacts}

\bigskip \bigskip

\centerline{\large \hilit They might be the most interesting results}


\note{
  Today, I'll provide a specific case study, with the key lesson
  being, ``Follow up artifacts, as they might be the most interesting
  of your results.''
}

\end{frame}


\begin{frame}[c]{}

\figh{Figs/mfdmaps_paper.png}{0.9}

\note{
  After I finished my PhD, I did a postdoc with a geneticist, Jim
  Weber, at the Marshfield Clinic. My central project was to develop
  new human genetic maps.
}

\end{frame}



\begin{frame}[c]{Eucalypt genetic map}

\figh{Figs/eucalypt_map.pdf}{0.75}

\vspace{5mm}

\hfill {\lolit \scriptsize
Byrne et al., Theor Appl Genet 91:869--875, 1995}

\note{
  A genetic map specifies the order of a set of markers along
  chromosomes.

  This is part of a genetic map for eucalyptus trees. It is the first
  map that I had looked at in detail.

  The original genetic maps were for observeable mutations, in
  Drosophila (fruit flies). Later markers were more directly
  DNA-based, and really chosen due to the convenience of measurement.
}

\end{frame}



\begin{frame}[c]{Meiosis}

\figh{Figs/meiosis.png}{0.95}

\note{
  Distances on a genetic map are according to recombination at
  meiosis. Meiosis is the cell division process that produces sperm
  and egg cells. DNA duplicates, and then homologous chromosomes find
  each other and become intimately associated with each other and then
  actually exchange material at locations called chiasmata. Two cell
  divisions later you have gametes with one copy of each chromosome,
  which will generally be mosaics of the original chromosomes, with
  the points of exchange called crossovers.

  Distance on a genetic map is measured by the frequency of
  crossovers. Two points are d cM apart if there is an average of d
  crossovers in the interval per 100 meiotic products.
}

\end{frame}





\begin{frame}{Ordering markers}

\vspace{5mm}

$$
\begin{array}{c|} \mathsf{A} \\ \mathsf{B} \\ \mathsf{C} \end{array}
\left. \enspace
\begin{array}{|c} \mathsf{a} \\ \mathsf{b} \\ \mathsf{c} \end{array}
\right.
\hspace*{1cm} \longrightarrow \hspace*{1cm}
\begin{array}{ccc} \mathsf{ABC} & \hspace*{3mm} & \mathsf{abc} \\
\only<3>{\vhilit} \mathsf{ABc} & \hspace*{3mm} & \only<3>{\vhilit} \mathsf{abC} \\
\only<4>{\vhilit} \mathsf{Abc} & \hspace*{3mm} & \only<4>{\vhilit} \mathsf{aBC} \\
\only<2>{\vhilit} \mathsf{AbC} & \hspace*{3mm} & \only<2>{\vhilit}\mathsf{aBc} \end{array}
$$

\vspace{5mm}
\begin{center}

{\hilit Marker orders: \hspace{5mm}
{ \only<2>{ \vhilit } A--B--C } \hspace{5mm} { \only<3>{ \vhilit }
  A--C--B } \hspace{5mm} { \only<4>{ \vhilit } B--A--C }}

\vspace{15mm}

\onslide<5>{
With {\hilit $\mathsf{M}$} markers, there are {\hilit
  $\mathsf{M}!/\mathsf{2}$} possible orderings.

\vspace{3mm}

For {\hilit $\mathsf{M = {\vhilit 100}}$}, {\hilit $\mathsf{M!/2
    \approx {\vhilit 10^{157}}}$}
}

\end{center}

\note{
  We can use this sort of marker information to order markers along
  chromosomes. Consider a case where the parent cell has ABC on one
  chromosome and abc on the other chromosome. There are 8 possible
  daughter cells. The {\hilit least} frequent of them will indicate
  which of the three possible marker orders is the true one.

  With M markers, there are M!/2 possible orderings, which is too many
  to consider exhaustively, so we need heuristic methods to try to
  find the good ones.
}

\end{frame}



\begin{frame}[c]{CEPH pedigrees}

\figw{Figs/ceph_pedigrees.pdf}{1.0}

\note{
   In my postdoc, I focused on data on a set of large 8 human
   families. A mother/father pair with 10-15 offspring. Most of the
   families also included data on the grandparents.
}

\end{frame}




\begin{frame}{Marshfield genetic maps: Tasks}

\bbi
\item Assemble data

\item Understand marker names

  \bi
  \item[] AFM, UT, CHLC (GATA etc.), Mfd, D*S*
  \ei

\item Identify cryptic duplicates

\item Order markers and identify genotyping errors

  \bi
  \item[] Removed 764 / 969,425 genotypes
  \ei
\ei

\note{
  The main tasks were to assemble genotype data from multiple sources,
  get an understanding of the different marker names, figure out which
  markers were actually duplicates even though they didn't look like
  duplicates, and the {\hilit mostly} to determine the order of the
  markers while simultaneously identifying genotyping errors. In the
  end I removed 765 out of nearly one million genotypes as likely errors.
}

\end{frame}






\begin{frame}[c]{CRIMAP chrompic}

\begin{center}

{\fontsize{6.9pt}{7.6}\selectfont \tt

1332-03 ma {\color{purple} -11-i--11--111-i111-11-1111i--1111i-1111-i--11---1--11-1111-1-1i1---1}... \\
1332-03 pa {\color{green} 0000----0000000o00o00-000-000-0000o00-000-00000-0000}{\color{purple} 1}---{\color{green} 000-00-o000-0}... \\[6pt]
1332-04 ma {\color{purple} -11-i--11--111-1111-11-i111i--i1111-1111-i--11---1--11-1111-1-11i--11}... \\
1332-04 pa {\color{purple} 1111----1111111111i11-1i1-111-i111i11-111-11111-11111---111-11-1i1111}... \\[6pt]
1332-05 ma {\color{purple} -11-i--11--111-i111-11-1111}{\color{green} o--0000o-0000-o--00---0--00-0000-0-0o0--00}... \\
1332-05 pa {\color{green} 0000----0000000o00o00-000}-{\color{purple} 111-1111i11-111-1111--11111---111-11-i11111}... \\[6pt]
1332-06 ma {\color{green} -00-o--00--000-o000-00-0000o--0000o-0000-o--00---0--00-0000}-{\color{purple} 1-11i--11}... \\
1332-06 pa {\color{purple} 1111----1111111i11i11-111-111-1111i11-111-11111-11111---111-11-1i1111}... \\[6pt]
1332-07 ma {\color{green} -00-o--00--000-o000-00-0000o--0000o-0000-o--00---0--00-0000-0-0o0--00}... \\
1332-07 pa {\color{purple} 1111----1111111i11i11-111-111-1111i11-111-1111--11111---111-11-i11111}... \\[6pt]
1332-08 ma {\color{purple} -1}{\color{green} 0-o--00--000-00-0-00-0000o--o0000-0000-o--00---0}--{\color{purple} 11-1111-1-1i1--11}... \\
1332-08 pa {\color{green} 0000----000000000-o00-0}{\color{purple} 1}{\color{green} 0-000-o000o00-000-00000-00000---000-00-o00000}... \\[6pt]
1332-10 ma {\color{purple} -11-i--1---111-i111-11-1111i--1111i-1111-i--11---1--11-1111-1-1i1--11}... \\
1332-10 pa {\color{purple} 1}{\color{green} 000-----000000o00o00-000-000-0000o00-000-00000-00000---000-00-o00000}... \\[6pt]
1332-11 ma {\color{purple} -11}-{\color{green} o--00--000-o000-00-0000o--0000o-0000-o--00---0--00-0000-0-0o0--00}... \\
1332-11 pa {\color{purple} 1111----1111111i11i11-111-111-1111i11-111-11111-11111---111-11-i11111}... \\[6pt]
1332-12 ma {\color{green} -00}-{\color{purple} i--11--111-i111-11---11i--1111i-1111-i--11---1--11-1111-1-1i1---1}... \\
1332-12 pa {\color{green} 0000----0000000o00o00-0---000-0000o00-000-00000-00000---000-00-o000-0}... \\[6pt]
1332-17 ma {\color{purple} -11-i--1---11--i111-1--1111i--1111i-1111-i--11---1--11-11}{\color{green} 00-0-00o--00}... \\
1332-17 pa {\color{green} 0000-----0000--o00o00-000-000-0000o-0-000-0000--00000---000-00-0o0000}...

}

\end{center}


\note{
    I spent a lot of time looking at output like this, from the software
    CRIMAP which I had used to order and Q/C the data.

    Each row is one chromosome (``ma'' for maternal and ``pa'' for
    paternal). The 1's and 0's indicate grandfather's and grandmother's
    DNA; the dashes mean indeterminate. The i's and o's mean the
    grandparents' genotypes weren't informative and so grandparental
    origin was determined based on surrounding markers.
}


\end{frame}

\begin{frame}[c]{CRIMAP chrompic}

\begin{center}

{\fontsize{6.9pt}{7.6}\selectfont \tt

1332-03 ma {\lolit -11-i--11--111-i111-11-1111i--1111i-1111-i--11---1--11-1111-1-1i1---1}... \\
1332-03 pa {\lolit 0000----0000000o00o00-000-000-0000o00-000-00000-0000}{\vhilit 1}{\lolit ---000-00-o000-0}... \\[6pt]
1332-04 ma {\lolit -11-i--11--111-1111-11-i111i--i1111-1111-i--11---1--11-1111-1-11i--11}... \\
1332-04 pa {\lolit 1111----1111111111i11-1i1-111-i111i11-111-11111-11111---111-11-1i1111}... \\[6pt]
1332-05 ma {\lolit -11-i--11--111-i111-11-1111}{\lolit o--0000o-0000-o--00---0--00-0000-0-0o0--00}... \\
1332-05 pa {\lolit 0000----0000000o00o00-000-}{\lolit 111-1111i11-111-1111--11111---111-11-i11111}... \\[6pt]
1332-06 ma {\lolit -00-o--00--000-o000-00-0000o--0000o-0000-o--00---0--00-0000}{\lolit -}{\lolit 1-11i--11}... \\
1332-06 pa {\lolit 1111----1111111i11i11-111-111-1111i11-111-11111-11111---111-11-1i1111}... \\[6pt]
1332-07 ma {\lolit -00-o--00--000-o000-00-0000o--0000o-0000-o--00---0--00-0000-0-0o0--00}... \\
1332-07 pa {\lolit 1111----1111111i11i11-111-111-1111i11-111-1111--11111---111-11-i11111}... \\[6pt]
1332-08 ma {\lolit -}{\vhilit 1}{\lolit 0-o--00--000-00-0-00-0000o--o0000-0000-o--00---0--}{\lolit 11-1111-1-1i1--11}... \\
1332-08 pa {\lolit 0000----000000000-o00-0}{\vhilit 1}{\lolit 0-000-o000o00-000-00000-00000---000-00-o00000}... \\[6pt]
1332-10 ma {\lolit -11-i--1---111-i111-11-1111i--1111i-1111-i--11---1--11-1111-1-1i1--11}... \\
1332-10 pa {\vhilit 1}{\lolit 000-----000000o00o00-000-000-0000o00-000-00000-00000---000-00-o00000}... \\[6pt]
1332-11 ma {\lolit -11-}{\lolit o--00--000-o000-00-0000o--0000o-0000-o--00---0--00-0000-0-0o0--00}... \\
1332-11 pa {\lolit 1111----1111111i11i11-111-111-1111i11-111-11111-11111---111-11-i11111}... \\[6pt]
1332-12 ma {\lolit -00-}{\lolit i--11--111-i111-11---11i--1111i-1111-i--11---1--11-1111-1-1i1---1}... \\
1332-12 pa {\lolit 0000----0000000o00o00-0---000-0000o00-000-00000-00000---000-00-o000-0}... \\[6pt]
1332-17 ma {\lolit -11-i--1---11--i111-1--1111i--1111i-1111-i--11---1--11-11}{\lolit 00-0-00o--00}... \\
1332-17 pa {\lolit 0000-----0000--o00o00-000-000-0000o-0-000-0000--00000---000-00-0o0000}...

}

\end{center}

\note{
    I spent most of my time hunting for misplaced 1's amidst surrounding 0's, or
    misplaced 0's amidst surrounding 1's, and then trying to decide if it
    was an error that should be deleted, or if the marker should maybe be
    moved somewhere else.
}

\end{frame}


\begin{frame}[fragile]{Top of chr 22}

\begin{center}

{\fontsize{8pt}{9.5}\selectfont

\begin{verbatim}

 Marker       Dnumber     sex-ave(cM)       female(cM)        male(cM)

1 ATA2G02     Unknown             0.00             0.00             0.00
                          1.79             0.00             2.60
2 GATA198B05  Unknown             1.79             0.00             2.60
                          2.27             3.32             0.00
3 AFM217xf4   D22S420             4.06             3.32             2.60
                          4.26             4.51             5.42
4 AFM288we5   D22S427             8.32             7.83             8.02
                          5.25             7.52             3.00
5 265yf5      D22S425            13.57            15.35            11.02
                          0.03             0.00             0.65
6 GGAA10F06   D22S686            13.60            15.35            11.67
                          0.84             0.00             0.82
7 AFMa037zd1  D22S539            14.44            15.35            12.49
                          0.00             0.00             0.00
8 AFM292va9   D22S446            14.44            15.35            12.49
                          3.27             5.91             0.00
9 Mfd51       D22S257            17.71            21.26            12.49
\end{verbatim}

}
\end{center}

\note{
    The result of the work was a set of plain text files with the
    marker positions on each chromosome.

    Perhaps surprising is that much of the positive feedback I got
    about the work was really about the ease of use of these
    plain-text files. Most other genetic maps were distributed as
    images rather than providing the direct data.
}

\end{frame}


\begin{frame}[c]{Marker search}

\figh{Figs/markersearch.jpg}{0.8}

\note{
    I also created a simple perl script and web form, where you could
    paste in a set of marker names, and it would pull out the
    locations of just those markers.

    This was hardly any work for me, but it was hugely useful to the
    community. And probably the most important thing I learned from
    this project is that it's these little things that can have the
    biggest impact.
}

\end{frame}






\begin{frame}{10th worst graph}


\figh{Figs/recrate.jpg}{0.85}

\vspace{0mm}

\hfill {\scriptsize \lolit
Broman et al., Am J Hum Genet 63:861--869, 1998}

\note{
   My interest in this project was not so much in the genetic maps as
   in what they tell us about the recombination process at meiosis.

   This graph shows the relative rates of recombination in females vs
   males. Female recombination is generally higher, but it varies a
   great deal between and along chromosomes, and at the ends of the
   chromosome, males tend to have higher recombination.

   I call this the ``10th worst graph'' because I had included it on a
   web page of the ``top ten worst graphs in the scientific
   literature.''  The problem here is that I'm plotting a ratio, and
   it over-emphasizes where female recombination is greater than male
   (which stretches from 1 to infinity) and under-emphasizes where
   male recombination is greater than female (sandwiched between 0 and
   1).

   This plot should have been on a log scale, and really whenever I
   submit a paper I do a quick search for the word ``which'' and see
   if I should change it to ``that,'' and then I look at the plots and
   see if they would be better on the log scale.

   When measurements span multiple orders of magnitude, you should
   probably take logs. And for ratios, you almost surely want to take
   logs. And I recommend log base 2, because you can multiply by 2
   easily (2, 4, 8, 16, ...) and because the values are closer
   together than 10's.
}

\end{frame}



\begin{frame}{Total no. crossovers}

\vspace{6mm}

\figw{Figs/totalXO.jpg}{1.0}

\vspace{10mm}


\hfill {\scriptsize \lolit
Broman et al., Am J Hum Genet 63:861--869, 1998}

\note{
  Another interesting thing we learned was about individual variation
  in recombination rates. If you count up the total number of
  recombination events in each egg and in each sperm that went on to
  be the children in these families, you notice that there is
  remarkable variation among women in their recombination rate, but
  little variation among the men.

  Note also here the huge difference in the overall rate between
  mothers and fathers: an egg has an average of about 40 crossovers,
  where a sperm cell has more like 23.
}

\end{frame}



\begin{frame}{Crossover locations}

\vspace{2mm}

\figh{Figs/xoloc.jpg}{0.75}

\vspace{4mm}

\hfill {\scriptsize \lolit
Broman and Weber, Am J Hum Genet 66:1911--1926, 2000}

\note{
   What I was really interested in was crossover interference: the
   tendency of the crossovers to not be too close together on
   chromosomes. The open and hatched segments here are the
   grandmother's and grandfather's DNA, and the black bars are the
   intervals in which crossovers occurred. We can't determine the
   crossover locations exactly, the data are ``interval censored.''

   So the next thing I was going to look at was this dependence in
   crossover locations.

  In thinking about how best to handle that interval censoring, I was
  reminded of some cases where there were really large,
  non-informative intervals.
}

\end{frame}




\begin{frame}[c]{Family 884, chr 6}


\figw{Figs/autozygosity.pdf}{1.0}

\note{
  These are the maternal and paternal chromosomes 6 in one family. The
  purple and green are the grandmother and grandfather DNA; the gray
  is indeterminate.

  Note the big chunk of gray on the maternal chromosomes, and the two
  big chunks on the paternal chromosomes. What could be causing that?

  We basically have long stretches of markers where the mother or
  father is homozygous, where the haplotypes they got from their
  parents were the same.
}

\end{frame}



\begin{frame}[c]{}

\figh{Figs/autozyg_paper.png}{0.9}

\note{
  This is what is called ``autozygosity.'' Where an individual is
  homozygous for a region because their parents are related, perhaps
  distantly, and they received two copies of the same bit of DNA,
  inherit from some ancestor, twice.

  Have realized what I was looking at, I went and looked for all other
  such regions, and found bunches of them in these families.
}

\end{frame}


\begin{frame}{Autozygosity}

  \vspace{4mm}

\figw{Figs/autozyg_table.jpg}{1.0}

\vspace{8mm}

\hfill {\scriptsize \lolit Broman and Weber, Am J Hum Genet 65:1493--1500, 1999}

\note{
  For example, here's a single individual who has autozygous regions
  of various sizes on ten different chromosomes.
}

\end{frame}



\begin{frame}[c]{}

\figh{Figs/xoi_paper.png}{0.9}

\note{
  I did then get to my analysis of crossover interference (the
  tendency of crossovers to not be too close together).
}

\end{frame}






\begin{frame}{Crossover interference}

\figh{Figs/xodist.jpg}{0.8}

\vspace{3mm}

\hfill {\scriptsize \lolit Broman and Weber, Am J Hum Genet 66:1911--1926, 2000}

\note{
   A main part of the result concerned fitting different models to the
   inter-crossover distance data.  One model fit much better than others.
}

\end{frame}



\begin{frame}[c]{Maternal chr 8}
\figw{Figs/chr8m.png}{1.0}

\note{
  But on one particular chromosome (maternal chromosome 8), my
  favorite model really didn't fit well at all.
}

\end{frame}



\begin{frame}{Apparent triple XOs}

\figh{Figs/inversion_genotypes.jpg}{0.83}

\vspace{2mm}

\hfill {\scriptsize \lolit Broman et al., In: \emph{Science and Statistics: A Festschrift for Terry Speed}, 2003}

\note{
  I could have just left it at that, but I was curious about what was
  going on, and in studying the problem, I found that there were two
  families that showed an apparent triple-crossover event in a small
  region. This really shouldn't happen.

  My initial reaction was that I had the marker order messed up; if I
  were to invert this region, the triple crossovers would become
  single crossovers.

  But there were other families that showed a crossover in the region.
  If I invert the region, these single crossovers will become triple
  crossovers.

  So then I thought: suppose the region is inverted in these two
  families but not in the other families? This was a pretty crazy
  idea, because the region is quite large (12 cM, which turned out to
  be about 5 Mbp), and we would need individuals to be homozygous for
  each of the two orientations to have recombination occur.

  So a crazy idea: a very long inversion polymorphism where the two
  orientations were each reasonably common.
}

\end{frame}


\begin{frame}{Chr 8p inversion}

\figh{Figs/inversion_fish.jpg}{0.83}

\vspace{2mm}

\hfill {\scriptsize \lolit Broman et al., In: \emph{Science and Statistics: A Festschrift for Terry Speed}, 2003}

\note{
  I posed the hypothesis to my postdoc advisor, who talked to a friend
  whose lab had the ability to investigate this sort of thing, and
  sure enough, we had discovered the largest common inversion
  polymorphism in the human genome.

  This picture shows chromosome 8 with the green and red lighting up
  the two ends of the region. On the left, green is above red on both
  chromosomes. On the right, red is above green on both chromosomes,
  and in the middle green is above red on one chromosome and red is
  above green on the other.

  So this is the best possible example of the importance of following
  up artifacts. Lack of model fit for a particular chromosome led me
  to investigate the cause of the problem, which led me to postulate
  this idea of an inversion polymorphism, which really seemed
  kind of crazy at the time.  But it turned out to be real, and it's
  the coolest thing I've discovered in all my work as a data
  scientist.
}

\end{frame}



\begin{frame}{Comparison to sequence}


\vspace{4mm}

\figh{Figs/matise_fig.jpg}{0.7}

\vspace{6mm}

\hfill {\scriptsize \lolit Matise et al., Am J Hum Genet 70:1398--1410, 2002}

\note{
  A few years later, when the human genome sequence was available,
  there was a paper that investigated the order of markers in my maps.
  I had a few rather embarrassing errors, mostly due to markers whose
  locations were poorly resolved as they had only been genotyped in a
  subset of the families.

  I didn't provide any measures of uncertainty, and it would have been
  good to at least flag the markers that were especially uncertain.
}

\end{frame}


\begin{frame}{Comparison to sequence}

\vspace{5mm}

\figh{Figs/chr22rev.pdf}{0.7}

\note{
   Funny thing though; it turned out that 10 years later, the worst of
   those problems were seen to be not a problem with my genetic map,
   but rather with the initial human genome sequence. I still made
   some mistakes, but the biggest mistake on that previous slide was
   likely in the initial draft of the human genome sequence.
}

\end{frame}








\begin{frame}[c]{Lesson 1}


\centerline{\Large Follow up artifacts}

\bigskip \bigskip

\centerline{\large \hilit They might be the most interesting results}

\note{
  Two key lessons here.

  First, follow up artifacts, as they can be the most important
  findings. Here, the autozygosity, and then the large common inversion.
}

\end{frame}




\begin{frame}[c]{Lesson 2}


  \centering

\Large The simplest things \\[4pt]
    can be the most important

\note{
  Second, the simplest things can be the most important. In this
  project, most of the positive feedback I got were due to way in
  which I provided the results, plus that I took a little time to
  create a web form for searching for markers.
}

\end{frame}



\end{document}
