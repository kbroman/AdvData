\documentclass[aspectratio=169,12pt,t]{beamer}
\usepackage{graphicx}
\setbeameroption{hide notes}
\setbeamertemplate{note page}[plain]
\usepackage{listings}

\input{../LaTeX/header.tex}

%%%%%%%%%%%%%%%%%%%%%%%%%%%%%%%%%%%%%%%%%%%%%%%%%%%%%%%%%%%%%%%%%%%%%%
% end of header
%%%%%%%%%%%%%%%%%%%%%%%%%%%%%%%%%%%%%%%%%%%%%%%%%%%%%%%%%%%%%%%%%%%%%%

\title{Containers for reproducibility}
\author{\href{https://kbroman.org}{Karl Broman}}
\institute{Biostatistics \& Medical Informatics, UW{\textendash}Madison}
\date{\href{https://kbroman.org}{\tt \scriptsize \color{foreground} kbroman.org}
\\[-4pt]
\href{https://github.com/kbroman}{\tt \scriptsize \color{foreground} github.com/kbroman}
\\[-4pt]
\href{https://twitter.com/kwbroman}{\tt \scriptsize \color{foreground} @kwbroman}
\\[-4pt]
{\scriptsize Course web: \href{https://kbroman.org/AdvData}{\tt kbroman.org/AdvData}}
}

\begin{document}

{
\setbeamertemplate{footline}{} % no page number here
\frame{
  \titlepage

\note{
  In this lecture, we'll look at the how to keep track of dependencies
  to further enhance reproducible research. In particular, we will
  focus on Docker containers.
}
} }





\begin{frame}[c]{Reproducible research}

\begin{quotation}
  organize the data and code in a way \\[4pt]
  that you can hand them to someone else \\[4pt]
  and they can re-run the code \\[4pt]
  and get the same results \\[4pt]
  \quad (the same figures and tables)
\end{quotation}

\note{
  A central theme of this course has been reproducible
  research: organizational strategies and tools so that your
  computational work can be reproduced.
}
\end{frame}







\begin{frame}{Dependency Hell}

\bbi
  \item What software does your project depend on?
    \bi
  \item operating system
  \item system libraries
  \item R or python
  \item packages or modules
  \item other tools (e.g. pandoc and \LaTeX)
    \ei
  \item Can you install all necessary dependencies?
  \item Have dependencies changed? Do you need particular versions?
  \item How much time does it take to set things up?
\ei

\note{
  We have not much touched on how to keep track of dependencies. This
  can be a major challenge. In some research areas, things are
  evolving rapidly. And the more tools you use, the greater chance
  that some will change over time and that things will be broken.

  Even if you don't rely on much other software, it can still be a
  huge pain for users to collect and install all of the necessary
  tools. So much so that they end up giving up.
}
\end{frame}






\begin{frame}{Capturing dependencies}

\bbi
  \item R: \href{https://rstudio.github.io/renv}{\tt renv}
    \bi
  \item[] {\tt renv::init()}
  \item[] {\tt renv::snapshot()}
  \item[] {\tt renv::restore()}
    \bigskip

  \item[] \hspace{-8mm} {\color{foreground} Also see} \href{https://mran.microsoft.com/MRAN}{MRAN}
    \ei

\bigskip

  \item Python: \href{https://docs.conda.io}{\tt conda}
    \bi
    \item[] {\tt conda create}
    \item[] {\tt conda install}
    \item[] {\tt conda activate}
    \item[] {\tt conda env list --explicit}
    \ei

\ei

\note{
   These best ways to keep track of dependencies: for R, use the renv
   package from RStudio. For python, use conda.

   Also for R, see MRAN: Microsoft is taking daily snapshots of CRAN.

}
\end{frame}




\begin{frame}{Or create package/module}

  \bbi
\item R package
  \bi
\item dependencies in {\tt DESCRIPTION} file
\item data in {\tt inst/ext\_data}
\item analyses as vignettes
  \ei
\item Python module
  \bi
\item dependencies?
  \ei
  \ei

  \note{
    Another approach would be to make your analysis project an R
    package, identifying all of the package dependencies, including
    the data, and with the analyses included as vignettes.

    I'm not sure what the corresponding Python module might look like,
    but it seems like a python module could at least document the
    appropriate dependencies in a structured way.
  }

\end{frame}



\end{document}
