\documentclass[aspectratio=169,12pt,t]{beamer}
\usepackage{graphicx}
\setbeameroption{hide notes}
\setbeamertemplate{note page}[plain]
\usepackage{listings}

\input{../LaTeX/header.tex}

%%%%%%%%%%%%%%%%%%%%%%%%%%%%%%%%%%%%%%%%%%%%%%%%%%%%%%%%%%%%%%%%%%%%%%
% end of header
%%%%%%%%%%%%%%%%%%%%%%%%%%%%%%%%%%%%%%%%%%%%%%%%%%%%%%%%%%%%%%%%%%%%%%

\title{The bootstrap}
\subtitle{Confidence intervals for QTL location}
\author{\href{https://kbroman.org}{Karl Broman}}
\institute{Biostatistics \& Medical Informatics, UW{\textendash}Madison}
\date{\href{https://kbroman.org}{\tt \scriptsize \color{foreground} kbroman.org}
\\[-4pt]
\href{https://github.com/kbroman}{\tt \scriptsize \color{foreground} github.com/kbroman}
\\[-4pt]
\href{https://twitter.com/kwbroman}{\tt \scriptsize \color{foreground} @kwbroman}
\\[-4pt]
{\scriptsize Course web: \href{https://kbroman.org/AdvData}{\tt kbroman.org/AdvData}}
}

\begin{document}

{
\setbeamertemplate{footline}{} % no page number here
\frame{
  \titlepage

\note{
  In this lecture, we'll look at the bootstrap; a method to get
  standard errors and confidence intervals by resampling one's own
  data. But then we'll proceed to an example where the bootstrap
  performs terribly.
}
} }


\begin{frame}[c]{}

  \figh{Figs/bootstrap_cover.jpg}{0.9}

\note{
  The bootstrap is crazy useful. Efron has a big book about it, but
  I'd start with this one.

  As I've emphasized before, computer simulation is really useful. The
  bootstrap can be seen as an extension of that point.
}

\end{frame}




\begin{frame}{}

\figh{Figs/visscher1996.png}{0.9}

\note{
}
\end{frame}



\begin{frame}{LOD support interval}


\figh{Figs/hyper_lod.pdf}{0.85}

\vfill

\hfill {\footnotesize \lolit Data from Sugiyama et al.\ (2001) Genomics 71:70--77}

\note{
}

\end{frame}




\begin{frame}{Approximate Bayes interval}


\figh{Figs/hyper_bayes.pdf}{0.85}

\vfill

\hfill {\footnotesize \lolit Data from Sugiyama et al.\ (2001) Genomics 71:70--77}

\note{
}
\end{frame}



\begin{frame}{Bootstrap CI}


\figh{Figs/hyper_boot.pdf}{0.85}

\vfill

\hfill {\footnotesize \lolit Data from Sugiyama et al.\ (2001) Genomics 71:70--77}

\note{
}
\end{frame}











\begin{frame}{}

\figh{Figs/manichaikul2006.png}{0.9}

\note{
}
\end{frame}



\begin{frame}{Simulation study}


  \vspace*{-5mm}

  \bbi
  \item {\hilit Backcross}, 200 individuals

\item {\lolit One chromosome of length 100 cM }

\item Markers at {\hilit 10 cM spacing}

\item {\hilit Single QTL} responsible for 10\% of phenotypic variance

\item {\lolit Normally distributed residual variation }

\item {\hilit Varied location} of QTL, at positions 0, 1, \dots, 100 cM

\item {\lolit Analysis by standard interval mapping;
  calculations every 1 cM}

\item {\lolit 10,000 simulations for each QTL position}

\item {\lolit Bootstrap used 1000 replicates }
\ei

\note{
}
\end{frame}


\end{document}
