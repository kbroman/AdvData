\documentclass[aspectratio=169,12pt,t]{beamer}
\usepackage{graphicx}
\setbeameroption{hide notes}
\setbeamertemplate{note page}[plain]
\usepackage{listings}

\input{../LaTeX/header.tex}

%%%%%%%%%%%%%%%%%%%%%%%%%%%%%%%%%%%%%%%%%%%%%%%%%%%%%%%%%%%%%%%%%%%%%%
% end of header
%%%%%%%%%%%%%%%%%%%%%%%%%%%%%%%%%%%%%%%%%%%%%%%%%%%%%%%%%%%%%%%%%%%%%%

\title{The bootstrap}
\subtitle{Confidence intervals for QTL location}
\author{\href{https://kbroman.org}{Karl Broman}}
\institute{Biostatistics \& Medical Informatics, UW{\textendash}Madison}
\date{\href{https://kbroman.org}{\tt \scriptsize \color{foreground} kbroman.org}
\\[-4pt]
\href{https://github.com/kbroman}{\tt \scriptsize \color{foreground} github.com/kbroman}
\\[-4pt]
\href{https://twitter.com/kwbroman}{\tt \scriptsize \color{foreground} @kwbroman}
\\[-4pt]
{\scriptsize Course web: \href{https://kbroman.org/AdvData}{\tt kbroman.org/AdvData}}
}

\begin{document}

{
\setbeamertemplate{footline}{} % no page number here
\frame{
  \titlepage

\note{
  In this lecture, we'll look at the bootstrap; a method to get
  standard errors and confidence intervals by resampling one's own
  data. But then we'll proceed to an example where the bootstrap
  performs terribly.
}
} }


\begin{frame}[c]{}

  \figh{Figs/bootstrap_cover.jpg}{0.9}

\note{
  The bootstrap is crazy useful. Efron has a big book about it, but
  I'd start with this one.

  As I've emphasized before, computer simulation is really useful. The
  bootstrap can be seen as an extension of that point.
}

\end{frame}



\end{document}
