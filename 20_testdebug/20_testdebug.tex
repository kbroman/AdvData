\documentclass[aspectratio=169,12pt,t]{beamer}
\usepackage{graphicx}
\setbeameroption{hide notes}
\setbeamertemplate{note page}[plain]
\usepackage{listings}

% header.tex: boring LaTeX/Beamer details + macros

% get rid of junk
\usetheme{default}
\beamertemplatenavigationsymbolsempty
\hypersetup{pdfpagemode=UseNone} % don't show bookmarks on initial view


% font
\usepackage{fontspec}
\setsansfont
  [ ExternalLocation = ../fonts/ ,
    UprightFont = *-regular ,
    BoldFont = *-bold ,
    ItalicFont = *-italic ,
    BoldItalicFont = *-bolditalic ]{texgyreheros}
\setbeamerfont{note page}{family*=pplx,size=\footnotesize} % Palatino for notes
% "TeX Gyre Heros can be used as a replacement for Helvetica"
% I've placed them in fonts/; alternatively you can install them
% permanently on your system as follows:
%     Download http://www.gust.org.pl/projects/e-foundry/tex-gyre/heros/qhv2.004otf.zip
%     In Unix, unzip it into ~/.fonts
%     In Mac, unzip it, double-click the .otf files, and install using "FontBook"

% named colors
\definecolor{offwhite}{RGB}{255,250,240}
\definecolor{gray}{RGB}{155,155,155}
\definecolor{purple}{RGB}{177,13,201}
\definecolor{green}{RGB}{46,204,64}

\definecolor{background}{RGB}{255,255,255}
\definecolor{foreground}{RGB}{24,24,24}
\definecolor{title}{RGB}{27,94,134}
\definecolor{subtitle}{RGB}{22,175,124}
\definecolor{hilit}{RGB}{122,0,128}
\definecolor{vhilit}{RGB}{255,0,128}
\definecolor{codehilit}{RGB}{255,0,128}
\definecolor{lolit}{RGB}{95,95,95}
\definecolor{myyellow}{rgb}{1,1,0.7}
\definecolor{nhilit}{RGB}{128,0,128}  % hilit color in notes
\definecolor{nvhilit}{RGB}{255,0,128} % vhilit for notes

\newcommand{\hilit}{\color{hilit}}
\newcommand{\vhilit}{\color{vhilit}}
\newcommand{\nhilit}{\color{nhilit}}
\newcommand{\nvhilit}{\color{nvhilit}}
\newcommand{\lolit}{\color{lolit}}

% use those colors
\setbeamercolor{titlelike}{fg=title}
\setbeamercolor{subtitle}{fg=subtitle}
\setbeamercolor{institute}{fg=lolit}
\setbeamercolor{normal text}{fg=foreground,bg=background}
\setbeamercolor{item}{fg=foreground} % color of bullets
\setbeamercolor{subitem}{fg=lolit}
\setbeamercolor{itemize/enumerate subbody}{fg=lolit}
\setbeamertemplate{itemize subitem}{{\textendash}}
\setbeamerfont{itemize/enumerate subbody}{size=\footnotesize}
\setbeamerfont{itemize/enumerate subitem}{size=\footnotesize}

% page number
\setbeamertemplate{footline}{%
    \raisebox{5pt}{\makebox[\paperwidth]{\hfill\makebox[20pt]{\lolit
          \scriptsize\insertframenumber}}}\hspace*{5pt}}

% add a bit of space at the top of the notes page
\addtobeamertemplate{note page}{\setlength{\parskip}{12pt}}

% default link color
\hypersetup{colorlinks, urlcolor={hilit}}

\lstset{language=bash,
        basicstyle=\ttfamily\scriptsize,
        frame=single,
        commentstyle=,
        backgroundcolor=\color{offwhite},
        showspaces=false,
        showstringspaces=false
        }


% a few macros
\newcommand{\bi}{\begin{itemize}}
\newcommand{\bbi}{\vspace{24pt} \begin{itemize} \itemsep8pt}
\newcommand{\ei}{\end{itemize}}
\newcommand{\be}{\begin{enumerate}}
\newcommand{\bbe}{\vspace{24pt} \begin{enumerate} \itemsep8pt}
\newcommand{\ee}{\end{enumerate}}
\newcommand{\ig}{\includegraphics}
\newcommand{\subt}[1]{{\footnotesize \color{subtitle} {#1}}}
\newcommand{\ttsm}{\tt \small}
\newcommand{\ttfn}{\tt \footnotesize}
\newcommand{\figh}[2]{\centerline{\includegraphics[height=#2\textheight]{#1}}}
\newcommand{\figw}[2]{\centerline{\includegraphics[width=#2\textwidth]{#1}}}


%%%%%%%%%%%%%%%%%%%%%%%%%%%%%%%%%%%%%%%%%%%%%%%%%%%%%%%%%%%%%%%%%%%%%%
% end of header
%%%%%%%%%%%%%%%%%%%%%%%%%%%%%%%%%%%%%%%%%%%%%%%%%%%%%%%%%%%%%%%%%%%%%%

\title{Testing and debugging}
\author{\href{https://kbroman.org}{Karl Broman}}
\institute{Biostatistics \& Medical Informatics, UW{\textendash}Madison}
\date{\href{https://kbroman.org}{\tt \scriptsize \color{foreground} kbroman.org}
\\[-4pt]
\href{https://github.com/kbroman}{\tt \scriptsize \color{foreground} github.com/kbroman}
\\[-4pt]
\href{https://twitter.com/kwbroman}{\tt \scriptsize \color{foreground} @kwbroman}
\\[-4pt]
{\scriptsize Course web: \href{https://kbroman.org/AdvData}{\tt kbroman.org/AdvData}}
}

\begin{document}

{
\setbeamertemplate{footline}{} % no page number here
\frame{
  \titlepage

\note{We spend a lot of time debugging. We'd spend a lot less time if
  we tested our code properly.

  We want to get the right answers. We can't be sure that we've done
  so without testing our code.

  Set up a formal testing system, so that you can be
  confident in your code, and so that problems are identified and
  corrected early.

  Even with a careful testing system, you'll still spend time
  debugging. Debugging can be frustrating, but the right tools and
  skills can speed the process.
}
} }





\begin{frame}[c]{}

\centering
\vspace{80pt}

I don't need tests; I have users.

\vspace{36pt}

\hfill \lolit {\textendash} me

\note{
  I really did say this.
}
\end{frame}



\begin{frame}[c]{}

\centering
\vspace{80pt}

If you use software that lacks automated tests,\\
you are the tests.

\vspace{36pt}

\hfill \lolit {\textendash} Jenny Bryan

\note{
  Well, it's comforting that Jenny and I agree.
}
\end{frame}





\begin{frame}[c]{}

\centerline{``I tried it, and it worked.''}

\note{This is about the limit of most programmers' testing efforts.

{\nhilit But}: Does it {\nvhilit still} work? Can you reproduce what you did?
With what variety of inputs did you try?
}

\end{frame}



\begin{frame}[c]{}

\centering
\vspace{80pt}

It's not that we don't test our code, \\[8pt]
it's that we don't store our tests \\
so they can be re-run automatically.

\vspace{36pt}

\hfill \lolit {\textendash} Hadley Wickham

\vspace{45pt}
{\footnotesize
\hfill \href{http://journal.r-project.org/archive/2011-1/RJournal_2011-1_Wickham.pdf}{R Journal 3(1):5{\textendash}10, 2011}
}

\note{
  This is from Hadley's paper about his {\tt testthat} package.
}
\end{frame}

\begin{frame}{Types of tests}

\bbi
\onslide<2->{\item Check inputs
  \bi
  \item Stop if the inputs aren't as expected.
  \ei
}
\item Unit tests
  \bi
  \item For each small function: does it give the right results in
  specific cases?
  \ei
\item Integration tests
  \bi
  \item Check that larger multi-function tasks are working.
  \ei
\item Regression tests
  \bi
  \item Compare output to saved results, to check that things that
    worked continue working.
  \ei
\ei

\note{Your first line of defense should be to include checks of the
  inputs to a function: If they don't meet your specifications, you
  should issue an error or warning.
  But that's not really {\nhilit testing}.

  Your main effort should focus on {\nhilit unit tests}. For each
  small function (and your code {\nhilit should} be organized as a
  series of small functions), write small tests to check that the
  function gives the correct output in specific cases.

  In addition, create larger {\nhilit integration tests} to check that
  larger features are working. It's best to construct these as
  {\nhilit regression tests}: compare the output to some saved version
  (e.g. by printing the results and comparing files). This way, if
  some change you've made leads to a change in the results, you'll
  notice it automatically and immediately.
}
\end{frame}




\begin{frame}[c,fragile]{Check inputs}


\begin{lstlisting}
winsorize <-
function(x, q=0.006)
{
  if(!is.numeric(x)) stop("x should be numeric")

  if(!is.numeric(q)) stop("q should be numeric")
  if(length(q) > 1) {
    q <- q[1]
    warning("length(q) > 1; using q[1]")
  }
  if(q < 0 || q > 1) stop("q should be in [0,1]")

  lohi <- quantile(x, c(q, 1-q), na.rm=TRUE)
  if(diff(lohi) < 0) lohi <- rev(lohi)

  x[!is.na(x) & x < lohi[1]] <- lohi[1]
  x[!is.na(x) & x > lohi[2]] <- lohi[2]
  x
}
\end{lstlisting}


\note{
  The {\tt winsorize} function in my R/broman package hadn't included
  any checks that the inputs were okay.

  The simplest thing to do is to include some {\tt if} statements with
  calls to {\tt stop} or {\tt warning}.

  The input {\tt x} is supposed to be a numeric vector, and {\tt q} is
  supposed to be a single number between 0 and 1.
}
\end{frame}




\begin{frame}[c,fragile]{Check inputs}


\begin{lstlisting}
winsorize <-
function(x, q=0.006)
{
  stopifnot(is.numeric(x))
  stopifnot(is.numeric(q), length(q)==1, q>=0, q<=1)

  lohi <- quantile(x, c(q, 1-q), na.rm=TRUE)
  if(diff(lohi) < 0) lohi <- rev(lohi)

  x[!is.na(x) & x < lohi[1]] <- lohi[1]
  x[!is.na(x) & x > lohi[2]] <- lohi[2]
  x
}
\end{lstlisting}


\note{The {\tt stopifnot} function makes this a bit easier.
}
\end{frame}




\begin{frame}[c,fragile]{\href{http://github.com/hadley/assertthat}{\tt assertthat} package}


\begin{lstlisting}
#' import assertthat
winsorize <-
function(x, q=0.006)
{
  if(all(is.na(x)) || is.null(x)) return(x)

  assert_that(is.numeric(x))
  assert_that(is.number(q), q>=0, q<=1)

  lohi <- quantile(x, c(q, 1-q), na.rm=TRUE)
  if(diff(lohi) < 0) lohi <- rev(lohi)

  x[!is.na(x) & x < lohi[1]] <- lohi[1]
  x[!is.na(x) & x > lohi[2]] <- lohi[2]
  x
}
\end{lstlisting}


\note{Hadley Wickham's {\tt assertthat} package adds some functions that
  simplify some of this.

  How is the {\tt assertthat} package used in practice? Look at
  packages which depend on it, such as {\tt dplyr}. Download the
  source for {\tt dplyr} and try {\tt grep assert\_that dplyr/R/*} and
  you'll see a bunch of examples if its use.

  Also try {\tt grep stopifnot dplyr/R/*} and you'll see that both
  are being used.
}
\end{frame}




\begin{frame}{\tt Tests in R packages}

\bbi
\item Examples in {\tt .Rd} files
\item Vignettes
\item {\tt tests/} directory
  \bi
  \item {\tt some\_test.R} and {\tt some\_test.Rout.save}
  \ei
\ei

\vspace{24pt}

\centerline{\hilit {\tt R CMD check} is your friend.}


\note{
  The examples and vignettes should focus on helping users to
  understand how to use the code. But since they get run whenever {\tt
  R CMD check} is run, they also serve as crude tests that the
  package is working. But note that this is only testing for things
  that {\nhilit halt the code}; it's not checking that the code is
  {\nvhilit giving the right answers}.

  Things that you put in the {\tt tests} directory can be more
  rigorous: these are basically regression tests. The R output (in the
  file {\tt some\_test.Rout}) will be compared to a saved version, if
  available.

  If you want your package on CRAN, you'll need all of these tests and
  examples to be {\nhilit fast}, as CRAN runs {\tt R CMD check} on
  every package every day on multiple systems.

  The {\tt R CMD check} system is another important reason for
  assembling R code as a package.
}
\end{frame}


\begin{frame}[c,fragile]{An example example}

\begin{lstlisting}
#' @examples
#' x <- sample(c(1:10, rep(NA, 10), 21:30))
#' winsorize(x, 0.2)
\end{lstlisting}

\note{
  This example doesn't provide a proper {\nhilit test}.  You'll get a
  note if it gives an error, but you don't get any indication about
  whether it's giving the right answer.
}
\end{frame}



\begin{frame}[c,fragile]{A {\tt tests/} example}

\begin{lstlisting}
library(qtl)

# read data
csv <- read.cross("csv", "", "listeria.csv")

# write
write.cross(csv, "csv", filestem="junk")

# read back in
csv2 <- read.cross("csv", "", "junk.csv",
                   genotypes=c("AA", "AB", "BB",
                               "not BB", "not AA"))

# check for a change
comparecrosses(csv, csv2)

unlink("junk.csv")
\end{lstlisting}

\note{
  An advantage of the {\tt tests/} subdirectory is that you can more
  easily test input/output.

  A useful technique here: if you have a pair of functions that are
  the inverse of each other (e.g., write and read), check that if you
  apply one and then the other, you get back to the original.

  Note that {\tt unlink("junk.csv")} deletes the file.
}
\end{frame}



\begin{frame}{\href{http://github.com/hadley/testthat}{\tt testthat} package}

\vspace{-12pt}

\bbi
\item Expectations
 \bi
 \item[] {\tt expect\_equal(10, 10 + 1e-7)}
 \item[] {\tt expect\_identical(10, 10)}
 \item[] {\tt expect\_equivalent(c("one"=1), 1)}
 \item[] {\tt expect\_warning(log(-1))}
 \item[] {\tt expect\_error(1 + "a")}
 \ei
\item Tests
 \bi
 \item[] {\tt test\_that("winsorize small vectors", \{ ... \})}
 \ei
\item Contexts
 \bi
 \item[] {\tt context("Group of related tests")}
 \ei
\item Store tests in {\tt tests/testthat}
\item {\tt tests/testthat.R} file containing
 \bi
 \item[] {\tt library(testthat) \\
              test\_check("{\color{hilit} mypkg}")}
 \ei
\ei


\note{
  The {\tt testthat} package simplifies unit testing of code in
  R packages.

  There are a bunch of functions for defining ``expectations.''
  Basically, for testing whether something worked as expected. (It can
  be good to check that something gives an error when it's supposed to
  give an error.)

  You then define a set of tests, with a character string to explain
  where the problem is, if there is a problem.

  You can group tests into ``contexts.'' When the tests run, that
  character string will be printed, so you can see what part of the
  code is being tested.

  Put your tests in {\tt .R} files within {\tt tests/testthat}. Put
  another file within {\tt tests/} that will ensure that these tests
  are run when you do {\tt R CMD check}.
}
\end{frame}


\begin{frame}[c,fragile]{Example {\tt testthat} test}


\begin{lstlisting}
context("winsorize")

test_that("winsorize works for small vectors", {

  x <-       c(2, 3, 7, 9, 6, NA, 5, 8, NA, 0, 4, 1, 10)
  result1 <- c(2, 3, 7, 9, 6, NA, 5, 8, NA, 1, 4, 1,  9)
  result2 <- c(2, 3, 7, 8, 6, NA, 5, 8, NA, 2, 4, 2,  8)

  expect_identical(winsorize(x, 0.1), result1)
  expect_identical(winsorize(x, 0.2), result2)

})
\end{lstlisting}



\note{
  These are the sort of tests you might do with the testthat
  package. The value of this: finally, we are checking whether the
  code is giving the right answer!

  Ideally, you include tests like this for every function you write.

  It's not really clear that the second test here is needed. If the
  first test is successful, what's the chance that the second will
  fail?
}
\end{frame}


\begin{frame}[c,fragile]{Example {\tt testthat} test}


\begin{lstlisting}
test_that("winsorize works for a long vector", {

  set.seed(94745689)
  n <- 1000
  nmis <- 10
  p <- 0.05
  input <- rnorm(n)
  input[sample(1:n, nmis)] <- NA
  quL <- quantile(input, p, na.rm=TRUE)
  quH <- quantile(input, 1-p, na.rm=TRUE)

  result <- winsorize(input, p)
  middle <- !is.na(input) & input >= quL & input <= quH
  low <- !is.na(input) & input <= quL
  high <- !is.na(input) & input >= quH

  expect_identical(is.na(input), is.na(result))
  expect_identical(input[middle], result[middle])
  expect_true( all(result[low] == quL) )
  expect_true( all(result[high] == quH) )

})
\end{lstlisting}



\note{
  Here's a bigger, more interesting test.

  The code to test a function will generally be longer than the
  function itself.
}
\end{frame}


\begin{frame}{Workflow}

\bbi
\item Write tests as you're coding.
\item Run {\tt test()}
 \bi
 \item[] with {\tt devtools}, and working in your package directory
 \ei
\item Consider {\tt auto\_test("R", "tests")}
 \bi
 \item[] automatically runs tests when any file changes
 \ei
\item Periodically run {\tt R CMD check}
 \bi
 \item[] also {\tt R CMD check --as-cran}
 \ei
\ei

\note{
  Read Hadley's paper about {\tt testthat}. It's pretty easy to
  incorporate testing into your development workflow.

  It's really important to write the tests {\nhilit as you're
  coding}. You're will check to see if the code works; save the test
  code as a formal test.
}
\end{frame}


\begin{frame}{What to test?}

\bbi
\item You can't test {\hilit everything}.
\item Focus on the {\hilit boundaries}
   \bi
   \item (Depends on the nature of the problem)
   \item Vectors of length 0 or 1
   \item Things exactly matching
   \item Things with no matches
   \ei
\item Test handling of missing data.
   \bi
   \item[] {\tt NA}, {\tt Inf}, {\tt -Inf}
   \ei
\item Automate the construction of test cases
  \bi
  \item Create a table of inputs and expected outputs
  \item Run through the values in the table
  \ei
\ei

\note{
  You want your code to produce the correct output for {\nhilit any}
  input, but you can't test {\nhilit all} possible inputs, and you don't really
  need to.

  This is an experimental design question: what is the minimal set of
  inputs to verify that the code is correct, with little uncertainty?

  We generally focus on boundary cases, as those tend to be the
  places where things go wrong.

  The mishandling of different kinds of missing data is also a common
  source of problems and so deserving of special tests.
}
\end{frame}


\begin{frame}[c,fragile]{Another example}

\begin{lstlisting}
test_that("running mean with constant x or position", {

  n <- 100
  x <- rnorm(n)
  pos <- rep(0, n)

  expect_equal( runningmean(pos, x, window=1), rep(mean(x), n) )
  expect_equal( runningmean(pos, x, window=1, what="median"),
                rep(median(x), n) )
  expect_equal( runningmean(pos, x, window=1, what="sd"),
                rep(sd(x), n) )

  x <- rep(0, n)
  pos <- runif(n, 0, 5)

  expect_equal( runningmean(pos, x, window=1), x)
  expect_equal( runningmean(pos, x, window=1, what="median"), x)
  expect_equal( runningmean(pos, x, window=5, what="sd"),
                rep(0, n))
})
\end{lstlisting}

\note{
  Here's another example of unit tests, for a function calculating a
  running mean.

  Writing these tests revealed a bug in the code: with
  constant $x$'s, the code should give SD = 0, but it was giving {\tt
  NaN}'s due to round-off error that led to $\sqrt{\epsilon}$ for
  $\epsilon < 0$.

  This situation {\nhilit can} come up in practice, and this is
  exactly the sort of boundary case where problems tend to arise.
}
\end{frame}


\begin{frame}{Debugging tools}


\bbi
\item {\tt cat}, {\tt print}
\item {\tt traceback}, {\tt browser}, {\tt debug}
\item \href{http://www.rstudio.com/ide/docs/debugging/overview}{RStudio breakpoints}
\item \href{http://www.eclipse.org/eclipse}{Eclipse}/\href{http://www.walware.de/goto/statet}{StatET}
\item \href{http://www.sourceware.org/gdb/}{gdb}
\ei

\note{
  I'm going to say just a little bit about debugging.

  I still tend to just insert {\tt cat} or {\tt print} statements to
  isolate a problem.

  R does include a number of debugging tools, and RStudio has made
  these even easier to use.

  Eclipse/StatET is another development environment for R; it seems
  hard to set up.

  The GNU project debugger (gdb) is useful for compiled code.
}
\end{frame}


\begin{frame}[c]{Debugging}

\centerline{Step 1: Reproduce the problem}

\vspace{24pt}

\onslide<2->{\centerline{Step 2: Turn it into a test}}


\note{Try to create the minimal example the produces the problem. This
  helps both for refining your understanding of the problem and for
  speed in testing.

  Once you've created a minimal example that produces the problem,
  {\nhilit add that to your battery of automated tests!} The problem
  may suggest related tests to also add.
}
\end{frame}








\begin{frame}[c]{Debugging}


\centerline{Isolate the problem: where do things go bad?}

\note{
  The most common strategy I use in debugging involves a sort of
  divide-and-conquer: if R is crashing, figure out exactly where. If
  data are getting corrupted, step back to find out where it's gone
  from good to bad.
}
\end{frame}




\begin{frame}[c]{Debugging}


\centerline{Don't make the same mistake twice.}

\note{
  If you figure out some mistake you've made, search for all other
  possible instances of that mistake.

}
\end{frame}









\begin{frame}[c]{The most pernicious bugs}

\centerline{The code is right, but your thinking is wrong.}

\vspace{24pt}

\onslide<2->{\centerline{You were mistaken about what the code would do.}}

\vspace{24pt}

\onslide<3->{\centerline{\hilit $\rightarrow$ Write trivial programs to
    test your understanding.}}

\note{
  A number of times I've combed through code looking for the problem,
  but there really wasn't a problem in the code: the problem was just
  that my understanding of the algorithm was wrong.

  Or I was mistaken about some basic aspect of R.

  It can be useful to write small tests, to check your understanding.

  For an EM algorithm, always evaluate the likelihood function.
  It should be non-decreasing.
}
\end{frame}



\begin{frame}{Summary}

\bbi
\item If you don't test your code, how do you know it works?
\item If you test your code, save and automate those tests.
\item Check the input to each function.
\item Write unit tests for each function.
\item Write some larger regression tests.
\item Turn bugs into tests.
\ei

\note{
  Testing isn't fun. Like much of this course: somewhat painful
  initial investment, with great potential pay-off in the long run.
}
\end{frame}

\end{document}
