\documentclass[aspectratio=169,12pt,t]{beamer}
\usepackage{graphicx}
\setbeameroption{hide notes}
\setbeamertemplate{note page}[plain]
\usepackage{listings}

% header.tex: boring LaTeX/Beamer details + macros

% get rid of junk
\usetheme{default}
\beamertemplatenavigationsymbolsempty
\hypersetup{pdfpagemode=UseNone} % don't show bookmarks on initial view


% font
\usepackage{fontspec}
\setsansfont
  [ ExternalLocation = ../fonts/ ,
    UprightFont = *-regular ,
    BoldFont = *-bold ,
    ItalicFont = *-italic ,
    BoldItalicFont = *-bolditalic ]{texgyreheros}
\setbeamerfont{note page}{family*=pplx,size=\footnotesize} % Palatino for notes
% "TeX Gyre Heros can be used as a replacement for Helvetica"
% I've placed them in fonts/; alternatively you can install them
% permanently on your system as follows:
%     Download http://www.gust.org.pl/projects/e-foundry/tex-gyre/heros/qhv2.004otf.zip
%     In Unix, unzip it into ~/.fonts
%     In Mac, unzip it, double-click the .otf files, and install using "FontBook"

% named colors
\definecolor{offwhite}{RGB}{255,250,240}
\definecolor{gray}{RGB}{155,155,155}
\definecolor{purple}{RGB}{177,13,201}
\definecolor{green}{RGB}{46,204,64}

\definecolor{background}{RGB}{255,255,255}
\definecolor{foreground}{RGB}{24,24,24}
\definecolor{title}{RGB}{27,94,134}
\definecolor{subtitle}{RGB}{22,175,124}
\definecolor{hilit}{RGB}{122,0,128}
\definecolor{vhilit}{RGB}{255,0,128}
\definecolor{codehilit}{RGB}{255,0,128}
\definecolor{lolit}{RGB}{95,95,95}
\definecolor{myyellow}{rgb}{1,1,0.7}
\definecolor{nhilit}{RGB}{128,0,128}  % hilit color in notes
\definecolor{nvhilit}{RGB}{255,0,128} % vhilit for notes

\newcommand{\hilit}{\color{hilit}}
\newcommand{\vhilit}{\color{vhilit}}
\newcommand{\nhilit}{\color{nhilit}}
\newcommand{\nvhilit}{\color{nvhilit}}
\newcommand{\lolit}{\color{lolit}}

% use those colors
\setbeamercolor{titlelike}{fg=title}
\setbeamercolor{subtitle}{fg=subtitle}
\setbeamercolor{institute}{fg=lolit}
\setbeamercolor{normal text}{fg=foreground,bg=background}
\setbeamercolor{item}{fg=foreground} % color of bullets
\setbeamercolor{subitem}{fg=lolit}
\setbeamercolor{itemize/enumerate subbody}{fg=lolit}
\setbeamertemplate{itemize subitem}{{\textendash}}
\setbeamerfont{itemize/enumerate subbody}{size=\footnotesize}
\setbeamerfont{itemize/enumerate subitem}{size=\footnotesize}

% page number
\setbeamertemplate{footline}{%
    \raisebox{5pt}{\makebox[\paperwidth]{\hfill\makebox[20pt]{\lolit
          \scriptsize\insertframenumber}}}\hspace*{5pt}}

% add a bit of space at the top of the notes page
\addtobeamertemplate{note page}{\setlength{\parskip}{12pt}}

% default link color
\hypersetup{colorlinks, urlcolor={hilit}}

\lstset{language=bash,
        basicstyle=\ttfamily\scriptsize,
        frame=single,
        commentstyle=,
        backgroundcolor=\color{offwhite},
        showspaces=false,
        showstringspaces=false
        }


% a few macros
\newcommand{\bi}{\begin{itemize}}
\newcommand{\bbi}{\vspace{24pt} \begin{itemize} \itemsep8pt}
\newcommand{\ei}{\end{itemize}}
\newcommand{\be}{\begin{enumerate}}
\newcommand{\bbe}{\vspace{24pt} \begin{enumerate} \itemsep8pt}
\newcommand{\ee}{\end{enumerate}}
\newcommand{\ig}{\includegraphics}
\newcommand{\subt}[1]{{\footnotesize \color{subtitle} {#1}}}
\newcommand{\ttsm}{\tt \small}
\newcommand{\ttfn}{\tt \footnotesize}
\newcommand{\figh}[2]{\centerline{\includegraphics[height=#2\textheight]{#1}}}
\newcommand{\figw}[2]{\centerline{\includegraphics[width=#2\textwidth]{#1}}}


%%%%%%%%%%%%%%%%%%%%%%%%%%%%%%%%%%%%%%%%%%%%%%%%%%%%%%%%%%%%%%%%%%%%%%
% end of header
%%%%%%%%%%%%%%%%%%%%%%%%%%%%%%%%%%%%%%%%%%%%%%%%%%%%%%%%%%%%%%%%%%%%%%

% title info
\title{The EM algorithm}
\subtitle{Analysis of a T cell frequency assay}
\author{\href{https://kbroman.org}{Karl Broman}}
\institute{Biostatistics \& Medical Informatics, UW{\textendash}Madison}
\date{\href{https://kbroman.org}{\tt \scriptsize \color{foreground} kbroman.org}
\\[-4pt]
\href{https://github.com/kbroman}{\tt \scriptsize \color{foreground} github.com/kbroman}
\\[-4pt]
\href{https://twitter.com/kwbroman}{\tt \scriptsize \color{foreground} @kwbroman}
\\[-4pt]
{\scriptsize Course web: \href{https://kbroman.org/AdvData}{\tt kbroman.org/AdvData}}
}


\begin{document}

% title slide
{
\setbeamertemplate{footline}{} % no page number here
\frame{
  \titlepage

\note{
   In this lecture, we'll look at the EM algorithm, which is a broadly
   useful algorithm for getting maximum likelihood estimates. We'll
   learn about it through a case study on a T cell frequency assay,
   for measuring the effectiveness of a vaccine.
}

} }


\begin{frame}[c]{}

  \begin{columns}[c]
    \column{0.3\textwidth}
    \fontsize{10pt}{11}\selectfont
      \bbi
      \item[\hilit Goal:] Estimate the frequency of T-cells in a blood
        sample that respond to two test antigens.

      \item[\hilit Real goal:] Determine whether a vaccine causes an
        increase in the frequency of responding T-cells.
      \ei

      \vspace{10mm}

      \hspace*{-0.2\textwidth} {\fontsize{6pt}{6}\selectfont Broman K, Speed T, Tigges M (1996)
        J Immunol Meth 198:119-132
        \href{https://doi.org/b54v33}{\tt doi.org/b54v33}}


    \column{0.5\textwidth}
        \figw{Figs/immunology.png}{1.0}

  \end{columns}

\note{
    We're considering an assay that seeks to estimate the frequency of
    T cells in a blood sample that respond to a particular antigen.
    Antigens are bits of foreign protein, such as chewed-up virus.
    Antigen-presenting cells take up these proteins and present them
    on their surface. This ultimately leads to the proliferation of
    specific T cells with receptors complementary to the antigen,
    which differentiate to create memory cells (which lead to a faster
    response later) or effector cells (which go around destroying
    antigen).

    The real goal was to determine whether a particular vaccine was
    looking to be effective by leading to an increase in the frequency
    of responding T-cells.
}

\end{frame}



\begin{frame}[c]{The assay}

  \begin{columns}[c]
    \column{0.5\textwidth}
    \fontsize{10pt}{11}\selectfont

    \bbi
  \item Combine:
    \bi
  \item diluted blood cells + growth medium
  \item antigen
  \item $^{\text{3}}$H-thymidine
    \ei

    \item Replicating cells take up $^{\text{3}}$H-thymidine.

    \item Extract the DNA and measure its radioactivity
      \ei

    \column{0.5\textwidth}
        \figw{Figs/assay.png}{1.0}

  \end{columns}

\note{
    In the assay we're considering, you combine diluted blood cells
    and growth medium with antigen and $^3$H-thymidine (the T in DNA,
    but radioactive). Replicating cells take up $^3$H-thymidine, and
    you then extract the DNA and measure the radioactivity.
}

\end{frame}


\begin{frame}{Usual approaches}

  \bbi

\item Use 3 wells with antigen and 3 wells without antigen,\\
  and take the ratio of the averages

  \item Limiting dilution assay
    \bi
  \item Several dilutions of cells
  \item Many wells at each dilution
    \ei
    \ei

\note{
  The usual approaches use three wells with antigen and three cells
  without, and just look at like the ratio of the average response.
  This was viewed as too crude for the current context.

  Alternatively, you could do a limiting dilution assay: use several
  dilutions of cells, with many wells at each dilution. But this was
  viewed as requiring much more blood than the subjects would
  generally be willing to offer.
}

\end{frame}


\begin{frame}[c]{Our assay}

  \begin{columns}[c]
    \column{0.5\textwidth}
    Study a single plate or pair of plates at a single dilution.

    \column{0.5\textwidth}
    \figw{Figs/microtiter_plate.png}{1.0}


  \end{columns}


\note{
    And so our collaborator was interested in trying to get by with a
    single 8$\times$12 plate of wells, or maybe a pair of plates, at a
    single dilution. There would be some control wells (with just
    cells), and then a couple of sections of wells with one of the two
    test antigens, and then some wells with Tetanus toxin (as a
    positive control; all subjects will have been exposed to Tetanus
    toxin, via a tetanus vaccine). The PHA wells are another positive
    control.
}

\end{frame}

\begin{frame}[c]{Data}
  \figh{Figs/lda713_data.png}{1.0}

\note{
    Here's an example data set. Higher numbers mean more
    radioactivity, indicating cells have taken up $^3$H-thymidine into
    their DNA.
}

\end{frame}


\begin{frame}[c]{Traditional analysis}
  \bbi
  \item Split wells into +/-- using a cutoff (e.g., mean + 3 SD of
    ``cells alone'' wells)
    \bi
  \item[] positive = one or more responding cells
  \item[] negative = no responding cells
    \ei

  \item Imagine that the number of responding cells in a well is
    Poisson($\lambda_i$) for group $i$

    \vspace{4mm}

  \qquad Pr(no responding cells) = $e^{-\lambda_i}$

    \vspace{4mm}

  \qquad $\hat{\lambda}_i = -\log\left(\frac{\text{\# negative wells}}{\text{\# wells}}\right)$

    \ei


\note{
    The tranditional analysis of such data was to split wells into
    positive and negative wells according to whether they seem to have
    one or more responding cells, or to have no responding cells.
    Assuming that the number of responding cells in a well follows a
    Poisson distribution, you can use the proportion of negative cells
    to get an estimate of the average number of responding cells per
    well.

    The Poisson distribution is like a binomial(n,p) distribution
    where n is really large and p is really small, with the mean of
    the Poisson distribution $\lambda = n p$.
}

\end{frame}

\begin{frame}[c]{Analysis}
  \figh{Figs/simple_method.png}{0.9}

\note{
   Applying the simple method to the example data, we might derive a
   cutoff between negative and positive wells as mean + 3 SD of the
   cells-alone wells. Then count the negative wells and convert to get
   estimates of the number of responding cells in a well.

   Ultimately, we subtract off the baseline estimate for the
   cells-alone wells and re-scale to number of responders per million
   cells.
}

\end{frame}



\begin{frame}{Problems}
  \bbi
\item Hard to choose cutoff
\item Potential loss of information
  \ei

\note{
   But it can be hard to choose a cutoff, and there's a potential loss
   of information by converting the quantitative values into binary
   data.

   And since we're trying to get by with just one or two plates of
   data at a single dilution, we really want to try to extract as much
   information as possible from the observed data.
}

\end{frame}


\begin{frame}{Response vs no. cells}
\figw{Figs/mean_response.pdf}{1.0}

\note{
    For a couple of subjects we had a dilution series of data. If you
    look at the average response in each section of cells by cell
    dilution in a subject, there is a clear dose-response relationship
    which suggests that there's more information about the cellular
    response than just positive/negative status of the wells.
}

\end{frame}


\begin{frame}[c]{Model}

  \vspace{3mm}

  $k_{ij}$ = Number of responding cells (unobserved)

  $y_{ij}$ = square-root of response

  \vspace{10mm}

  Assume {\hilit $k_{ij}$ $\sim \text{Poisson}(\lambda_i)$}

  \vspace{2mm}

  \hspace*{15mm} {\hilit $y_{ij}$ $|$ $k_{ij}$ $\sim \text{Normal}(a +
    b k_{ij}, \sigma)$}

  \vspace{2mm}

  \hspace*{15mm} {\hilit ($k_{ij}$, $y_{ij}$) mutually independent}

  \vspace{5mm}

  \figw{Figs/model_distribution.pdf}{1.0}


\note{
    To extract more information from the quantitative response, let's
    create a model for these observed data. It's natural to assume
    that the number of responding cells in a well follows a Poisson
    distribution. Let's further assume that some transformation of the
    quantitative response is linear in the number of responding cells,
    with normally distributed residual variation. And partly due to
    the shape of the curves on the previous slide, we went with
    square-root of the response.

    So we have normally distributed response for wells with 0
    responders, a shift upwards for wells with 1 responder, etc.  The
    overall response distribution is a ``mixture'' of these normally
    distributed components, with the relative proportions of the
    components being according to the Poisson probabilities.
}

\end{frame}




\begin{frame}[c]{log Likelihood}


  \begin{eqnarray*}
    l(\boldsymbol{\lambda}, a, b, \sigma)
       & = & \sum_{i,j} \log \text{Pr}(y_{ij} | \lambda_i, a, b, \sigma) \\
    & = & \sum_{i,j} \log \left[ \sum_k
      \text{Pr}(k | \lambda_i)
      \text{Pr}(y_{ij} | k, a, b, \sigma)
      \right] \\
    & = & \sum_{i,j} \log \left[ \sum_k
      \left(\frac{e^{-\lambda_i} \lambda_i^k}{ k! }\right) \, \phi\left( \frac{y_{ij} - a -
        bk}{\sigma} \right)
      \right]
  \end{eqnarray*}


\note{
    Our goal is to get estimates of the mean numbers of responding
    cells in each group of wells, as well as the plate-specific
    parameters $a$, $b$, and $\sigma$.

    We can write down the likelihood for the data; optimizing this
    likelihood would give us the MLEs for the parameters.
}

\end{frame}



\begin{frame}{EM algorithm}

\vspace{-3mm}

  \bbi
  \item Iterative algorithm useful when there is missing data that if
    observed would make things easy
  \item Dempster et al. (1977) JRSS-B 39:1-22
    \href{https://doi.org/gfxzrv}{\tt doi.org/gfxzrv}
  \item Start with some initial estimates
  \item {\hilit E-step}: expected value of missing data given current estimates
  \item {\hilit M-step}: MLEs replacing missing data with their expected values
  \item {\hilit Advantages}
    \bi
  \item often easy to code
  \item super stable
  \item log likelihood is non-decreasing
    \ei
    \ei


\note{
  The EM algorithm is a particular optimization method for getting the
  MLEs that is useful in this sort of situation where we have missing
  data (the $k$'s). If we {\hilit knew} the number of responding cells
  in each well, we could estimate the averages by just taking the
  averages of the $k$'s in each group, and we could estimate $a$, $b$,
  and $\sigma$ by linear regression of the responses on the $k$'s.

  Not knowing the $k$'s, the EM algorithm works by iterating between
  an E step (where you get expected values for the $k$'s, given the
  observed data and given current estimates of the parameters) and an
  M step (where you derive new and improved estimates, using the
  expected values of the $k$'s in place of their true (but unknown)
  values.

  The EM algorithm has a number of advantages: it is often easy to
  implement in software, it can be super stable (meaning no matter
  what starting values you use, the algorithm will converge to some
  finite values that are at least reasonable), and it can be proven
  that across iterations, the log likelihood is non-decreasing.
}

\end{frame}



\begin{frame}{Normal/Poisson model}

  {\hilit E-step}:

\vspace{-3mm}

  \begin{eqnarray*}
      \text{Pr}(k = s | y, \lambda, a, b, \sigma) & = & \frac{\text{Pr}(k = s | \lambda)
      \text{Pr}(y | k = s, a, b, \sigma)}{\sum_s \text{Pr}(k = s | \lambda)
        \text{Pr}(y | k = s, a, b, \sigma)} \\
      & = & \frac{\left(\frac{e^{-\lambda} \lambda^s}{s!}\right) \phi\left(\frac{y-a-bs}{\sigma}\right)}{
\sum_s \left(\frac{e^{-\lambda} \lambda^s}{s!}\right)
\phi\left(\frac{y-a-bs}{\sigma}\right)} \\[18pt]
      \text{E}(k | y, \lambda, a, b, \sigma)
      & = & \frac{\sum_s s \left(\frac{e^{-\lambda}
          \lambda^s}{s!}\right)
        \phi\left(\frac{y-a-bs}{\sigma}\right)}{\sum_s
        \left(\frac{e^{-\lambda} \lambda^s}{s!}\right)
        \phi\left(\frac{y-a-bs}{\sigma}\right)}
  \end{eqnarray*}


  \vspace{5mm}

  {\hilit M-step}: Regress y on $\text{E}(k|y)$



\note{
  For this particular model, the idea is to calculate expected values
  for the $k$'s given the observed data and given current values for
  the parameters, using Bayes's rule. (In practice, the sums go up to
  some maximum $k$, like 20, where the posterior probability for $k$
  has gotten really small.)

  At the M step, you then take averages of these values, and
  regression y on these values.

  Go back and forth between the two steps until the estimates converge.
}

\end{frame}




\begin{frame}{Oops, that didn't work}
\figw{Figs/em_loglik_wrong.pdf}{1.0}

\note{
  Here's the log likelihood plotted against iteration. As you can see,
  the thing didn't really work as it should have. The log likelihood
  is going up and down rather than being non-decreasing, as I said it
  should be.
}

\end{frame}



\begin{frame}{EM algorithm, more formally}

  \bbi

  \item Calculate expected complete-data log likelihood, given
    observed data and observed parameters, and then maximize that.

    $$ l^{(s)}(\theta) = \text{E}\{\log f(y, k | \theta) | y,
    \hat{\theta}^{(s)}\} $$

  \item In practice, it's usually a linear combination of the
    sufficient statistics, so you focus on those.

  \item Here, we need not just $\sum k$ and $\sum k y$, but also $\sum k^2$.

    \ei


\note{
   It turns out that, formally, you need to calculate the expected
   value of the complete-data log likelihood function. In practice,
   this is a linear combination of the sufficient statistics.

   And in this situation the sufficient statistics include not just
   $\sum k$ and $\sum ky$, but also $\sum k^2$.

   Note taking account of $\sum k^2$ was our problem, as E($k^2$)
   $\ne$ [E($k$)]$^2$.

   This took a while for me to figure out, as I initially thought
   there was a bug in my code, but really there was a bug in my
   {\hilit understanding} of the EM algorithm. Calculating the log
   likelihood and following it by iteration was a super-useful
   diagnostic.
}

\end{frame}

\begin{frame}{EM algorithm, again}

{\hilit E step}: we also need

\vspace{-4mm}

\begin{eqnarray*}
      \text{E}(k^2 | y, \lambda, a, b, \sigma)
      & = & \frac{\sum_s s^2 \left(\frac{e^{-\lambda}
          \lambda^s}{s!}\right)
        \phi\left(\frac{y-a-bs}{\sigma}\right)}{\sum_s
        \left(\frac{e^{-\lambda} \lambda^s}{s!}\right)
        \phi\left(\frac{y-a-bs}{\sigma}\right)}
\end{eqnarray*}

\vspace{5mm}

{\hilit M step}: we want $\hat{\beta} = (X'X)^{-1}(X'y)$

\begin{eqnarray*}
\text{where }  (X'X) & \text{ is like } &
  \begin{pmatrix} n & \sum k \\ \sum k & \sum k^2 \end{pmatrix}
  \\[18pt]
\text{ and }  (X'y) & \text{ is like } &
  \begin{pmatrix}\sum y \\ \sum k y \end{pmatrix}
\end{eqnarray*}




\note{
   Going back to the EM algorithm, we need to include the calculation of
   the expected value of $k^2$ given the observed data and given the
   parameter estimates. This is easy enough, because it's just like
   calculation the expected value of $k$.

   More difficult is that the linear regression part of the M step
   involves sort of going back-to-basics. You need to figure out where
   the $k^2$ values enter into things and plug in E($k$) for $k$ and
   E($k^2$) for $k^2$.

   I think the easiest way to do this is to look at the $X'X$ matrix,
   which will be 2$\times$2, and stick the E($k^2$) values in there.
}

\end{frame}



\begin{frame}{Ah, that's better}
\figw{Figs/em_loglik_right.pdf}{1.0}

\note{
  Having corrected our algorithm, here is the trace of the log
  likelihood by iteration. Non-decreasing, as it should be.
}

\end{frame}


\begin{frame}{Difficulties}

\bbi
\item Starting values
\item Multiple modes
\ei


\note{
  The EM algorithm may sound great, but I've swept some important
  difficulties under the rug.

  First, we need starting values. How do we get those? The easiest way
  is to use our crude method of splitting wells into
  positive/negative; that will give us some estimates of the $\lambda$
  values. We can look at the relationship between the average
  responses and those $\lambda$'s to get estimates of $a$, $b$, and
  $\sigma$.

  More difficult is that while the EM algorithm will converge to
  something, it won't necessarily converge to the global MLEs. There
  can be multiple modes in the likelihood surface. A solution to this
  is to use lots of random starting points and pick the best of the
  converged estimates.
}

\end{frame}




\begin{frame}[c]{Multiple modes}
\figw{Figs/multiple_modes.pdf}{1.0}

\note{
   Here's a particular example. I initiated the EM algorithm at 1000
   different random starting points, and it converged to 8 different
   sets of estimates.
}

\end{frame}




\begin{frame}[c]{Multiple modes}

\centering
\renewcommand{\arraystretch}{1.3}
\begin{tabular}{rrrrrrrrrr}
  \hline
 & $\lambda_0$ & $\lambda_D$ & $\lambda_B$ & $\lambda_T$ & $a$ &
  $b$ & $\sigma$ & log lik & no. hits \\
  \hline
1 & 0.32 & 3.03 & 2.82 & 4.37 & 16.73 & 10.34 & 3.52 & --289.73 & 331 \\
  2 & 1.18 & 5.40 & 4.95 & 7.49 & 12.16 & 6.69 & 2.15 & --289.80 & 26 \\
  3 & 0.17 & 2.10 & 1.95 & 3.07 & 17.44 & 14.56 & 4.18 & --290.50 & 415 \\
  4 & 0.51 & 3.89 & 3.56 & 5.58 & 15.72 & 8.35 & 3.58 & --290.70 & 180 \\
  5 & 0.73 & 4.62 & 4.25 & 6.58 & 14.58 & 7.27 & 3.43 & --291.08 & 30 \\
  6 & 1.64 & 6.79 & 6.29 & 9.35 & 10.81 & 5.51 & 1.89 & --291.40 & 7 \\
  7 & 1.57 & 6.22 & 5.80 & 8.61 & 10.60 & 6.02 & 2.13 & --291.59 & 10 \\
  8 & 2.59 & 7.76 & 7.25 & 10.34 & 5.75 & 5.47 & 1.88 & --292.27 & 1 \\
   \hline
\end{tabular}


\note{
   Here are the 8 modes we found, along with their log likelihood and
   the numbers of times they were hit, out of 1000.

   This is maybe a bit concerning. Actually, it is quite concerning.
   Our estimated mode has estimated $\lambda$'s around 3, but there's
   another mode just slightly lower in likelihood that has the
   estimates around 5, and another mode just below that with estimates
   at 2.

   The model is great, but the data aren't quite sufficient to fit it,
   is what you might conclude. The different modes here have a lot in
   common with that choice of cutoff in the traditional approach to
   analyzing these sort of data.
}

\end{frame}


\begin{frame}[c]{Estimate vs. starting point}
\figw{Figs/starting_pts.pdf}{1.0}

\note{
  Here's a plot of the estimates vs the starting value that I had used
  for that parameter. Each panel is one parameter and the 1000 points
  are the 1000 starting points. As you can see, for most of the
  parameters there seems no real relationship between where you start
  and where you end up.

  But the $b$ parameter (slope in response vs responding cells) shows
  a very strong relationship: the initial value for $b$ has a big
  effect on where you end up.
}

\end{frame}




\begin{frame}{Principles}

  \bbi
\item Start with an understanding of the problem and data
\item Think about a model for the data-generating process
  \ei

\note{
    Some important principles that were guiding this work: start with
    an understanding of the problem and the data, and think about a
    model for the data-generating process. That led us to this
    normal/Poisson mixture model.
}

\end{frame}



\begin{frame}{Lessons}

  \bbi
\item The EM algorithm is really useful
\item Use the log likelihood as a diagnostic when implementing an EM
  algorithm
  \ei

\note{
    And what have we learned? The EM algorithm is really useful, and
    you should use the log likelihood as a diagnostic when
    implementing it.
}

\end{frame}


\begin{frame}{Software development time}

  \bbi
\item Formulating the problem
\item Writing the code
\item Debugging the code
\item Executing the code
  \ei


\note{
    Real selling points for the EM algorithm come up when you look at
    the time involved in writing a program to give parameter
    estimates. Programming time includes the time to formulate the
    problem, the time to write the code, debug the code, and execute
    the code.

    Where the EM algorithm really shines is on the ``writing the
    code'' and ``debugging the code'' sections. The EM algorithm is
    often rather simple to code, and it comes with a built-in
    diagnostic.
}

\end{frame}


\begin{frame}{Impact}

  \bbi
\item I'm pretty sure that the vaccine they were working on didn't
  work well.
\item R package \href{https://github.com/kbroman/npem}{\tt npem}, but
  I never put it on \href{https://cran.r-project.org}{CRAN}, and no
  one has ever asked me about it.
\item Our paper has like 9 citations: no one has ever really used the
  method.
  \ei


\note{
  What can we say about the impact of this work?

  It didn't seem to have much impact, I think. The vaccine we were
  studying didn't seem to be very effective.

  The R package I wrote, implementing the method, is on GitHub, but I
  never did put it on CRAN. I distributed it just through my personal
  web site. No one has ever asked me about it, so probably it has
  never been used.

  The paper we wrote about this work has just like 9 citations. No one
  has every really used the method. Bummer.
}

\end{frame}


\begin{frame}{Further things}

  \bbi

\item Standard errors should always be required.
  \bi
\item But usually painful to obtain
\item We used the SEM algorithm of Meng and Rubin (1991)
  \href{https://doi.org/dk27}{\tt doi.org/dk27}
  \ei

\item Could more formally investigate the appropriate transformation
  \bi
\item See Box and Cox (1964) \href{https://doi.org/gfrhvs}{\tt doi.org/gfrhvs}
\item Box-Cox transformation is $g(y) = (y^c-1)/c$ for $c \ne 0$ and
  $= \log y$ for $c=0$
\item Key issue is change-of-variables in the density; as a result you
  add $\sum_{ij} (c-1) \log y_{ij}$ to the log likelihood
    \ei
    \ei

\note{
    A couple of further points. Standard errors should always be
    required. If you give an estimate, you need to give a standard
    error.  Getting standard errors here is a bit painful. We used the
    SEM algorithm, which uses the rate of convergence of EM. It's a
    bit clunky, but it doesn't require writing a whole lot of new code,
    and it seems to give reasonable results.

    Further, we could more formally investigate the appropriate
    transformation. See the Box and Cox (1964) paper, The key issue is
    the change-of-variables in the density, so that you have to add a
    quantity to the log likelihood. It's slightly tricky to get right,
    but it is a nice way to use the data to determine the appropriate
    transformation.
}

\end{frame}


\end{document}
