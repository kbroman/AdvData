\documentclass[aspectratio=169,12pt,t]{beamer}
\usepackage{graphicx}
\setbeameroption{hide notes}
\setbeamertemplate{note page}[plain]
\usepackage{listings}

% header.tex: boring LaTeX/Beamer details + macros

% get rid of junk
\usetheme{default}
\beamertemplatenavigationsymbolsempty
\hypersetup{pdfpagemode=UseNone} % don't show bookmarks on initial view


% font
\usepackage{fontspec}
\setsansfont
  [ ExternalLocation = ../fonts/ ,
    UprightFont = *-regular ,
    BoldFont = *-bold ,
    ItalicFont = *-italic ,
    BoldItalicFont = *-bolditalic ]{texgyreheros}
\setbeamerfont{note page}{family*=pplx,size=\footnotesize} % Palatino for notes
% "TeX Gyre Heros can be used as a replacement for Helvetica"
% I've placed them in fonts/; alternatively you can install them
% permanently on your system as follows:
%     Download http://www.gust.org.pl/projects/e-foundry/tex-gyre/heros/qhv2.004otf.zip
%     In Unix, unzip it into ~/.fonts
%     In Mac, unzip it, double-click the .otf files, and install using "FontBook"

% named colors
\definecolor{offwhite}{RGB}{255,250,240}
\definecolor{gray}{RGB}{155,155,155}
\definecolor{purple}{RGB}{177,13,201}
\definecolor{green}{RGB}{46,204,64}

\definecolor{background}{RGB}{255,255,255}
\definecolor{foreground}{RGB}{24,24,24}
\definecolor{title}{RGB}{27,94,134}
\definecolor{subtitle}{RGB}{22,175,124}
\definecolor{hilit}{RGB}{122,0,128}
\definecolor{vhilit}{RGB}{255,0,128}
\definecolor{codehilit}{RGB}{255,0,128}
\definecolor{lolit}{RGB}{95,95,95}
\definecolor{myyellow}{rgb}{1,1,0.7}
\definecolor{nhilit}{RGB}{128,0,128}  % hilit color in notes
\definecolor{nvhilit}{RGB}{255,0,128} % vhilit for notes

\newcommand{\hilit}{\color{hilit}}
\newcommand{\vhilit}{\color{vhilit}}
\newcommand{\nhilit}{\color{nhilit}}
\newcommand{\nvhilit}{\color{nvhilit}}
\newcommand{\lolit}{\color{lolit}}

% use those colors
\setbeamercolor{titlelike}{fg=title}
\setbeamercolor{subtitle}{fg=subtitle}
\setbeamercolor{institute}{fg=lolit}
\setbeamercolor{normal text}{fg=foreground,bg=background}
\setbeamercolor{item}{fg=foreground} % color of bullets
\setbeamercolor{subitem}{fg=lolit}
\setbeamercolor{itemize/enumerate subbody}{fg=lolit}
\setbeamertemplate{itemize subitem}{{\textendash}}
\setbeamerfont{itemize/enumerate subbody}{size=\footnotesize}
\setbeamerfont{itemize/enumerate subitem}{size=\footnotesize}

% page number
\setbeamertemplate{footline}{%
    \raisebox{5pt}{\makebox[\paperwidth]{\hfill\makebox[20pt]{\lolit
          \scriptsize\insertframenumber}}}\hspace*{5pt}}

% add a bit of space at the top of the notes page
\addtobeamertemplate{note page}{\setlength{\parskip}{12pt}}

% default link color
\hypersetup{colorlinks, urlcolor={hilit}}

\lstset{language=bash,
        basicstyle=\ttfamily\scriptsize,
        frame=single,
        commentstyle=,
        backgroundcolor=\color{offwhite},
        showspaces=false,
        showstringspaces=false
        }


% a few macros
\newcommand{\bi}{\begin{itemize}}
\newcommand{\bbi}{\vspace{24pt} \begin{itemize} \itemsep8pt}
\newcommand{\ei}{\end{itemize}}
\newcommand{\be}{\begin{enumerate}}
\newcommand{\bbe}{\vspace{24pt} \begin{enumerate} \itemsep8pt}
\newcommand{\ee}{\end{enumerate}}
\newcommand{\ig}{\includegraphics}
\newcommand{\subt}[1]{{\footnotesize \color{subtitle} {#1}}}
\newcommand{\ttsm}{\tt \small}
\newcommand{\ttfn}{\tt \footnotesize}
\newcommand{\figh}[2]{\centerline{\includegraphics[height=#2\textheight]{#1}}}
\newcommand{\figw}[2]{\centerline{\includegraphics[width=#2\textwidth]{#1}}}


%%%%%%%%%%%%%%%%%%%%%%%%%%%%%%%%%%%%%%%%%%%%%%%%%%%%%%%%%%%%%%%%%%%%%%
% end of header
%%%%%%%%%%%%%%%%%%%%%%%%%%%%%%%%%%%%%%%%%%%%%%%%%%%%%%%%%%%%%%%%%%%%%%

% title info
\title{The EM algorithm}
\subtitle{Analysis of a T cell frequency assay}
\author{\href{https://kbroman.org}{Karl Broman}}
\institute{Biostatistics \& Medical Informatics \\ UW{\textendash}Madison}
\date{\href{https://kbroman.org}{\tt \scriptsize \color{foreground} kbroman.org}
\\[-4pt]
\href{https://github.com/kbroman}{\tt \scriptsize \color{foreground} github.com/kbroman}
\\[-4pt]
\href{https://twitter.com/kwbroman}{\tt \scriptsize \color{foreground} @kwbroman}
\\[-4pt]
{\scriptsize Course web: \href{https://kbroman.org/AdvData}{\tt kbroman.org/AdvData}}
}


\begin{document}

% title slide
{
\setbeamertemplate{footline}{} % no page number here
\frame{
  \titlepage

  \note{}

} }


\begin{frame}[c]{}

  \begin{columns}[c]
    \column{0.3\textwidth}
    \fontsize{10pt}{11}\selectfont
      \bbi
      \item[\hilit Goal:] Estimate the frequency of T-cells in a blood
        sample that respond to two test antigens.

      \item[\hilit Real goal:] Determine whether a vaccine causes an
        increase in the frequency of responding T-cells.
      \ei

      \vspace{10mm}

      \hspace*{-0.2\textwidth} {\fontsize{6pt}{6}\selectfont Broman K, Speed T, Tigges M (1996)
        J Immunol Meth 198:119-132
        \href{https://doi.org/b54v33}{\tt doi.org/b54v33}}


    \column{0.5\textwidth}
        \figw{Figs/immunology.png}{1.0}

  \end{columns}


\end{frame}



\begin{frame}[c]{The assay}

  \begin{columns}[c]
    \column{0.5\textwidth}
    \fontsize{10pt}{11}\selectfont

    \bbi
  \item Combine:
    \bi
  \item diluted blood cells + growth medium
  \item antigen
  \item $^{\text{3}}$H-thymidine
    \ei

    \item Replicating cells take up $^{\text{3}}$H-thymidine.

    \item Extract the DNA and measure its radioactivity
      \ei

    \column{0.5\textwidth}
        \figw{Figs/assay.png}{1.0}

  \end{columns}

\end{frame}


\begin{frame}{Usual approaches}

  \bbi

\item Use 3 wells with antigen and 3 wells without antigen,\\
  and take the ratio of the averages

  \item Limiting dilution assay
    \bi
  \item Several dilutions of cells
  \item Many wells at each dilution
    \ei
    \ei

\end{frame}


\begin{frame}[c]{Our assay}

  \begin{columns}[c]
    \column{0.5\textwidth}
    Study a single plate or pair of plates at a single dilution.

    \column{0.5\textwidth}
    \figw{Figs/microtiter_plate.png}{1.0}


  \end{columns}


\end{frame}

\begin{frame}[c]{Data}
  \figh{Figs/lda713_data.png}{1.0}
\end{frame}


\begin{frame}[c]{Traditional analysis}
  \bbi
  \item Split wells into +/-- using a cutoff (e.g., mean + 3 SD of
    ``cells alone'' wells)
    \bi
  \item[positive =] one or more responding cells
  \item[negative =] no responding cells
    \ei

  \item Imagine that the number of responding cells in a well is
    Poisson($\lambda_i$) for group $i$

    \vspace{4mm}

  \qquad Pr(no responding cells) = $e^{-\lambda_i}$

    \vspace{4mm}

  \qquad $\hat{\lambda}_i = -\log\left(\frac{\text{\# negative wells}}{\text{\# wells}}\right)$

    \ei

\end{frame}

\begin{frame}[c]{Analysis}
  \figh{Figs/simple_method.png}{0.9}
\end{frame}



\begin{frame}{Problems}
  \bbi
\item Hard to choose cutoff
\item Potential loss of information
  \ei
\end{frame}


\begin{frame}{Response vs no. cells}
\figw{Figs/mean_response.pdf}{1.0}
\end{frame}


\begin{frame}[c]{Model}

  k$_{ij}$ = Number of responding cells (unobserved)

  y$_{ij}$ = square-root of response

  \vspace{10mm}

  Assume {\hilit k$_{ij}$ $\sim \text{Poisson}(\lambda_i)$}

  \vspace{2mm}

  \hspace*{15mm} {\hilit y$_{ij}$ $|$ k$_{ij}$ $\sim \text{Normal}(a +
    b k_{ij}, \sigma)$}

  \vspace{2mm}

  \hspace*{15mm} {\hilit (k$_{ij}$, y$_{ij}$) mutually independent}

  \vspace{5mm}

  \figw{Figs/model_distribution.pdf}{1.0}

\end{frame}

\end{document}
