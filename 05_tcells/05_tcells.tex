\documentclass[aspectratio=169,12pt,t]{beamer}
\usepackage{graphicx}
\setbeameroption{hide notes}
\setbeamertemplate{note page}[plain]
\usepackage{listings}

\input{../LaTeX/header.tex}

%%%%%%%%%%%%%%%%%%%%%%%%%%%%%%%%%%%%%%%%%%%%%%%%%%%%%%%%%%%%%%%%%%%%%%
% end of header
%%%%%%%%%%%%%%%%%%%%%%%%%%%%%%%%%%%%%%%%%%%%%%%%%%%%%%%%%%%%%%%%%%%%%%

% title info
\title{The EM algorithm}
\subtitle{Analysis of a T cell frequency assay}
\author{\href{https://kbroman.org}{Karl Broman}}
\institute{Biostatistics \& Medical Informatics \\ UW{\textendash}Madison}
\date{\href{https://kbroman.org}{\tt \scriptsize \color{foreground} kbroman.org}
\\[-4pt]
\href{https://github.com/kbroman}{\tt \scriptsize \color{foreground} github.com/kbroman}
\\[-4pt]
\href{https://twitter.com/kwbroman}{\tt \scriptsize \color{foreground} @kwbroman}
\\[-4pt]
{\scriptsize Course web: \href{https://kbroman.org/AdvData}{\tt kbroman.org/AdvData}}
}


\begin{document}

% title slide
{
\setbeamertemplate{footline}{} % no page number here
\frame{
  \titlepage

  \note{}

} }


\begin{frame}[c]{}

  \begin{columns}[c]
    \column{0.3\textwidth}
    \fontsize{10pt}{11}\selectfont
      \bbi
      \item[\hilit Goal:] Estimate the frequency of T-cells in a blood
        sample that respond to two test antigens.

      \item[\hilit Real goal:] Determine whether a vaccine causes an
        increase in the frequency of responding T-cells.
      \ei

      \vspace{10mm}

      \hspace*{-0.2\textwidth} {\fontsize{6pt}{6}\selectfont Broman K, Speed T, Tigges M (1996)
        J Immunol Meth 198:119-132
        \href{https://doi.org/b54v33}{\tt doi.org/b54v33}}


    \column{0.5\textwidth}
        \figw{Figs/immunology.png}{1.0}

  \end{columns}


\end{frame}



\begin{frame}[c]{The assay}

  \begin{columns}[c]
    \column{0.5\textwidth}
    \fontsize{10pt}{11}\selectfont

    \bbi
  \item Combine:
    \bi
  \item diluted blood cells + growth medium
  \item antigen
  \item $^{\text{3}}$H-thymidine
    \ei

    \item Replicating cells take up $^{\text{3}}$H-thymidine.

    \item Extract the DNA and measure its radioactivity
      \ei

    \column{0.5\textwidth}
        \figw{Figs/assay.png}{1.0}

  \end{columns}

\end{frame}


\begin{frame}{Usual approaches}

  \bbi

\item Use 3 wells with antigen and 3 wells without antigen,\\
  and take the ratio of the averages

  \item Limiting dilution assay
    \bi
  \item Several dilutions of cells
  \item Many wells at each dilution
    \ei
    \ei

\end{frame}


\begin{frame}[c]{Our assay}

  \begin{columns}[c]
    \column{0.5\textwidth}
    Study a single plate or pair of plates at a single dilution.

    \column{0.5\textwidth}
    \figw{Figs/microtiter_plate.png}{1.0}


  \end{columns}


\end{frame}

\begin{frame}[c]{Data}
  \figh{Figs/lda713_data.png}{1.0}
\end{frame}


\begin{frame}[c]{Traditional analysis}
  \bbi
  \item Split wells into +/-- using a cutoff (e.g., mean + 3 SD of
    ``cells alone'' wells)
    \bi
  \item[positive =] one or more responding cells
  \item[negative =] no responding cells
    \ei

  \item Imagine that the number of responding cells in a well is
    Poisson($\lambda_i$) for group $i$

    \vspace{4mm}

  \qquad Pr(no responding cells) = $e^{-\lambda_i}$

    \vspace{4mm}

  \qquad $\hat{\lambda}_i = -\log\left(\frac{\text{\# negative wells}}{\text{\# wells}}\right)$

    \ei

\end{frame}

\begin{frame}[c]{Analysis}
  \figh{Figs/simple_method.png}{0.9}
\end{frame}




\end{document}
