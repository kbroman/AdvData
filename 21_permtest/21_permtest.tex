\documentclass[aspectratio=169,12pt,t]{beamer}
\usepackage{graphicx}
\setbeameroption{hide notes}
\setbeamertemplate{note page}[plain]
\usepackage{listings}

\input{../LaTeX/header.tex}

%%%%%%%%%%%%%%%%%%%%%%%%%%%%%%%%%%%%%%%%%%%%%%%%%%%%%%%%%%%%%%%%%%%%%%
% end of header
%%%%%%%%%%%%%%%%%%%%%%%%%%%%%%%%%%%%%%%%%%%%%%%%%%%%%%%%%%%%%%%%%%%%%%

\title{Permutation tests}
\author{\href{https://kbroman.org}{Karl Broman}}
\institute{Biostatistics \& Medical Informatics, UW{\textendash}Madison}
\date{\href{https://kbroman.org}{\tt \scriptsize \color{foreground} kbroman.org}
\\[-4pt]
\href{https://github.com/kbroman}{\tt \scriptsize \color{foreground} github.com/kbroman}
\\[-4pt]
\href{https://twitter.com/kwbroman}{\tt \scriptsize \color{foreground} @kwbroman}
\\[-4pt]
{\scriptsize Course web: \href{https://kbroman.org/AdvData}{\tt kbroman.org/AdvData}}
}

\begin{document}

{
\setbeamertemplate{footline}{} % no page number here
\frame{
  \titlepage

\note{
  In this lecture, we'll look at permutation tests. When they are
  appropriate, I prefer them.
}
} }



\begin{frame}[c]{Randomized experiment}

\figh{Figs/experiment.pdf}{0.9}

\note{
   Consider a randomized experiment with two treatment groups, treated
   (T) and control (C).

   How to tell whether the treatment has an effect?
}
\end{frame}



\begin{frame}[c]{Experimental results}

\figh{Figs/experiment_results.pdf}{0.9}

\note{
  The standard t-test gives a p-value of 0.01.

  What assumptions are made here?

  I find the permutation test to be more natural, and it only relies
  on the assumption of random assignment of treatment groups.
}
\end{frame}




\begin{frame}[c]{Permuted results}

\only<1>{\figh{Figs/perm_results1.pdf}{0.9}}
\only<2|handout 0>{\figh{Figs/perm_results2.pdf}{0.9}}
\only<3|handout 0>{\figh{Figs/perm_results3.pdf}{0.9}}
\only<4|handout 0>{\figh{Figs/perm_results4.pdf}{0.9}}

\note{
  Here are results when you permute the treatment assignments. Do the
  observed results show a sufficiently strong effect that we can be
  confident that it's real?
}
\end{frame}




\end{document}
