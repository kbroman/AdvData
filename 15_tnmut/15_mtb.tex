\documentclass[aspectratio=169,12pt,t]{beamer}
\usepackage{graphicx}
\setbeameroption{hide notes}
\setbeamertemplate{note page}[plain]
\usepackage{listings}

% header.tex: boring LaTeX/Beamer details + macros

% get rid of junk
\usetheme{default}
\beamertemplatenavigationsymbolsempty
\hypersetup{pdfpagemode=UseNone} % don't show bookmarks on initial view


% font
\usepackage{fontspec}
\setsansfont
  [ ExternalLocation = ../fonts/ ,
    UprightFont = *-regular ,
    BoldFont = *-bold ,
    ItalicFont = *-italic ,
    BoldItalicFont = *-bolditalic ]{texgyreheros}
\setbeamerfont{note page}{family*=pplx,size=\footnotesize} % Palatino for notes
% "TeX Gyre Heros can be used as a replacement for Helvetica"
% I've placed them in fonts/; alternatively you can install them
% permanently on your system as follows:
%     Download http://www.gust.org.pl/projects/e-foundry/tex-gyre/heros/qhv2.004otf.zip
%     In Unix, unzip it into ~/.fonts
%     In Mac, unzip it, double-click the .otf files, and install using "FontBook"

% named colors
\definecolor{offwhite}{RGB}{255,250,240}
\definecolor{gray}{RGB}{155,155,155}
\definecolor{purple}{RGB}{177,13,201}
\definecolor{green}{RGB}{46,204,64}

\definecolor{background}{RGB}{255,255,255}
\definecolor{foreground}{RGB}{24,24,24}
\definecolor{title}{RGB}{27,94,134}
\definecolor{subtitle}{RGB}{22,175,124}
\definecolor{hilit}{RGB}{122,0,128}
\definecolor{vhilit}{RGB}{255,0,128}
\definecolor{codehilit}{RGB}{255,0,128}
\definecolor{lolit}{RGB}{95,95,95}
\definecolor{myyellow}{rgb}{1,1,0.7}
\definecolor{nhilit}{RGB}{128,0,128}  % hilit color in notes
\definecolor{nvhilit}{RGB}{255,0,128} % vhilit for notes

\newcommand{\hilit}{\color{hilit}}
\newcommand{\vhilit}{\color{vhilit}}
\newcommand{\nhilit}{\color{nhilit}}
\newcommand{\nvhilit}{\color{nvhilit}}
\newcommand{\lolit}{\color{lolit}}

% use those colors
\setbeamercolor{titlelike}{fg=title}
\setbeamercolor{subtitle}{fg=subtitle}
\setbeamercolor{institute}{fg=lolit}
\setbeamercolor{normal text}{fg=foreground,bg=background}
\setbeamercolor{item}{fg=foreground} % color of bullets
\setbeamercolor{subitem}{fg=lolit}
\setbeamercolor{itemize/enumerate subbody}{fg=lolit}
\setbeamertemplate{itemize subitem}{{\textendash}}
\setbeamerfont{itemize/enumerate subbody}{size=\footnotesize}
\setbeamerfont{itemize/enumerate subitem}{size=\footnotesize}

% page number
\setbeamertemplate{footline}{%
    \raisebox{5pt}{\makebox[\paperwidth]{\hfill\makebox[20pt]{\lolit
          \scriptsize\insertframenumber}}}\hspace*{5pt}}

% add a bit of space at the top of the notes page
\addtobeamertemplate{note page}{\setlength{\parskip}{12pt}}

% default link color
\hypersetup{colorlinks, urlcolor={hilit}}

\lstset{language=bash,
        basicstyle=\ttfamily\scriptsize,
        frame=single,
        commentstyle=,
        backgroundcolor=\color{offwhite},
        showspaces=false,
        showstringspaces=false
        }


% a few macros
\newcommand{\bi}{\begin{itemize}}
\newcommand{\bbi}{\vspace{24pt} \begin{itemize} \itemsep8pt}
\newcommand{\ei}{\end{itemize}}
\newcommand{\be}{\begin{enumerate}}
\newcommand{\bbe}{\vspace{24pt} \begin{enumerate} \itemsep8pt}
\newcommand{\ee}{\end{enumerate}}
\newcommand{\ig}{\includegraphics}
\newcommand{\subt}[1]{{\footnotesize \color{subtitle} {#1}}}
\newcommand{\ttsm}{\tt \small}
\newcommand{\ttfn}{\tt \footnotesize}
\newcommand{\figh}[2]{\centerline{\includegraphics[height=#2\textheight]{#1}}}
\newcommand{\figw}[2]{\centerline{\includegraphics[width=#2\textwidth]{#1}}}


%%%%%%%%%%%%%%%%%%%%%%%%%%%%%%%%%%%%%%%%%%%%%%%%%%%%%%%%%%%%%%%%%%%%%%
% end of header
%%%%%%%%%%%%%%%%%%%%%%%%%%%%%%%%%%%%%%%%%%%%%%%%%%%%%%%%%%%%%%%%%%%%%%

\title{Bayesian analysis}
\subtitle{Identifying essential genes by mutagenesis}
\author{\href{https://kbroman.org}{Karl Broman}}
\institute{Biostatistics \& Medical Informatics, UW{\textendash}Madison}
\date{\href{https://kbroman.org}{\tt \scriptsize \color{foreground} kbroman.org}
\\[-4pt]
\href{https://github.com/kbroman}{\tt \scriptsize \color{foreground} github.com/kbroman}
\\[-4pt]
\href{https://twitter.com/kwbroman}{\tt \scriptsize \color{foreground} @kwbroman}
\\[-4pt]
{\scriptsize Course web: \href{https://kbroman.org/AdvData}{\tt kbroman.org/AdvData}}
}

\begin{document}

{
\setbeamertemplate{footline}{} % no page number here
\frame{
  \titlepage

\note{
  In this lecture, I'll present a case study on random transposon
  mutagenesis in Mycobacterium tuberculosis. Traditional frequentist
  methods are not suitable for this problem, and so I resorted to
  using Bayesian statistics, with Markov chain Monte Carlo.
}
} }



\begin{frame}{Mycobacterium tuberculosis}

  \bbi
\item The organism that causes tuberculosis.
  \bi
\item Cost for treatment: $\sim$\$15,000
\item Other bacterial pneumonias: $\sim$\$35
  \ei

\item 4.4 Mbp circular genome, completely sequenced
\item 4250 known or inferred genes
  \ei

\note{
  Tuberculosis can be surprisingly difficult and expensive to treat.
  So there's good reason to try to better understand its genome, to
  identify potential targets for new drugs.
}


\end{frame}




\begin{frame}[c]{}

  \centerline{\Large {\color{title} Goal}: identify the {\hilit essential} genes}

  \bigskip
  \bigskip
  \bigskip
  \bigskip

  \centerline{\Large {\color{title} Method}: random transposon mutagenesis}

\note{
  The current project seeks to identify the {\hilit essential} genes
  in the M.\ tuberculosis genome; the ones that if you knock them out,
  you get a non-viable mutant.

  This is done by random transposon mutageneis. A transposon is a bit
  of DNA that likes to insert itself into other DNA. So we'll randomly
  disrupt genes to find out which ones matter and which ones don't so much.
}

\end{frame}


\begin{frame}{\emph{Himar1\/} transposon}


\centerline{\includegraphics[width=0.7\textwidth]{Figs/himar.pdf}}

\vspace{-17mm}
\begin{center}
{\tt
\hspace*{2.5mm} 5'-TCGAAGCCTGCGAC{\vhilit
\textbf{TA}}ACGTT{\vhilit \textbf{TA}}AAGTTTG-3'

\hspace*{2.5mm} 3'-AGCTTCGGACGCTG{\vhilit
\textbf{AT}}TGCAA{\vhilit \textbf{AT}}TTCAAAC-5'
}
\end{center}

\hfill

\centerline{\hilit Note: $\ge$ 30 stop codons in each reading frame}

\note{
   They used the \emph{Himar1\/} transposon which randomly inserts
   itself into a site {\hilit TA}.

   ``Stop codons'' are codes for where a gene stops. The point is that
   if this thing gets inserted into a gene, it will end up with a
   truncated product.
  }
\end{frame}





\begin{frame}[c]{Sequence of the gene MT598}

\figw{Figs/mt598.pdf}{1.0}

\note{
}

\end{frame}


\begin{frame}[c]{Random transposon mutagenesis}

\only<1|handout 0>{\figh{Figs/mut1.pdf}{0.9}}
\only<2|handout 0>{\figh{Figs/mut2.pdf}{0.9}}
\only<3|handout 0>{\figh{Figs/mut3.pdf}{0.9}}
\only<4|handout 0>{\figh{Figs/mut4.pdf}{0.9}}
\only<5|handout 0>{\figh{Figs/mut5.pdf}{0.9}}
\only<6>{\figh{Figs/mut6.pdf}{0.9}}

\note{
}

\end{frame}


\begin{frame}{Random transposon mutagenesis}

  \bbi
  \item Location of transposon insertion determined by sequencing across
junctions

\item Viable insertion within a gene $\implies$ gene is non-essential

\item Essential genes: we will never see a viable insertion

\item {\hilit Complication}: Insertions in the very distal portion of
  an essential gene may not be sufficiently disruptive.

Thus, we omit from consideration insertions sites within the last 20\%
and last 100 bp of a gene.
\ei

\note{
}
\end{frame}






\begin{frame}{The data}

  \bbi
\item Number, locations of genes
\item Number of insertion sites in each gene
\item {\hilit $n$} viable mutants with exactly one transposon
    insertion
\item Location of the transposon insertion in each mutant
\ei

\note{
}
\end{frame}



\begin{frame}{TA sites in M. tuberculosis}

  \figh{Figs/numTAs.pdf}{0.4}

  \bbi
\item 74,403 sites
\item 65,649 sites within a gene
\item 57,934 sites within proximal portion of a gene
\item 4204/4250 genes with at least one TA site
  \ei

\note{
}
\end{frame}



\end{document}
