\documentclass[aspectratio=169,12pt,t]{beamer}
\usepackage{graphicx}
\setbeameroption{hide notes}
\setbeamertemplate{note page}[plain]
\usepackage{listings}

% header.tex: boring LaTeX/Beamer details + macros

% get rid of junk
\usetheme{default}
\beamertemplatenavigationsymbolsempty
\hypersetup{pdfpagemode=UseNone} % don't show bookmarks on initial view


% font
\usepackage{fontspec}
\setsansfont
  [ ExternalLocation = ../fonts/ ,
    UprightFont = *-regular ,
    BoldFont = *-bold ,
    ItalicFont = *-italic ,
    BoldItalicFont = *-bolditalic ]{texgyreheros}
\setbeamerfont{note page}{family*=pplx,size=\footnotesize} % Palatino for notes
% "TeX Gyre Heros can be used as a replacement for Helvetica"
% I've placed them in fonts/; alternatively you can install them
% permanently on your system as follows:
%     Download http://www.gust.org.pl/projects/e-foundry/tex-gyre/heros/qhv2.004otf.zip
%     In Unix, unzip it into ~/.fonts
%     In Mac, unzip it, double-click the .otf files, and install using "FontBook"

% named colors
\definecolor{offwhite}{RGB}{255,250,240}
\definecolor{gray}{RGB}{155,155,155}
\definecolor{purple}{RGB}{177,13,201}
\definecolor{green}{RGB}{46,204,64}

\definecolor{background}{RGB}{255,255,255}
\definecolor{foreground}{RGB}{24,24,24}
\definecolor{title}{RGB}{27,94,134}
\definecolor{subtitle}{RGB}{22,175,124}
\definecolor{hilit}{RGB}{122,0,128}
\definecolor{vhilit}{RGB}{255,0,128}
\definecolor{codehilit}{RGB}{255,0,128}
\definecolor{lolit}{RGB}{95,95,95}
\definecolor{myyellow}{rgb}{1,1,0.7}
\definecolor{nhilit}{RGB}{128,0,128}  % hilit color in notes
\definecolor{nvhilit}{RGB}{255,0,128} % vhilit for notes

\newcommand{\hilit}{\color{hilit}}
\newcommand{\vhilit}{\color{vhilit}}
\newcommand{\nhilit}{\color{nhilit}}
\newcommand{\nvhilit}{\color{nvhilit}}
\newcommand{\lolit}{\color{lolit}}

% use those colors
\setbeamercolor{titlelike}{fg=title}
\setbeamercolor{subtitle}{fg=subtitle}
\setbeamercolor{institute}{fg=lolit}
\setbeamercolor{normal text}{fg=foreground,bg=background}
\setbeamercolor{item}{fg=foreground} % color of bullets
\setbeamercolor{subitem}{fg=lolit}
\setbeamercolor{itemize/enumerate subbody}{fg=lolit}
\setbeamertemplate{itemize subitem}{{\textendash}}
\setbeamerfont{itemize/enumerate subbody}{size=\footnotesize}
\setbeamerfont{itemize/enumerate subitem}{size=\footnotesize}

% page number
\setbeamertemplate{footline}{%
    \raisebox{5pt}{\makebox[\paperwidth]{\hfill\makebox[20pt]{\lolit
          \scriptsize\insertframenumber}}}\hspace*{5pt}}

% add a bit of space at the top of the notes page
\addtobeamertemplate{note page}{\setlength{\parskip}{12pt}}

% default link color
\hypersetup{colorlinks, urlcolor={hilit}}

\lstset{language=bash,
        basicstyle=\ttfamily\scriptsize,
        frame=single,
        commentstyle=,
        backgroundcolor=\color{offwhite},
        showspaces=false,
        showstringspaces=false
        }


% a few macros
\newcommand{\bi}{\begin{itemize}}
\newcommand{\bbi}{\vspace{24pt} \begin{itemize} \itemsep8pt}
\newcommand{\ei}{\end{itemize}}
\newcommand{\be}{\begin{enumerate}}
\newcommand{\bbe}{\vspace{24pt} \begin{enumerate} \itemsep8pt}
\newcommand{\ee}{\end{enumerate}}
\newcommand{\ig}{\includegraphics}
\newcommand{\subt}[1]{{\footnotesize \color{subtitle} {#1}}}
\newcommand{\ttsm}{\tt \small}
\newcommand{\ttfn}{\tt \footnotesize}
\newcommand{\figh}[2]{\centerline{\includegraphics[height=#2\textheight]{#1}}}
\newcommand{\figw}[2]{\centerline{\includegraphics[width=#2\textwidth]{#1}}}


% shortcuts
\newcommand{\yi}{\text{y}_\text{i}}
\newcommand{\n}{\text{n}}
\newcommand{\xxi}{\text{x}_\text{i}}
\newcommand{\xxj}{\text{x}_\text{j}}
\newcommand{\one}{\text{1}}
\newcommand{\zero}{\text{0}}
\newcommand{\boldy}{\text{\textbf{y}}}
\newcommand{\thetai}{{\vhilit \theta_\text{i}}}
\newcommand{\thetaj}{{\vhilit \theta_\text{j}}}
\newcommand{\sumi}{\sum_\text{i}}
\newcommand{\sumj}{\sum_\text{j}}
\newcommand{\boldtheta}{{\vhilit \boldsymbol{\theta}}}



%%%%%%%%%%%%%%%%%%%%%%%%%%%%%%%%%%%%%%%%%%%%%%%%%%%%%%%%%%%%%%%%%%%%%%
% end of header
%%%%%%%%%%%%%%%%%%%%%%%%%%%%%%%%%%%%%%%%%%%%%%%%%%%%%%%%%%%%%%%%%%%%%%

\title{Bayesian analysis}
\subtitle{Identifying essential genes by mutagenesis}
\author{\href{https://kbroman.org}{Karl Broman}}
\institute{Biostatistics \& Medical Informatics, UW{\textendash}Madison}
\date{\href{https://kbroman.org}{\tt \scriptsize \color{foreground} kbroman.org}
\\[-4pt]
\href{https://github.com/kbroman}{\tt \scriptsize \color{foreground} github.com/kbroman}
\\[-4pt]
\href{https://twitter.com/kwbroman}{\tt \scriptsize \color{foreground} @kwbroman}
\\[-4pt]
{\scriptsize Course web: \href{https://kbroman.org/AdvData}{\tt kbroman.org/AdvData}}
}

\begin{document}

{
\setbeamertemplate{footline}{} % no page number here
\frame{
  \titlepage

\note{
  In this lecture, I'll present a case study on random transposon
  mutagenesis in Mycobacterium tuberculosis. Traditional frequentist
  methods are not suitable for this problem, and so I resorted to
  using Bayesian statistics, with Markov chain Monte Carlo.
}
} }



\begin{frame}{Mycobacterium tuberculosis}

  \bbi
\item The organism that causes tuberculosis.
  \bi
\item Cost for treatment: $\sim$\$15,000
\item Other bacterial pneumonias: $\sim$\$35
  \ei

\item 4.4 Mbp circular genome, completely sequenced
\item 4250 known or inferred genes
  \ei

\note{
  Tuberculosis can be surprisingly difficult and expensive to treat.
  So there's good reason to try to better understand its genome, to
  identify potential targets for new drugs.
}


\end{frame}




\begin{frame}[c]{}

  \centerline{\Large {\color{title} Goal}: identify the {\hilit essential} genes}

  \bigskip
  \bigskip
  \bigskip
  \bigskip

  \centerline{\Large {\color{title} Method}: random transposon mutagenesis}

\note{
  The current project seeks to identify the {\hilit essential} genes
  in the M.\ tuberculosis genome; the ones that if you knock them out,
  you get a non-viable mutant.

  This is done by random transposon mutageneis. A transposon is a bit
  of DNA that likes to insert itself into other DNA. So we'll randomly
  disrupt genes to find out which ones matter and which ones don't so much.
}

\end{frame}


\begin{frame}{\emph{Himar1\/} transposon}


\centerline{\includegraphics[width=0.7\textwidth]{Figs/himar.pdf}}

\vspace{-17mm}
\begin{center}
{\tt
\hspace*{2.5mm} 5'-TCGAAGCCTGCGAC{\vhilit
\textbf{TA}}ACGTT{\vhilit \textbf{TA}}AAGTTTG-3'

\hspace*{2.5mm} 3'-AGCTTCGGACGCTG{\vhilit
\textbf{AT}}TGCAA{\vhilit \textbf{AT}}TTCAAAC-5'
}
\end{center}

\hfill

\centerline{\hilit Note: $\ge$ 30 stop codons in each reading frame}

\note{
   They used the \emph{Himar1\/} transposon which randomly inserts
   itself into a site {\hilit TA}.

   ``Stop codons'' are codes for where a gene stops. The point is that
   if this thing gets inserted into a gene, it will end up with a
   truncated product.
  }
\end{frame}





\begin{frame}[c]{Sequence of the gene MT598}

\figw{Figs/mt598.pdf}{1.0}

\note{
}

\end{frame}


\begin{frame}[c]{Random transposon mutagenesis}

\only<1|handout 0>{\figh{Figs/mut1.pdf}{0.9}}
\only<2|handout 0>{\figh{Figs/mut2.pdf}{0.9}}
\only<3|handout 0>{\figh{Figs/mut3.pdf}{0.9}}
\only<4|handout 0>{\figh{Figs/mut4.pdf}{0.9}}
\only<5|handout 0>{\figh{Figs/mut5.pdf}{0.9}}
\only<6>{\figh{Figs/mut6.pdf}{0.9}}

\note{
}

\end{frame}


\begin{frame}{Random transposon mutagenesis}

  \bbi
  \item Location of transposon insertion determined by sequencing across
junctions

\item Viable insertion within a gene $\implies$ gene is non-essential

\item Essential genes: we will never see a viable insertion

\item {\hilit Complication}: Insertions in the very distal portion of
  an essential gene may not be sufficiently disruptive.

Thus, we omit from consideration insertions sites within the last 20\%
and last 100 bp of a gene.
\ei

\note{
}
\end{frame}






\begin{frame}{The data}

  \bbi
\item Number, locations of genes
\item Number of insertion sites in each gene
\item {\hilit $n$} viable mutants with exactly one transposon
    insertion
\item Location of the transposon insertion in each mutant
\ei

\note{
}
\end{frame}



\begin{frame}{TA sites in M. tuberculosis}

  \figh{Figs/numTAs.pdf}{0.4}

  \bbi
\item 74,403 sites
\item 65,649 sites within a gene
\item 57,934 sites within proximal portion of a gene
\item 4204/4250 genes with at least one TA site
  \ei

\note{
}
\end{frame}


\begin{frame}{1425 insertion mutants}

  \begin{columns}

    \column{0.4\textwidth}

      \vspace{-10mm}
      \figw{Figs/circlefig.pdf}{1.0}

    \column{0.6\textwidth}

    \bi
    \item 1425 insertion mutants
    \item 1025 within proximal portion of a gene
    \item 21 double-hits
    \item 770 unique genes hit
    \ei
  \end{columns}

\only<2->{
\hfill \begin{minipage}{0.9\textwidth}
  \bbi
  \item[\color{title} Questions:] {\hilit Proportion of essential genes in M.\ tuberculosis?}
  \item[] {\hilit Which genes are likely essential?}
  \ei
\end{minipage}
}

\note{
}
\end{frame}



\begin{frame}{Model}

  {\color{title} Transposon inserts completely at random}

  \bigskip

  \bi
\item Each TA site equally likely
\item Genes are either completely essential or completely
  non-essential
  \ei


  \note{
  }
  \end{frame}


\begin{frame}{Model}

\begin{columns}

\column{0.15\textwidth}

N genes \\[12pt]

n mutants

\column{0.85\textwidth}

$x_i$ = no. TA sites in gene $i$ \\[12pt]

$y_i$ = no. mutants with insertion in gene $i$.
\end{columns}


\bigskip\bigskip\bigskip

  $$  \theta_i = \left\{ \begin{array}{c} 1 \\ 0 \end{array} \text{ if gene $i$ is  }
      \begin{array}{c} \text{non-essential}  \\ \text{essential} \end{array} \right. $$


      \bigskip\bigskip\bigskip

      {\color{title} Model}: $\boldsymbol{y} \sim
      \text{multinomial}(n,\boldsymbol{p})$ \qquad where $p_i = x_i  \theta_i / \sum_j x_j \theta_j$

\bigskip\bigskip

      {\color{title} Goal}: Estimate $\theta_+ = \sum_i \theta_i$
      \qquad or \qquad $1 - \theta_+/N$

  \note{
  }
\end{frame}



\begin{frame}{The likelihood}


  \begin{eqnarray*}
    L(\boldsymbol{\theta} \ | \ \boldsymbol{y}) & = & {n \choose y} \prod_i (x_i \theta_i)^{y_i} / \sum_j (x_j \theta_j)^n \\[14pt]
                     & \propto & \left\{ \begin{array}{cl} (\sum_i x_i  \theta_i)^{-n} & \text{if $\theta_i$ = 1 whenever $y_i >$ 0} \\[12pt] 0  & \text{otherwise} \end{array}\right.
  \end{eqnarray*}



  \bigskip \bigskip

  {\color{title} Notes:}
  \bi
  \item Depends only on which $y_i>0$ and not on the specific values
  \item The MLE is $\hat{\theta}_i = 1\{y_i > 0\}$
    \ei


\note{
}
\end{frame}




\begin{frame}{The prior}

  \bigskip

  $\theta_+ \sim$ uniform on \{ 0, 1, ..., N \}

  \bigskip

$\boldsymbol{\theta} \ | \ \theta_+ \sim$  uniform over all sequences of 0's and 1's with $\theta_+$ 1's


  \bigskip \bigskip

  {\color{title} Notes:}
  \bi
\item We are assuming that $\text{Pr}(\theta_i = 1) = 1/2$
\item This is quite different from taking $\theta_i$ iid
  Bernoulli(1/2)
\item We are assuming that $\theta_i$ is independent of $x_i$ and the
  length of the gene
\item We could make use of information about the essential status of
  particular genes (e.g. known viable knock-outs)
  \ei


\note{
}

\end{frame}



\begin{frame}[c]{}

\figw{Figs/unif_v_binom.pdf}{1.0}

\note{
}
\end{frame}



\begin{frame}{A Gibbs sampler}

\bigskip

  {\color{title} Goal}: Estimate $\text{Pr}(\boldsymbol{\theta} \ | \ \boldsymbol{y})$

\bigskip

{\color{title} Gibbs sampler:}
\bi
\item Begin with some initial assignment $\boldsymbol{\theta}^{(0)}$
\item For iteration $s$, consider each gene one at a time
 \bi
  \item Let $\boldsymbol{\theta}_{-i}^{(s)} = (\theta_1^{(s+1)}, ..., \theta_{i-1}^{(s+1)}, \theta_{i+1}^{(s)}, ..., \theta_N^{(s)})$
  \item Calculate $\text{Pr}(\theta_i = 1 \ | \ \boldsymbol{\theta}_{-i}^{(s)}, \boldsymbol{y})$
  \item Assign $\theta_i^{(s)} = 1$ at random with that probability
 \ei
\item Repeat many times
  \ei

\bigskip \bigskip

This is an example of {\vhilit Markov chain Monte Carlo (MCMC)}.

  \note{
  }
\end{frame}


\begin{frame}[c]{MCMC in action}

\only<1|handout 0>{\figw{Figs/mcmc1.pdf}{1.0}}
\only<2|handout 0>{\figw{Figs/mcmc2.pdf}{1.0}}
\only<3|handout 0>{\figw{Figs/mcmc3.pdf}{1.0}}
\only<4|handout 0>{\figw{Figs/mcmc4.pdf}{1.0}}
\only<5|handout 0>{\figw{Figs/mcmc5.pdf}{1.0}}
\only<6|handout 0>{\figw{Figs/mcmc6.pdf}{1.0}}
\only<7|handout 0>{\figw{Figs/mcmc7.pdf}{1.0}}
\only<8|handout 0>{\figw{Figs/mcmc8.pdf}{1.0}}
\only<9|handout 0>{\figw{Figs/mcmc9.pdf}{1.0}}
\only<10|handout 0>{\figw{Figs/mcmc10.pdf}{1.0}}
\only<11|handout 0>{\figw{Figs/mcmc11.pdf}{1.0}}
\only<12|handout 0>{\figw{Figs/mcmc12.pdf}{1.0}}
\only<13>{\figw{Figs/mcmc13.pdf}{1.0}}

\note{
}
\end{frame}


\begin{frame}{The conditional probabilities}

If $y_i > 0$, then $\text{Pr}({\vhilit \theta_i} =
1 \ | \ \boldsymbol{y}, {\vhilit
\boldsymbol{\theta}_\text{-i}^\text{(s)}}) = 1$

\vspace{10mm}

If $y_i = 0$,

\vspace{-15mm}

\begin{eqnarray*}
\text{{\hilit Let} A } & = & \textstyle{\sum_{\text{j} < \text{i}} {\vhilit
\theta_j^\text{(s+1)}} + \sum_{\text{j} > \text{i}} {\vhilit
\theta_j^\text{(s)}}} \\
\text{B } & = & \textstyle{\sum_{\text{j} < \text{i}}
x_j \ {\vhilit
\theta_j^\text{(s+1)}} + \sum_{\text{j} > \text{i}}
x_j \ {\vhilit
\theta_j^\text{(s)}}} \\
\\
\text{\hilit Then } \text{Pr}({\vhilit
\boldsymbol{\theta}_\text{-i}^\text{(s)}}, {\vhilit
\theta_i}=\text{k}) & = & \textstyle{{n \choose
\text{A}+\text{k}} / n} \\
\text{Pr}(\boldsymbol{y} \ | \ {\vhilit
\boldsymbol{\theta}_\text{-i}^\text{(s)}}, {\vhilit
\theta_i}=\text{k}) & = & \textstyle{(\text{B} + \text{k} \
x_i)^\text{-n}}
\\ \\
\text{\hilit And so } \text{Pr}({\vhilit \theta_i} = 1
\ | \ \boldsymbol{y}, {\vhilit
\boldsymbol{\theta}_\text{-i}^\text{(s)}}) & = &  \dots \\
%\textstyle{\frac{{n \choose \text{A}+1} (\text{B} +
%x_i)^\text{-n} / n}{{n \choose \text{A}+1} (\text{B} +
%x_i)^\text{-n} / n + {n \choose \text{A}}
%(\text{B})^\text{-n} / n}} \\
& = & \frac{(1 +
x_i/\text{B})^\text{-n}}{(1 +
x_i/\text{B})^\text{-n} + (n - \text{A}) /
(\text{A} + 1)}
\end{eqnarray*}


\note{
}
\end{frame}




\begin{frame}{Estimators}


The Gibbs sampler produces ${\vhilit
\boldtheta^\text{(0)}},  {\vhilit
\boldtheta^\text{(1)}}, \dots, {\vhilit
\boldtheta^\text{(S)}}$

\smallskip

We discard the first 200 or so samples (``burn-in'').

\bigskip
\bigskip

{\color{title} Estimated number of non-essential genes}:
$\text{E}({\vhilit \theta_+}  \ | \ \boldy)$

\smallskip

\hspace{30mm} ${\vhilit \theta_+^\text{(s)}} = \sumi
{\vhilit \thetai^\text{(s)}}$ \hspace{15mm}
${\color{title} \longrightarrow}$ \hspace{15mm}
${\vhilit \hat{\theta}_+} = \textstyle{\frac{\one}{\text{S} -
\text{200}} \sum_{\text{s}=\text{201}}^\text{S} {\vhilit \theta_+^\text{(s)}}}$

\bigskip
\bigskip

{\color{title} Probability that gene i is non-essential}:
$\text{E}({\vhilit \thetai} \ | \ \boldy)$ =
$\text{Pr}({\vhilit \thetai} = \one \ | \ \boldy)$

\smallskip

\hspace{30mm} ${\vhilit \hat{\theta}_\text{i}} =
\textstyle{\frac{\one}{\text{S} - \text{200}} \sum_{\text{s} =
\text{201}}^\text{S} {\vhilit \thetai^\text{(s)}}}$
%\hfill  [${\vhilit \hat{\theta}_\text{i}}$ is really
%$\hat{\text{Pr}}({\vhilit \thetai} = \one \ | \ \boldy)$.]


\bigskip
\bigskip

\hspace{10mm} {\color{title} or Rao-Blackwellize:}

\smallskip

\hspace{30mm} ${\vhilit \hat{\theta}_\text{i}^\star} =
\textstyle{\frac{\one}{\text{S} - \text{200}} \sum_{\text{s} =
\text{201}}^\text{S} \text{Pr}({\vhilit \thetai} =
\one \ | \ \boldy, {\vhilit \boldtheta_\text{-i}^\text{(s)}})}$



\note{
}
\end{frame}







\begin{frame}{A further complication}

\begin{columns}

  \column{0.5\textwidth}

{\color{title} Many genes overlap}

\bi

\item Of 4250 genes, 1005 pairs overlap (mostly by exactly 4 bp).
\item The overlapping regions contain 547 insertion sites.
\item {\vhilit Omit TA sites in overlapping regions, unless in the
proximal portion of \emph{both\/} genes}.
\item The algebra gets a bit more complicated.
\ei

  \column{0.5\textwidth}

\figh{Figs/overlap.pdf}{0.9}

\end{columns}

\note{
}
\end{frame}




\begin{frame}[c]{Percent essential genes in M.\ tb.}

\figw{Figs/mtb_mcmc.pdf}{1.0}

\note{
}
\end{frame}


\begin{frame}[c]{Percent essential genes in M.\ tb.}

\figw{Figs/mtb_hist.pdf}{1.0}

\note{
}
\end{frame}



\begin{frame}[c]{Probability each gene is essential}

\figw{Figs/prob_essential.pdf}{1.0}

\note{
}
\end{frame}




\begin{frame}{Yet another complication}

  \begin{columns}
    \column{0.15\textwidth}

\hfill {\color{title} Operon:}

\column{0.85\textwidth}

A group of adjacent genes that are transcribed
together as a single unit.
\end{columns}


\figw{Figs/operon.pdf}{1.0}

\bi
\item Insertion at a TA site could disrupt all downstream genes

\item If a gene is essential, insertion in any upstream gene would be
non-viable

\item Re-define the meaning of ``essential gene''.

\item If operons were known, one could get an improved estimate of the
proportion of essential genes.

\item If one ignores the presence of operons, estimates should still
be unbiased.
\ei

\note{
}
\end{frame}




\begin{frame}[c]{Frequentist properties}

  \figw{Figs/sim_fig.pdf}{1.0}

\note{
}
\end{frame}


\begin{frame}{Summary}

  \bbi
\item Bayesian method, using MCMC, to estimate the proportion of
essential genes in a genome with data from random transposon
mutagenesis.

\item Crucial assumptions:

\bi
\item {\hilit Randomness of transposon insertion.}

\item Essentiality is an all-or-none quality.

\item No relationship between essentiality and no.\ insertion
sites.

\item The 80\% rule.
\ei

\item For \emph{M.\ tuberculosis}, with data on 1400 mutants:
\bi
\item {\color{title} 28 -- 41\%} of genes are essential
\item 20 genes which have $\ge$ 64 TA sites and for which no mutant has been
observed, have {\color{title} $>$ 75\%} chance of being essential.
\ei

\ei

\note{
}
\end{frame}


\begin{frame}{References}

  \bbi
\item Lamichhane et al. (2003) Proc Natl Acad Sci USA 100:7213-7218
    \href{https://doi.org/10.1073/pnas.1231432100}{doi:10.1073/pnas.1231432100}

  \item Blades and Broman (2002) Tech Report MS02-20

\href{https://www.biostat.wisc.edu/~kbroman/publications/ms0220.pdf}{\tt bit.ly/ms0220}

\item R/negenes package

  \href{https://cran.r-project.org/package=negenes}{\tt cran.r-project.org/package=negenes}


    \ei

    \note{
      }
\end{frame}

\end{document}
