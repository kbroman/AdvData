\documentclass[aspectratio=169,12pt,t]{beamer}
\usepackage{graphicx}
\setbeameroption{hide notes}
\setbeamertemplate{note page}[plain]
\usepackage{listings}
\usepackage{eepic}

% header.tex: boring LaTeX/Beamer details + macros

% get rid of junk
\usetheme{default}
\beamertemplatenavigationsymbolsempty
\hypersetup{pdfpagemode=UseNone} % don't show bookmarks on initial view


% font
\usepackage{fontspec}
\setsansfont
  [ ExternalLocation = ../fonts/ ,
    UprightFont = *-regular ,
    BoldFont = *-bold ,
    ItalicFont = *-italic ,
    BoldItalicFont = *-bolditalic ]{texgyreheros}
\setbeamerfont{note page}{family*=pplx,size=\footnotesize} % Palatino for notes
% "TeX Gyre Heros can be used as a replacement for Helvetica"
% I've placed them in fonts/; alternatively you can install them
% permanently on your system as follows:
%     Download http://www.gust.org.pl/projects/e-foundry/tex-gyre/heros/qhv2.004otf.zip
%     In Unix, unzip it into ~/.fonts
%     In Mac, unzip it, double-click the .otf files, and install using "FontBook"

% named colors
\definecolor{offwhite}{RGB}{255,250,240}
\definecolor{gray}{RGB}{155,155,155}
\definecolor{purple}{RGB}{177,13,201}
\definecolor{green}{RGB}{46,204,64}

\definecolor{background}{RGB}{255,255,255}
\definecolor{foreground}{RGB}{24,24,24}
\definecolor{title}{RGB}{27,94,134}
\definecolor{subtitle}{RGB}{22,175,124}
\definecolor{hilit}{RGB}{122,0,128}
\definecolor{vhilit}{RGB}{255,0,128}
\definecolor{codehilit}{RGB}{255,0,128}
\definecolor{lolit}{RGB}{95,95,95}
\definecolor{myyellow}{rgb}{1,1,0.7}
\definecolor{nhilit}{RGB}{128,0,128}  % hilit color in notes
\definecolor{nvhilit}{RGB}{255,0,128} % vhilit for notes

\newcommand{\hilit}{\color{hilit}}
\newcommand{\vhilit}{\color{vhilit}}
\newcommand{\nhilit}{\color{nhilit}}
\newcommand{\nvhilit}{\color{nvhilit}}
\newcommand{\lolit}{\color{lolit}}

% use those colors
\setbeamercolor{titlelike}{fg=title}
\setbeamercolor{subtitle}{fg=subtitle}
\setbeamercolor{institute}{fg=lolit}
\setbeamercolor{normal text}{fg=foreground,bg=background}
\setbeamercolor{item}{fg=foreground} % color of bullets
\setbeamercolor{subitem}{fg=lolit}
\setbeamercolor{itemize/enumerate subbody}{fg=lolit}
\setbeamertemplate{itemize subitem}{{\textendash}}
\setbeamerfont{itemize/enumerate subbody}{size=\footnotesize}
\setbeamerfont{itemize/enumerate subitem}{size=\footnotesize}

% page number
\setbeamertemplate{footline}{%
    \raisebox{5pt}{\makebox[\paperwidth]{\hfill\makebox[20pt]{\lolit
          \scriptsize\insertframenumber}}}\hspace*{5pt}}

% add a bit of space at the top of the notes page
\addtobeamertemplate{note page}{\setlength{\parskip}{12pt}}

% default link color
\hypersetup{colorlinks, urlcolor={hilit}}

\lstset{language=bash,
        basicstyle=\ttfamily\scriptsize,
        frame=single,
        commentstyle=,
        backgroundcolor=\color{offwhite},
        showspaces=false,
        showstringspaces=false
        }


% a few macros
\newcommand{\bi}{\begin{itemize}}
\newcommand{\bbi}{\vspace{24pt} \begin{itemize} \itemsep8pt}
\newcommand{\ei}{\end{itemize}}
\newcommand{\be}{\begin{enumerate}}
\newcommand{\bbe}{\vspace{24pt} \begin{enumerate} \itemsep8pt}
\newcommand{\ee}{\end{enumerate}}
\newcommand{\ig}{\includegraphics}
\newcommand{\subt}[1]{{\footnotesize \color{subtitle} {#1}}}
\newcommand{\ttsm}{\tt \small}
\newcommand{\ttfn}{\tt \footnotesize}
\newcommand{\figh}[2]{\centerline{\includegraphics[height=#2\textheight]{#1}}}
\newcommand{\figw}[2]{\centerline{\includegraphics[width=#2\textwidth]{#1}}}


%%%%%%%%%%%%%%%%%%%%%%%%%%%%%%%%%%%%%%%%%%%%%%%%%%%%%%%%%%%%%%%%%%%%%%
% end of header
%%%%%%%%%%%%%%%%%%%%%%%%%%%%%%%%%%%%%%%%%%%%%%%%%%%%%%%%%%%%%%%%%%%%%%

% title info
\title{Model misspecification}
\subtitle{Estimating allele frequencies in sibships}
\author{\href{https://kbroman.org}{Karl Broman}}
\institute{Biostatistics \& Medical Informatics, UW{\textendash}Madison}
\date{\href{https://kbroman.org}{\tt \scriptsize \color{foreground} kbroman.org}
\\[-4pt]
\href{https://github.com/kbroman}{\tt \scriptsize \color{foreground} github.com/kbroman}
\\[-4pt]
\href{https://twitter.com/kwbroman}{\tt \scriptsize \color{foreground} @kwbroman}
\\[-4pt]
{\scriptsize Course web: \href{https://kbroman.org/AdvData}{\tt kbroman.org/AdvData}}
}


\begin{document}

% title slide
{
\setbeamertemplate{footline}{} % no page number here
\frame{
  \titlepage

  \note{}

} }


\begin{frame}[c]{Mapping disease genes}

  \centering
  \Large

  Look for genomic regions where \hspace*{20mm} \\[18pt]

  individuals with {\hilit similar phenotypes}

  also have {\hilit similar genotypes}

\end{frame}


\begin{frame}[c]{Affected sib pairs}
\only<1>{\figw{Figs/sibpairs.pdf}{1.0}}
\only<2>{\figw{Figs/sibpairs_wdata.pdf}{1.0}}
\end{frame}


\begin{frame}[c]{IBS vs IBD}


\large

IBS = identical by {\hilit state}

{\color{background} IBS}    = same allele number \\[18pt]

  IBD = identical by {\hilit descent}

{\color{background} IBD} = copies of the same ancestral allele \\[36pt]

non-inbred sibs are IBD = 0, 1, 2

{\color{background} non-inbred sibs} with probability = 1/4, 1/2, 1/4


\end{frame}

\begin{frame}[c]{Prostate cancer genome scan}
\figw{Figs/gh_results_bad.pdf}{1.0}
\end{frame}

\begin{frame}[c]{Lesson}

\centering
\Large
If it seems too good to be true, \\[12pt]
it probably is.

\end{frame}



\begin{frame}[c]{Prostate cancer pairs}
\only<1>{\figw{Figs/sibpairs_nopar.pdf}{1.0}}
\only<2>{\figw{Figs/sibpairs_nopar_wdata.pdf}{1.0}}
\end{frame}



\begin{frame}[c]{Prostate cancer genome scan -- corrected}
\figw{Figs/gh_results_good.pdf}{1.0}
\end{frame}


\begin{frame}{Estimating allele frequencies}

  \bigskip

  Usually, you would use the {\hilit founders} in the pedigrees. \\
  \qquad (assumed unrelated)

  \bigskip \bigskip

  What if you only have {\vhilit sibships}?

\end{frame}



\begin{frame}{Estimating allele frequencies with sibpairs}

  \bigskip

  {\hilit Method 1}: Use a random sibling from each

  \onslide<2->{

    $$ \text{var}(\hat{p}^{(1)}) = \frac{p(1-p)}{2n}$$

  }


  \bigskip \bigskip \bigskip

  {\hilit Method 2}: Use everyone, ignoring relationships


  \onslide<2->{

    $$ \text{var}(\hat{p}^{(2)}) = \only<2>{\, \text{\bf ?}}
    \only<3->{{\vhilit (3/4)} \, \left(\frac{p(1-p)}{2n}\right)}
    $$
  }

  \only<4->{\qquad relative efficiency = 4/3 = 1.33

    \qquad \qquad {\lolit (best possible = 1.5)}}

\end{frame}



\begin{frame}[c]{My favorite equations}
  \begin{eqnarray*}
    \text{E}(X) & = & \text{E}[ \text{E}(X|Z) ] \\[18pt]
    \text{var}(X) & = & \text{E}[ \text{var}(X|Z) ] + \text{var}[ \text{E}(X|Z) ] \\[18pt]
    \text{cov}(X,Y) & = & \text{E}[ \text{cov}(X,Y|Z) ] + \text{cov}[ \text{E}(X|Z), \text{E}(Y|Z) ]
  \end{eqnarray*}

  \bigskip\bigskip

  \hfill {\lolit Everything is a mixture}

\end{frame}


\begin{frame}[c]{Another fave}
  \begin{eqnarray*}
    \text{cov}(X, aY+bZ) & = & a \, \text{cov}(X,Y) + b \, \text{cov}(X,Z) \\[18pt]
    \text{{\lolit thus} \, var}(X+Y) & = & \text{cov}(X+Y,X+Y) \\
    & = & \text{cov}(X+Y, X) + \text{cov}(X+Y, Y) \\
    & = & \text{cov}(X, X) + \text{cov}(X, Y) + \text{cov}(Y,X) + \text{cov}(Y,Y) \\
    & = & \text{var}(X) + \text{var}(Y) + 2 \, \text{cov}(X,Y)
    \end{eqnarray*}
\end{frame}


\begin{frame}[c]{}
  \figh{Figs/mixture_univariate.pdf}{1.0}
\end{frame}

\begin{frame}[c]{}
  \figh{Figs/mixture_bivariate.pdf}{1.0}
\end{frame}



\begin{frame}{Back to that SE}


  Let X$_i$ = number of 1 alleles in sib $i$.

  \bigskip

  We want $\text{var}(X_1 + X_2)$

  \bigskip

  \begin{eqnarray*}
  \text{And so really } {\hilit \text{cov}(X_1, X_2)} & = & \text{E}[\text{cov}(X_1, X_2 | \text{IBD})] + \text{cov}[E(X_1|\text{IBD}), E(X_2|\text{IBD})] \\[12pt]
   & = & {\hilit \text{E}[\text{cov}(X_1, X_2|\text{IBD})]} \\[12pt]
   & = & \sum_{k=0}^{2} \text{cov}(X_1, X_2|\text{IBD}=k) \, \text{Pr}(\text{IBD}=k)
   \end{eqnarray*}

\end{frame}



\begin{frame}{Also}

\bigskip

  \begin{eqnarray*}
    \text{cov}(X,Y) & = & \text{E}(XY) - \text{E}(X) \text{E}(Y) \\[24pt]
    \text{cov}(X,Y|Z) & = & \text{E}(XY|Z) - \text{E}(X|Z) \text{E}(Y|Z)
   \end{eqnarray*}

\end{frame}



\begin{frame}[c]{}

  \figh{Figs/broman2001_table4.png}{0.9}

\vspace{3mm}

\hfill \footnotesize {\lolit Broman (2001)
  \href{https://doi.org/10.1002/gepi.2}{doi:10.1002/gepi.2}}

\end{frame}


\begin{frame}{Lessons}

  \bbi
  \itemsep36pt
\item Omitting data is usually bad
\item Crudely ignoring correlations can be good
  \bi
\item[] You might even be able to figure out the SE
  \ei
  \ei

  \end{frame}



\begin{frame}{Method 3}

{\hilit Account for relationships in the estimate}

\bigskip \bigskip

Missing data = IBD status for a sib pair at a marker

\bigskip \bigskip

Use {\vhilit EM algorithm}:

\bigskip

\hspace*{15mm} \begin{minipage}{8in}
\bi
\item[\hilit E step:] estimate IBD status given allele frequencies

\item[\hilit M step:] estimate allele frequencies given IBD status
\ei
\end{minipage}

\end{frame}




\begin{frame}{Method 4}

{\hilit Make use of the multiple markers on a chromosome}


\end{frame}





\begin{frame}{Average relative efficiency}
\figh{Figs/ave_rel_eff.png}{0.7}

\vspace{10mm}

\hfill \footnotesize {\lolit Broman (2001)
  \href{https://doi.org/10.1002/gepi.2}{doi:10.1002/gepi.2}}

\end{frame}

\begin{frame}{Method 3}
\figh{Figs/rel_eff_method3.png}{0.6}

\vspace{18mm}

\hfill \footnotesize {\lolit Broman (2001)
  \href{https://doi.org/10.1002/gepi.2}{doi:10.1002/gepi.2}}

\end{frame}

\begin{frame}{Method 4}
\figh{Figs/rel_eff_method4.png}{0.75}

\bigskip

\hfill \footnotesize {\lolit Broman (2001)
  \href{https://doi.org/10.1002/gepi.2}{doi:10.1002/gepi.2}}

\end{frame}


\begin{frame}[c]{Summary}
  \figw{Figs/summary_table.png}{0.7}
\end{frame}


\begin{frame}{One last thing}

  \bbi
\item Turns out, I made a {\hilit mistake} in Method 3
  \bi
\item[]  Mary Sara McPeek (U Chicago) spotted it
  \ei
\item Fixed problem, re-ran simulations, and... \\[18pt]
  \onslide<2->{the correct MLE was {\hilit worse} than my mistaken estimate}
 \ei

\end{frame}

\end{document}
