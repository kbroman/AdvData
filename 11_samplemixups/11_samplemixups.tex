\documentclass[12pt,t]{beamer}
\usepackage{graphicx}
\setbeameroption{hide notes}
\setbeamertemplate{note page}[plain]
\usepackage{listings}

% header.tex: boring LaTeX/Beamer details + macros

% get rid of junk
\usetheme{default}
\beamertemplatenavigationsymbolsempty
\hypersetup{pdfpagemode=UseNone} % don't show bookmarks on initial view


% font
\usepackage{fontspec}
\setsansfont
  [ ExternalLocation = ../fonts/ ,
    UprightFont = *-regular ,
    BoldFont = *-bold ,
    ItalicFont = *-italic ,
    BoldItalicFont = *-bolditalic ]{texgyreheros}
\setbeamerfont{note page}{family*=pplx,size=\footnotesize} % Palatino for notes
% "TeX Gyre Heros can be used as a replacement for Helvetica"
% I've placed them in fonts/; alternatively you can install them
% permanently on your system as follows:
%     Download http://www.gust.org.pl/projects/e-foundry/tex-gyre/heros/qhv2.004otf.zip
%     In Unix, unzip it into ~/.fonts
%     In Mac, unzip it, double-click the .otf files, and install using "FontBook"

% named colors
\definecolor{offwhite}{RGB}{255,250,240}
\definecolor{gray}{RGB}{155,155,155}
\definecolor{purple}{RGB}{177,13,201}
\definecolor{green}{RGB}{46,204,64}

\definecolor{background}{RGB}{255,255,255}
\definecolor{foreground}{RGB}{24,24,24}
\definecolor{title}{RGB}{27,94,134}
\definecolor{subtitle}{RGB}{22,175,124}
\definecolor{hilit}{RGB}{122,0,128}
\definecolor{vhilit}{RGB}{255,0,128}
\definecolor{codehilit}{RGB}{255,0,128}
\definecolor{lolit}{RGB}{95,95,95}
\definecolor{myyellow}{rgb}{1,1,0.7}
\definecolor{nhilit}{RGB}{128,0,128}  % hilit color in notes
\definecolor{nvhilit}{RGB}{255,0,128} % vhilit for notes

\newcommand{\hilit}{\color{hilit}}
\newcommand{\vhilit}{\color{vhilit}}
\newcommand{\nhilit}{\color{nhilit}}
\newcommand{\nvhilit}{\color{nvhilit}}
\newcommand{\lolit}{\color{lolit}}

% use those colors
\setbeamercolor{titlelike}{fg=title}
\setbeamercolor{subtitle}{fg=subtitle}
\setbeamercolor{institute}{fg=lolit}
\setbeamercolor{normal text}{fg=foreground,bg=background}
\setbeamercolor{item}{fg=foreground} % color of bullets
\setbeamercolor{subitem}{fg=lolit}
\setbeamercolor{itemize/enumerate subbody}{fg=lolit}
\setbeamertemplate{itemize subitem}{{\textendash}}
\setbeamerfont{itemize/enumerate subbody}{size=\footnotesize}
\setbeamerfont{itemize/enumerate subitem}{size=\footnotesize}

% page number
\setbeamertemplate{footline}{%
    \raisebox{5pt}{\makebox[\paperwidth]{\hfill\makebox[20pt]{\lolit
          \scriptsize\insertframenumber}}}\hspace*{5pt}}

% add a bit of space at the top of the notes page
\addtobeamertemplate{note page}{\setlength{\parskip}{12pt}}

% default link color
\hypersetup{colorlinks, urlcolor={hilit}}

\lstset{language=bash,
        basicstyle=\ttfamily\scriptsize,
        frame=single,
        commentstyle=,
        backgroundcolor=\color{offwhite},
        showspaces=false,
        showstringspaces=false
        }


% a few macros
\newcommand{\bi}{\begin{itemize}}
\newcommand{\bbi}{\vspace{24pt} \begin{itemize} \itemsep8pt}
\newcommand{\ei}{\end{itemize}}
\newcommand{\be}{\begin{enumerate}}
\newcommand{\bbe}{\vspace{24pt} \begin{enumerate} \itemsep8pt}
\newcommand{\ee}{\end{enumerate}}
\newcommand{\ig}{\includegraphics}
\newcommand{\subt}[1]{{\footnotesize \color{subtitle} {#1}}}
\newcommand{\ttsm}{\tt \small}
\newcommand{\ttfn}{\tt \footnotesize}
\newcommand{\figh}[2]{\centerline{\includegraphics[height=#2\textheight]{#1}}}
\newcommand{\figw}[2]{\centerline{\includegraphics[width=#2\textwidth]{#1}}}



\title{Sample mix-ups in eQTL data}
\author{\href{https://kbroman.org}{Karl Broman}}
\institute{Biostatistics \& Medical Informatics, UW{\textendash}Madison}
\date{\href{https://kbroman.org}{\tt \scriptsize \color{foreground} kbroman.org}
\\[-4pt]
\href{https://github.com/kbroman}{\tt \scriptsize \color{foreground} github.com/kbroman}
\\[-4pt]
\href{https://twitter.com/kwbroman}{\tt \scriptsize \color{foreground} @kwbroman}
\\[-4pt]
{\scriptsize Course web: \href{https://kbroman.org/AdvData}{\tt kbroman.org/AdvData}}
}


\begin{document}

{
\setbeamertemplate{footline}{} % no page number here
\frame{
  \titlepage

\note{
  In this case study, I'll talk about a QTL mapping experiment where I
  discovered that like 18\% of the genotyped samples were mixed up.

  A weakness of QTL mapping has been the poor precision in estimated
  QTL location; it's very hard to identify the underlying genes.
  One strategy to deal with this
  weakness is to also measure intermediate phenotypes, such as the mRNA
  expression of all genes in a relevant tissue. We then seek to identify
  genetic loci (called expression quantitative trait loci, eQTL) that
  affect mRNA expression, and to find genes for which genotype is
  associated with mRNA expression and also the clinical trait.

  In a recent study with 500 intercross mice and gene expression
  microarray data on six tissues, I identified a large number of
  sample mix-ups in the genotype data and a smaller number of mix-ups
  in each set of microarrays. I'll describe how I found and corrected
  these problems. In a nutshell: the expression of some genes is so
  strongly associated with genotype that the expression data can
  effectively serve as a DNA fingerprint for establishing individuals'
  identities.
}
} }




\begin{frame}[c]{Data}

\hspace{0mm}

\figw{Figs/data_fig.png}{1.15}

\note{
  Again, I'm talking about QTL mapping. The data consist of genotypes
  at a set of markers across the genome, plus some quantitative
  phenotype for each mouse. The goal is to identify regions of the
  genome where the genotype is associated with the phenotype.
}
\end{frame}


\begin{frame}[c]{QTL mapping}

\vspace{5mm}
\only<1 | handout 0>{\figh{Figs/lodcurve_insulin.pdf}{0.85}}
\only<2>{\figh{Figs/lodcurve_insulin_with_effects.pdf}{0.85}}

\note{
  Our goal is to identify quantitative trait loci (QTL): regions of
  the genome for which genotype is associated with the phenotype.

  The basic analysis is to consider each locus, one at a time, split
  the mice into the three genotype groups, and perform analysis of
  variance.

  We then plot a test statistic that indicates the strength of the
  genotype-phenotype association.  For historical reasons, we
  calculate a LOD score as the test statistic: the log$_{10}$
  likelihood ratio comparing the hypothesis that there's a QTL at that
  position to the null hypothesis of no QTL anywhere.

  Large LOD scores indicate evidence for QTL and correspond to there
  being a difference in the phenotype average for the three genotype
  groups.
}
\end{frame}




\begin{frame}[c]{Attie project}


{\hilit
$\sim$500 B6 $\times$ BTBR intercross mice, all ob/ob }

\vspace{6pt}

\begin{itemize}
\itemsep12pt
\item Genotypes at 2057 SNPs (Affymetrix arrays)

\item Gene expression in six tissues (Agilent arrays)

  \begin{itemize}
    \item adipose
    \item gastrocnemius muscle
    \item hypothalamus
    \item pancreatic islets
    \item kidney
    \item liver
  \end{itemize}

\item Numerous clinical phenotypes
  \begin{itemize}
    \item[] (e.g., body weight, insulin and glucose levels)
  \end{itemize}

\end{itemize}

\note{
  When I first got to UW-Madison, I joined a collaboration that was
  carrying out a very large QTL mapping experiment that included about
  500 mice with dense genotype data and numerous clinical phenotypes,
  but also with gene expression data in six different tissues.

  I had mostly been in the back of the room, heckling. But a couple of
  years into the project, I agreed to write the first paper.
}

\end{frame}




\begin{frame}[c]{Sex and the X chr}
\figh{Figs/xchr_fig.pdf}{0.85}
\note{
  In getting ready to prepare that first paper, I decided to go back
  to the basics and really check that all of the data were in good
  order, starting from the raw genotype files.

  I noticed that there were a number of mice whose X chromosome
  genotype data did not match their sex. The way the cross was carried
  out, female F$_2$ mice will be homozygous BTBR or heterozygous, and
  male F$_2$ mice will be hemizygous (and so look like homyzogous).
  But there were a number of females who were homozygous B6 on the X,
  and a number males who were heterozygous. (Previously, these
  incompatible genotypes had just been omitted.)

  The number of mice with this problem ($\sim$16 out of 500) was not
  large, but it was more than I'd expected, and I sat and pondered how
  to figure out which was correct: sex or genotype.

  I realized that I could maybe use the gene expression data to help.
}
\end{frame}


\begin{frame}[c]{Strong eQTL}
\only<1|handout 0>{\figh{Figs/eqtl_lod_1.pdf}{0.85}}
\only<2>{\figh{Figs/eqtl_lod_2.pdf}{0.85}}
\note{
  In many cases the gene expression traits have very strong genetic
  effects. In particular, for many genes the expression level is
  strongly affected by genotype right at the location of the gene. For
  other genes, expression is strongly affected by genotype at some
  other location. A locus that effects gene expression is called an
  expression QTL or eQTL.
}
\end{frame}




\begin{frame}[c]{E vs G}
\only<1|handout 0>{\figh{Figs/gve1a.pdf}{0.85}}
\only<2>{\figh{Figs/gve1b.pdf}{0.85}}
\note{
  I looked at the gene expression versus genotype at one of these
  eQTL and saw a very strange pattern. There was a very strong
  association, but there were also a lot of mice whose gene expression
  seemed to not match their genotype.

  I mean, there are basically three kinds of mice, expression-wise:
  low, high, or very high. And the low-expression mice are mostly RR,
  while the very-high mice are mostly BB, with the high-expression
  mice being BR. Except there are a bunch of mice that seem to be in
  the wrong ball, expression-wise. And the 16 six-swapped mice include
  9 that are in the wrong ball.

  It's like the sex-swapped mice had been assigned to a random
  genotype. If the genotypes are in the proportions 1:2:1, then we'd
  expected 3/8 to be correct just by chance, which is very similar to
  the 7/16 we see in these data.

  And note that there are 43 mice that look to be in the wrong ball.
  If they are all being assigned genotypes at random, that would
  suggest that there are like 43 $\times$ (8/3) $\approx$ 115 problem
  mice.
}
\end{frame}


\begin{frame}[c]{kNN classifier}
\figh{Figs/gve1c.pdf}{0.85}
\note{
  But we can use the gene expression data to figure out what we
  {\hilit think} each mouse's genotype at this location really is. For
  example, we can create a k-nearest-neighbor classifier, for
  predicting genotype from gene expression.

  If we do this at many strong eQTL, we could potentially reconstruct
  the true genotypes for each mouse, from their expression data.
}
\end{frame}


\begin{frame}[c]{E vs G}
\only<1|handout 0>{\figh{Figs/gve3a.pdf}{0.85}}
\only<2>{\figh{Figs/gve3b.pdf}{0.85}}
\note{
   Many times there will be two different genes whose expression maps
   to a common location. We can look at their expression jointly. In
   many cases, the gene expression clusters are even more clear. And
   again the sex-swapped mice are seen in the wrong ball with
   frequency like 9/16.
}
\end{frame}


\begin{frame}[c]{E vs G}
\only<1|handout 0>{\figh{Figs/gve2a.pdf}{0.85}}
\only<2>{\figh{Figs/gve2b.pdf}{0.85}}
\note{
   Here's one more case. Again, the sex-swapped mice are in the wrong
   ball with frequency like 9/16. The particular mice that are correct
   or not different among the eQTL.
}
\end{frame}


\begin{frame}[c]{Basic scheme}
\only<1|handout 0>{\figh{Figs/gve_scheme_1.pdf}{0.85}}
\only<2|handout 0>{\figh{Figs/gve_scheme_2.pdf}{0.85}}
\only<3|handout 0>{\figh{Figs/gve_scheme_3.pdf}{0.85}}
\only<4>{\figh{Figs/gve_scheme_4.pdf}{0.85}}
\note{
   So this leads to our basic scheme for identifying (and correcting)
   the sample mix-ups.

   We first identify a set of expression traits with very strong eQTL.
   We use the expression and corresponding eQTL genotypes to form
   classifiers for predicting eQTL genotype from gene expression. This
   gives us a matrix of inferred eQTL genotypes.

   We then compare the inferred eQTL genotypes to the observed eQTL
   genotypes. If a sample's observed eQTL genotypes don't match its
   inferred eQTL genotype, we conclude that the labels for one or the
   other are incorrect. And we might be able to find another row in
   the inferred eQTL genotypes that matches its observed genotypes.
}
\end{frame}

\begin{frame}[c]{Prop'n mismatches}
\figh{Figs/distmatall.png}{0.85}
\note{
  For each pair of samples, one DNA (genotype) sample and one RNA
  (gene expression) sample, we get a measure of distance as the
  proportion of mismatches between the observed eQTL genotypes and the
  inferred eQTL genotypes.

  Here's a picture of this distance matrix. It should be blue along
  the diagonal and red everywhere else.
}
\end{frame}

\begin{frame}[c]{Prop'n mismatches}
\figh{Figs/distmat001.pdf}{0.85}
\note{
  And if we look at the first 100 samples, that's exactly what we see:
  the samples are close to themselves and not to anyone else.
}
\end{frame}

\begin{frame}[c]{Prop'n mismatches}
\figh{Figs/distmat201.pdf}{0.85}
\note{
  But if we look at the middle 100 samples, we find a whole bunch of
  off-by-one and off-by-two errors. The samples are quite different
  from the corresponding one, but their close to the one next to it or
  two over.
}
\end{frame}


\begin{frame}[c]{Prop'n mismatches}
\figh{Figs/gve_dist.pdf}{0.85}
\note{
  If we look at histograms of the diagonal of the distance matrix (top
  panel) and the off-diagonal values (lower panel), we find that most
  samples are correct, but there are a bunch of values on the diagonal
  that really are non-matching, and a bunch of values off the diagonal
  that are indicative of matches.
}
\end{frame}


\begin{frame}[c]{Decisions}
\figh{Figs/gve_dist_byrow_left.pdf}{0.85}
\note{
   If for each row of the distance matrix we take the value on the
   diagonal (the self-self distance) and plot it against the minimum
   value in that row, we find a bunch of samples that look correct (in
   blue in the lower-left corner), as they are closest to themselves
   and that distance is small.

   There are a number of samples that are wrong but ``fixable'' (in
   green), as they are not close to themselves but they are close to
   some other sample.

   Then there are samples ``not found'' (in red) that are not close to
   anything. There were actually 550 total mice (good to have backups
   in case one dies), but only about 500 had gene expression data in
   any one tissue, and some of the DNA samples were apparently lost.

   We don't know, from these results, whether it is the DNA samples
   that were mislabelled or the mRNA samples, but because we have six
   sets of mRNA samples, for six different, we can compare the DNA to
   each of the mRNA samples and in doing so it is clear that it's the
   DNA that was wrong.
}
\end{frame}


\begin{frame}[c]{Genotype mix-ups}
\figh{Figs/plate_errors.pdf}{0.85}
\note{
  Even more incriminating, though, is the information about the
  locations of the DNA samples. DNA samples were arrayed in a set of
  six 8$\times$12 plates. In this figure, the black dots indicate the
  correct DNA sample was placed in the correct well, while the arrows
  point from where a DNA sample should have been to where it actually
  ended up.

  Two of the plates look fine, while half of each of two plates are
  entirely messed up.
}
\end{frame}


\begin{frame}[c]{Plate 1631}
\figh{Figs/plate_errors_1631.png}{0.65}
\note{
  Plate 1631 is a good example. Again, black dots indicate that the
  correct DNA was placed in the correct well.

  The little orange and purple arrow
  heads indicate that sample in well D7 is of unknown origin, and the
  sample that should have been there was lost.

  The pink circle around D2 indicates that that sample was duplicated:
  it was placed in the correct well (the black dot), but it was also
  placed in well B3. The sample that was supposed to be in B3 was
  placed in B4, the sample that was supposed to be in B4 was in E3,
  and the sample that was supposed to be in E3 was lost.

  (The purple arrow head for D7 means that the DNA was lost but that
  there is expression data for that sample, while the green arrow head
  for E3 means that the DNA was lost but there is no expression data
  for that sample.)
}
\end{frame}

\begin{frame}[c]{Plates 1632 and 1630}
\figh{Figs/plate_errors_1632_n_1630.png}{0.85}
\note{
  Plates 1632 and 1630 are where most of the problems are. There are
  some long-range swaps and other misplacements of samples, but most
  of the problems are due to a series of off-by-one and off-by two
  errors. Note that the red X's indicate DNAs that were omitted due as
  being of bad quality (possibly mixtures).
}
\end{frame}

\begin{frame}[c]{Plate 1630}
\figh{Figs/plate_errors_1630.png}{0.65}
\note{
  Consider plate 1630. The sample found in A1 really belonged back on
  plate 1632. The sample that was supposed to be in A1 was found in
  B1. The sample that was supposed to be in B1 was duplicated, in both
  C1 and D1. So then we're off by two for a while: the sample that
  should have been in C1 was in E1, and the sample that should have
  been in D1 was in F1, etc. At well F5 we're back to being off by one
  again, and then a DNA was lost at H7 and we're back to being
  correct.
}
\end{frame}


\begin{frame}[c]{E vs E}
\only<1|handout 0>{\figh{Figs/eve_1.pdf}{0.85}}
\only<2|handout 0>{\figh{Figs/eve_2.pdf}{0.85}}
\only<3|handout 0>{\figh{Figs/eve_3.jpg}{0.85}}
\only<4|handout 0>{\figh{Figs/eve_3b.pdf}{0.85}}
\only<5|handout 0>{\figh{Figs/eve_3c.pdf}{0.85}}
\only<6|handout 0>{\figh{Figs/eve_3d.pdf}{0.85}}
\only<7|handout 0>{\figh{Figs/eve_4.pdf}{0.85}}
\only<8|handout 0>{\figh{Figs/eve_5.pdf}{0.85}}
\only<9|handout 0>{\figh{Figs/eve_6.pdf}{0.85}}
\only<10|handout 0>{\figh{Figs/eve_7.pdf}{0.85}}
\only<11>{\figh{Figs/eve_8.pdf}{0.85}}
\note{
   We can use the same trick to look for mix-ups among the gene
   expression data sets.

   The basic scheme is to first identify a subset of expression traits
   that are highly correlated between two tissues.

   Then look at the correlation between samples, using just that
   subset of expression traits.

   When a sample is correctly labeled in both tissues, the expression
   values should be correlated. If not, we may find another sample in
   one tissue that is correlated, to indicate the true label.

   Again, we make use of the multiple tissues to figure out the truth.
   If we had just two tissues we could see that they were mixed up but
   not which was the correct label.
}
\end{frame}


\begin{frame}[c]{E vs E}
\only<1>{\figh{Figs/eve_9.pdf}{0.85}}
\only<2|handout 0>{\figh{Figs/eve_10.pdf}{0.85}}
\only<3|handout 0>{\figh{Figs/eve_11.pdf}{0.85}}
\note{
  Here's an example of a mislabelling: Mouse3598 liver looks more like
  Mouse3599 islet, and Mouse3598 islet looks more like Mouse3599
  liver. We use expression in the other four tissues to decide which
  is right.
}
\end{frame}

\begin{frame}[c]{Expression mix-ups}
\figh{Figs/expr_swaps.pdf}{0.85}
\note{
  Here are the set of mix-ups I found in the expression data. The
  arrows point from the correct label to how it appeared.

  Each tissue had some mistakes; hypothalmous was the worst. The pink
  circles indicate a sample duplicate. So, for example, in islet
  sample 3295 was correctly labeled but also appeared in duplicate
  with one sample labelled as 3296. The 3296 islet sample was lost.

  Adipose had a 3-way swap. 3187 was labelled as 3200 which was
  labelled as 3188 which was labelled as 3187. Note that most of the
  problems concern sample numbers that are close (but not necessarily
  immediately adjacent) in number.
}
\end{frame}

\begin{frame}[c]{Insulin QTL}
\figh{Figs/insulin_lod.pdf}{0.85}
\note{
  This shows the genome scan results for insulin (one of the more
  important clinical traits) before and after the 18\% sample mix-ups
  were corrected. With the mix-ups in the data, we did see four QTL
  (chr 2, 7, 12, and 19), but after correcting the mix-ups, the
  strength of evidence for the chr 2 locus increased considerably, and
  we see additional significant QTL on chromosomes 5, 6, and 9.

  We had been studying these data for a couple of years without
  noticing any problems. And it is sort of remarkable that with
  $\sim$20\% of the genotypes mis-labelled, you can still get good
  evidence for QTL. The evidence of course improves greatly when you
  correct the mix-ups, but it's not as drammatic as you might have
  expected.
}
\end{frame}


\begin{frame}[c]{Strong eQTL}
\figh{Figs/eqtl_lod_3.pdf}{0.85}
\note{
  The two strong eQTL that I had shown before also show dramatic
  increases in LOD score after correcting the sample mix-ups, for
  example the gene on chr 1 had LOD score like 150 and now has LOD
  score over 450.
}
\end{frame}


\begin{frame}[c]{Summary}


\small
\begin{itemize}
\itemsep3pt

\item Sample mix-ups happen


\item With eQTL data, we can both identify and {\hilit correct} mix-ups

\item There is great value in having expression on multiple tissues

\item The general idea here has wide application for high-throughput data

\item \href{https://www.ncbi.nlm.nih.gov/pubmed/26290572}{Broman et
  al. (2015) G3 5:2177-2186} \\
\href{http://doi.org/10.1534/g3.115.019778}{doi: 10.1534/g3.115.019778}

\item Related work:

\begin{itemize}
\item Westra et al. (2011) Bioinformatics 27:2104--2111
\item Schadt et al. (2012) Nat Genet 44:603--608
\item Ekstr{\o}m and Feenstra (2012) Stat Appl Genet Mol Biol
  3:Article 13
\item Lynch et al. (2012) PLoS ONE 7:e41815
\end{itemize}

\end{itemize}
\note{
  In summary, sample mix-ups happen. With eQTL data you can both
  identify and correct mix-ups. There was great value in having
  expression data from multiple tissues, in identifying the source of
  the problems.

  The general idea here has wide application for high-throughput data,
  generally. If you have mutiple rectangles of data whose rows are
  supposed to correspond, you should check to see if they do correspond.
  The strategy we used for aligning two expression datasets could
  work with little change in much broader contexts.

  The article describing this work was published in 2015. A number of
  others happened upon similar problems and similar solutions at about
  the same time that I did, but published much sooner (2011 and 2012).
  They're all interesting reads.
}
\end{frame}



\begin{frame}[c]{Lessons}

\small

\begin{itemize}
\itemsep8pt

\item Don't fully trust anyone
\begin{itemize}
\item Including yourself
\end{itemize}

\item Make lots of plots
\begin{itemize}
\item Don't rely on summary statistics, like LOD scores
\item Look at responses on the original scale
\end{itemize}

\item Follow up all aberrations

\item Take your time with data cleaning
\begin{itemize}
\item A month, two months, a year?
\end{itemize}

\item If you have big rectangles whose rows correspond, \\
  check that they {\hilit actually} correspond

\end{itemize}
\note{
  There are a number of important lessons to draw from this work.
  First, don't fully trush anyone, even yourself. That seems overly
  cynical, but really: if you care about the results, take the time to
  double-check that previous data cleaning efforts (perhaps by you six
  months ago) didn't skip over some critical problem.

  Also, make lots of plots. The long delay in us identifying problems
  was partly due to the fact that we had mostly focused on summary
  plots like the LOD curves. You can't really see the problems until
  you look at the phenotype/genotype relationships. Also, we had
  transformed all of the expression phenotypes by taking ranks and
  then converting them to normal quantiles. This was great for
  eliminating the effects of outliers, but it made it hard to identify
  problems.

  [go to the next slide for an illustration of this point]

}
\end{frame}


\begin{frame}[c]{E vs G}

\figw{Figs/gve1a_nqrank.pdf}{1.0}

\note{

  For example, the panel on the left here is the plot I showed before,
  for expression vs eQTL genotype. This is the one that had indicated
  to me that there was a problem.

  But we hadn't been looking at the plot on the left, with the
  untransformed expression values, but rather the plot on the right,
  in which the expression values were ranked and then transformed to
  normal quantiles.

  The odd pattern on the left is made less odd by the transformation.
  The plot on the right is a little weird, but it looks more like
  three overlapping normal distributions.

  Transformations are great, but for diagnostic purposes, and to
  assess the effects of covariates like QTL, it is best to return to
  the original scale of measurement, as transformations can obscure
  important features.
}
\end{frame}



\begin{frame}[c]{Lessons}

\small

\begin{itemize}
\itemsep8pt

\item Don't fully trust anyone
\begin{itemize}
\item Including yourself
\end{itemize}

\item Make lots of plots
\begin{itemize}
\item Don't rely on summary statistics, like LOD scores
\item Look at responses on the original scale
\end{itemize}

\item Follow up all aberrations

\item Take your time with data cleaning
\begin{itemize}
\item A month, two months, a year?
\end{itemize}

\item If you have big rectangles whose rows correspond, \\
  check that they {\hilit actually} correspond

\end{itemize}
\note{
  Returning to our lessons from this case study, we need to emphasize
  again to follow up all aberrations. I came to the realization of
  these sample mix-ups on the basis of just 16/500 mice whose sex
  didn't match their X chromosome genotypes. We might have just
  omitted the samples and moved on, but it was only by really puzzling
  through the cause of the problem that I was able to identify the
  much larger issue.

  Related to that: take your time in data cleaning. If you spend \$5
  million dollars gathering data, isn't it reasonable to spend a
  month, two months, even a year on the data cleaning? Sometimes it
  seems like my collaborators think that the more money you spend, the
  faster you should get results. But if you really care about getting
  the right answers, you should be willing to spend time verifying
  that the data are in good order.

  Finally, again, if you have mutiple rectangles of data whose rows
  are supposed to correspond, you should check to see if they actually
  do correspond.
}
\end{frame}




\begin{frame}[c]{Decisions}
\figh{Figs/gve_dist_byrow.pdf}{0.85}
\note{
  This is an extra slide to show that the evidence for the mixed-up
  samples' true identities is strong. On the left is the plot we saw
  before: the self-self distance vs the minimum distance.

  On the right is the 2nd biggest distance vs the minimum distance.
  That the blue and green points are well away from the diagonal
  indicates that we can tell which sample is really the smallest;
  there's not some nearby next-most-similar to confuse things.
}
\end{frame}

\end{document}
