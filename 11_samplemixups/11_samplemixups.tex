\documentclass[12pt,t]{beamer}
\usepackage{graphicx}
\setbeameroption{hide notes}
\setbeamertemplate{note page}[plain]
\usepackage{listings}

% header.tex: boring LaTeX/Beamer details + macros

% get rid of junk
\usetheme{default}
\beamertemplatenavigationsymbolsempty
\hypersetup{pdfpagemode=UseNone} % don't show bookmarks on initial view


% font
\usepackage{fontspec}
\setsansfont
  [ ExternalLocation = ../fonts/ ,
    UprightFont = *-regular ,
    BoldFont = *-bold ,
    ItalicFont = *-italic ,
    BoldItalicFont = *-bolditalic ]{texgyreheros}
\setbeamerfont{note page}{family*=pplx,size=\footnotesize} % Palatino for notes
% "TeX Gyre Heros can be used as a replacement for Helvetica"
% I've placed them in fonts/; alternatively you can install them
% permanently on your system as follows:
%     Download http://www.gust.org.pl/projects/e-foundry/tex-gyre/heros/qhv2.004otf.zip
%     In Unix, unzip it into ~/.fonts
%     In Mac, unzip it, double-click the .otf files, and install using "FontBook"

% named colors
\definecolor{offwhite}{RGB}{255,250,240}
\definecolor{gray}{RGB}{155,155,155}
\definecolor{purple}{RGB}{177,13,201}
\definecolor{green}{RGB}{46,204,64}

\definecolor{background}{RGB}{255,255,255}
\definecolor{foreground}{RGB}{24,24,24}
\definecolor{title}{RGB}{27,94,134}
\definecolor{subtitle}{RGB}{22,175,124}
\definecolor{hilit}{RGB}{122,0,128}
\definecolor{vhilit}{RGB}{255,0,128}
\definecolor{codehilit}{RGB}{255,0,128}
\definecolor{lolit}{RGB}{95,95,95}
\definecolor{myyellow}{rgb}{1,1,0.7}
\definecolor{nhilit}{RGB}{128,0,128}  % hilit color in notes
\definecolor{nvhilit}{RGB}{255,0,128} % vhilit for notes

\newcommand{\hilit}{\color{hilit}}
\newcommand{\vhilit}{\color{vhilit}}
\newcommand{\nhilit}{\color{nhilit}}
\newcommand{\nvhilit}{\color{nvhilit}}
\newcommand{\lolit}{\color{lolit}}

% use those colors
\setbeamercolor{titlelike}{fg=title}
\setbeamercolor{subtitle}{fg=subtitle}
\setbeamercolor{institute}{fg=lolit}
\setbeamercolor{normal text}{fg=foreground,bg=background}
\setbeamercolor{item}{fg=foreground} % color of bullets
\setbeamercolor{subitem}{fg=lolit}
\setbeamercolor{itemize/enumerate subbody}{fg=lolit}
\setbeamertemplate{itemize subitem}{{\textendash}}
\setbeamerfont{itemize/enumerate subbody}{size=\footnotesize}
\setbeamerfont{itemize/enumerate subitem}{size=\footnotesize}

% page number
\setbeamertemplate{footline}{%
    \raisebox{5pt}{\makebox[\paperwidth]{\hfill\makebox[20pt]{\lolit
          \scriptsize\insertframenumber}}}\hspace*{5pt}}

% add a bit of space at the top of the notes page
\addtobeamertemplate{note page}{\setlength{\parskip}{12pt}}

% default link color
\hypersetup{colorlinks, urlcolor={hilit}}

\lstset{language=bash,
        basicstyle=\ttfamily\scriptsize,
        frame=single,
        commentstyle=,
        backgroundcolor=\color{offwhite},
        showspaces=false,
        showstringspaces=false
        }


% a few macros
\newcommand{\bi}{\begin{itemize}}
\newcommand{\bbi}{\vspace{24pt} \begin{itemize} \itemsep8pt}
\newcommand{\ei}{\end{itemize}}
\newcommand{\be}{\begin{enumerate}}
\newcommand{\bbe}{\vspace{24pt} \begin{enumerate} \itemsep8pt}
\newcommand{\ee}{\end{enumerate}}
\newcommand{\ig}{\includegraphics}
\newcommand{\subt}[1]{{\footnotesize \color{subtitle} {#1}}}
\newcommand{\ttsm}{\tt \small}
\newcommand{\ttfn}{\tt \footnotesize}
\newcommand{\figh}[2]{\centerline{\includegraphics[height=#2\textheight]{#1}}}
\newcommand{\figw}[2]{\centerline{\includegraphics[width=#2\textwidth]{#1}}}



\title{Sample mix-ups in eQTL data}
\author{\href{https://kbroman.org}{Karl Broman}}
\institute{Biostatistics \& Medical Informatics, UW{\textendash}Madison}
\date{\href{https://kbroman.org}{\tt \scriptsize \color{foreground} kbroman.org}
\\[-4pt]
\href{https://github.com/kbroman}{\tt \scriptsize \color{foreground} github.com/kbroman}
\\[-4pt]
\href{https://twitter.com/kwbroman}{\tt \scriptsize \color{foreground} @kwbroman}
\\[-4pt]
{\scriptsize Course web: \href{https://kbroman.org/AdvData}{\tt kbroman.org/AdvData}}
}


\begin{document}

{
\setbeamertemplate{footline}{} % no page number here
\frame{
  \titlepage

\note{
}
} }




\begin{frame}[c]{Data}

\hspace{0mm}

\figw{Figs/data_fig.png}{1.15}

\note{
  The data consist of genotypes at a set of markers across the genome,
  plus some quantitative phenotype for each mouse. The goal is to
  identify regions of the genome where the genotype is associated with
  the phenotype.
}
\end{frame}


\begin{frame}[c]{QTL mapping}

\vspace{5mm}
\only<1 | handout 0>{\figh{Figs/lodcurve_insulin.pdf}{0.85}}
\only<2>{\figh{Figs/lodcurve_insulin_with_effects.pdf}{0.85}}

\note{
  Our goal is to identify quantitative trait loci (QTL): regions of
  the genome for which genotype is associated with the phenotype.

  The basic analysis is to consider each locus, one at a time, split
  the mice into the three genotype groups, and perform analysis of
  variance.

  We then plot a test statistic that indicates the strength of the
  genotype-phenotype association.  For historical reasons, we
  calculate a LOD score as the test statistic: the log$_{10}$
  likelihood ratio comparing the hypothesis that there's a QTL at that
  position to the null hypothesis of no QTL anywhere.

  Large LOD scores indicate evidence for QTL and correspond to there
  being a difference in the phenotype average for the three genotype
  groups.
}
\end{frame}




\begin{frame}[c]{Attie project}


{\hilit
$\sim$500 B6 $\times$ BTBR intercross mice, all ob/ob }

\vspace{6pt}

\begin{itemize}
\itemsep12pt
\item Genotypes at 2057 SNPs (Affymetrix arrays)

\item Gene expression in six tissues (Agilent arrays)

  \begin{itemize}
    \item adipose
    \item gastrocnemius muscle
    \item hypothalamus
    \item pancreatic islets
    \item kidney
    \item liver
  \end{itemize}

\item Numerous clinical phenotypes
  \begin{itemize}
    \item[] (e.g., body weight, insulin and glucose levels)
  \end{itemize}

\end{itemize}

\note{}

\end{frame}




\begin{frame}[c]{Sex and the X chr}
\figh{Figs/xchr_fig.pdf}{0.85}
\note{}
\end{frame}


\begin{frame}[c]{Genotype mix-ups}
\figw{Figs/plate_errors.pdf}{1.1}
\note{}
\end{frame}


\begin{frame}[c]{Sex and the X chr}
\figh{Figs/xchr_fig.pdf}{0.85}
\note{}
\end{frame}


\begin{frame}[c]{Strong eQTL}
\only<1|handout 0>{\figh{Figs/eqtl_lod_1.pdf}{0.85}}
\only<2>{\figh{Figs/eqtl_lod_2.pdf}{0.85}}
\note{}
\end{frame}




\begin{frame}[c]{E vs G}
\only<1|handout 0>{\figh{Figs/gve1a.pdf}{0.85}}
\only<2>{\figh{Figs/gve1b.pdf}{0.85}}
\note{}
\end{frame}


\begin{frame}[c]{kNN classifier}
\figh{Figs/gve1c.pdf}{0.85}
\note{}
\end{frame}


\begin{frame}[c]{E vs G}
\only<1|handout 0>{\figh{Figs/gve3a.pdf}{0.85}}
\only<2>{\figh{Figs/gve3b.pdf}{0.85}}
\note{}
\end{frame}


\begin{frame}[c]{E vs G}
\only<1|handout 0>{\figh{Figs/gve2a.pdf}{0.85}}
\only<2>{\figh{Figs/gve2b.pdf}{0.85}}
\note{}
\end{frame}


\begin{frame}[c]{Basic scheme}
\only<1|handout 0>{\figh{Figs/gve_scheme_1.pdf}{0.85}}
\only<2|handout 0>{\figh{Figs/gve_scheme_2.pdf}{0.85}}
\only<3|handout 0>{\figh{Figs/gve_scheme_3.pdf}{0.85}}
\only<4>{\figh{Figs/gve_scheme_4.pdf}{0.85}}
\note{}
\end{frame}

\begin{frame}[c]{Prop'n mismatches}
\figh{Figs/distmatall.png}{0.85}
\note{}
\end{frame}

\begin{frame}[c]{Prop'n mismatches}
\figh{Figs/distmat001.pdf}{0.85}
\note{}
\end{frame}

\begin{frame}[c]{Prop'n mismatches}
\figh{Figs/distmat201.pdf}{0.85}
\note{}
\end{frame}


\begin{frame}[c]{Prop'n mismatches}
\figh{Figs/gve_dist.pdf}{0.85}
\note{}
\end{frame}


\begin{frame}[c]{Decisions}
\figh{Figs/gve_dist_byrow_left.pdf}{0.85}
\note{}
\end{frame}


\begin{frame}[c]{Genotype mix-ups}
\figh{Figs/plate_errors.pdf}{0.85}
\note{}
\end{frame}


\begin{frame}[c]{Plate 1631}
\figh{Figs/plate_errors_1631.png}{0.65}
\note{}
\end{frame}

\begin{frame}[c]{Plates 1632 and 1630}
\figh{Figs/plate_errors_1632_n_1630.png}{0.85}
\note{}
\end{frame}

\begin{frame}[c]{Plate 1630}
\figh{Figs/plate_errors_1630.png}{0.65}
\note{}
\end{frame}


\begin{frame}[c]{E vs E}
\figh{Figs/eve_1.pdf}{0.85}
\note{}
\end{frame}


\begin{frame}[c]{E vs E}
\figh{Figs/eve_2.pdf}{0.85}
\note{}
\end{frame}


\begin{frame}[c]{E vs E}
\figh{Figs/eve_3.jpg}{0.85}
\note{}
\end{frame}


\begin{frame}[c]{E vs E}
\figh{Figs/eve_3b.pdf}{0.85}
\note{}
\end{frame}


\begin{frame}[c]{E vs E}
\figh{Figs/eve_3c.pdf}{0.85}
\note{}
\end{frame}


\begin{frame}[c]{E vs E}
\figh{Figs/eve_3d.pdf}{0.85}
\note{}
\end{frame}


\begin{frame}[c]{E vs E}
\figh{Figs/eve_4.pdf}{0.85}
\note{}
\end{frame}


\begin{frame}[c]{E vs E}
\figh{Figs/eve_5.pdf}{0.85}
\note{}
\end{frame}


\begin{frame}[c]{E vs E}
\figh{Figs/eve_6.pdf}{0.85}
\note{}
\end{frame}


\begin{frame}[c]{E vs E}
\figh{Figs/eve_7.pdf}{0.85}
\note{}
\end{frame}


\begin{frame}[c]{E vs E}
\figh{Figs/eve_8.pdf}{0.85}
\note{}
\end{frame}


\begin{frame}[c]{E vs E}
\figh{Figs/eve_9.pdf}{0.85}
\note{}
\end{frame}


\begin{frame}[c]{E vs E}
\figh{Figs/eve_10.pdf}{0.85}
\note{}
\end{frame}


\begin{frame}[c]{E vs E}
\figh{Figs/eve_11.pdf}{0.85}
\note{}
\end{frame}

\begin{frame}[c]{Expression mix-ups}
\figh{Figs/expr_swaps.pdf}{0.85}
\note{}
\end{frame}

\begin{frame}[c]{Insulin QTL}
\figh{Figs/insulin_lod.pdf}{0.85}
\note{}
\end{frame}


\begin{frame}[c]{Strong eQTL}
\figh{Figs/eqtl_lod_3.pdf}{0.85}
\note{}
\end{frame}


\begin{frame}[c]{Summary}


\small
\begin{itemize}
\itemsep3pt

\item Sample mix-ups happen


\item With eQTL data, we can both identify and {\hilit correct} mix-ups

\item There is great value in having expression on multiple tissues

\item The general idea here has wide application for high-throughput data

\item \href{https://www.ncbi.nlm.nih.gov/pubmed/26290572}{Broman et
  al. (2015) G3 5:2177-2186} \\
\href{http://doi.org/10.1534/g3.115.019778}{doi: 10.1534/g3.115.019778}

\item Related work:

\begin{itemize}
\item Westra et al. (2011) Bioinformatics 27:2104--2111
\item Schadt et al. (2012) Nat Genet 44:603--608
\item Ekstr{\o}m and Feenstra (2012) Stat Appl Genet Mol Biol
  3:Article 13
\item Lynch et al. (2012) PLoS ONE 7:e41815
\end{itemize}

\end{itemize}
\note{}
\end{frame}



\begin{frame}[c]{Lessons}

\small

\begin{itemize}
\itemsep8pt

\item Don't fully trust anyone
\begin{itemize}
\item Including yourself
\end{itemize}

\item Make lots of plots
\begin{itemize}
\item Don't rely on summary statistics, like LOD scores
\item Look at responses on the original scale
\end{itemize}

\item Follow up all aberrations

\item Take your time with data cleaning
\begin{itemize}
\item A month, two months, a year?
\end{itemize}

\item If you have big rectangles whose rows correspond, \\
  check that they {\hilit actually} correspond

\end{itemize}
\note{}
\end{frame}


\begin{frame}[c]{E vs G}

\figw{Figs/gve1a_nqrank.pdf}{1.0}

\note{}
\end{frame}



\begin{frame}[c]{Lessons}

\small

\begin{itemize}
\itemsep8pt

\item Don't fully trust anyone
\begin{itemize}
\item Including yourself
\end{itemize}

\item Make lots of plots
\begin{itemize}
\item Don't rely on summary statistics, like LOD scores
\item Look at responses on the original scale
\end{itemize}

\item Follow up all aberrations

\item Take your time with data cleaning
\begin{itemize}
\item A month, two months, a year?
\end{itemize}

\item If you have big rectangles whose rows correspond, \\
  check that they {\hilit actually} correspond

\end{itemize}
\note{}
\end{frame}




\begin{frame}[c]{Decisions}
\figh{Figs/gve_dist_byrow.pdf}{0.85}
\note{}
\end{frame}

\end{document}
