\documentclass[aspectratio=169,12pt,t]{beamer}
\usepackage{graphicx}
\setbeameroption{hide notes}
\setbeamertemplate{note page}[plain]
\usepackage{listings}

% header.tex: boring LaTeX/Beamer details + macros

% get rid of junk
\usetheme{default}
\beamertemplatenavigationsymbolsempty
\hypersetup{pdfpagemode=UseNone} % don't show bookmarks on initial view


% font
\usepackage{fontspec}
\setsansfont
  [ ExternalLocation = ../fonts/ ,
    UprightFont = *-regular ,
    BoldFont = *-bold ,
    ItalicFont = *-italic ,
    BoldItalicFont = *-bolditalic ]{texgyreheros}
\setbeamerfont{note page}{family*=pplx,size=\footnotesize} % Palatino for notes
% "TeX Gyre Heros can be used as a replacement for Helvetica"
% I've placed them in fonts/; alternatively you can install them
% permanently on your system as follows:
%     Download http://www.gust.org.pl/projects/e-foundry/tex-gyre/heros/qhv2.004otf.zip
%     In Unix, unzip it into ~/.fonts
%     In Mac, unzip it, double-click the .otf files, and install using "FontBook"

% named colors
\definecolor{offwhite}{RGB}{255,250,240}
\definecolor{gray}{RGB}{155,155,155}
\definecolor{purple}{RGB}{177,13,201}
\definecolor{green}{RGB}{46,204,64}

\definecolor{background}{RGB}{255,255,255}
\definecolor{foreground}{RGB}{24,24,24}
\definecolor{title}{RGB}{27,94,134}
\definecolor{subtitle}{RGB}{22,175,124}
\definecolor{hilit}{RGB}{122,0,128}
\definecolor{vhilit}{RGB}{255,0,128}
\definecolor{codehilit}{RGB}{255,0,128}
\definecolor{lolit}{RGB}{95,95,95}
\definecolor{myyellow}{rgb}{1,1,0.7}
\definecolor{nhilit}{RGB}{128,0,128}  % hilit color in notes
\definecolor{nvhilit}{RGB}{255,0,128} % vhilit for notes

\newcommand{\hilit}{\color{hilit}}
\newcommand{\vhilit}{\color{vhilit}}
\newcommand{\nhilit}{\color{nhilit}}
\newcommand{\nvhilit}{\color{nvhilit}}
\newcommand{\lolit}{\color{lolit}}

% use those colors
\setbeamercolor{titlelike}{fg=title}
\setbeamercolor{subtitle}{fg=subtitle}
\setbeamercolor{institute}{fg=lolit}
\setbeamercolor{normal text}{fg=foreground,bg=background}
\setbeamercolor{item}{fg=foreground} % color of bullets
\setbeamercolor{subitem}{fg=lolit}
\setbeamercolor{itemize/enumerate subbody}{fg=lolit}
\setbeamertemplate{itemize subitem}{{\textendash}}
\setbeamerfont{itemize/enumerate subbody}{size=\footnotesize}
\setbeamerfont{itemize/enumerate subitem}{size=\footnotesize}

% page number
\setbeamertemplate{footline}{%
    \raisebox{5pt}{\makebox[\paperwidth]{\hfill\makebox[20pt]{\lolit
          \scriptsize\insertframenumber}}}\hspace*{5pt}}

% add a bit of space at the top of the notes page
\addtobeamertemplate{note page}{\setlength{\parskip}{12pt}}

% default link color
\hypersetup{colorlinks, urlcolor={hilit}}

\lstset{language=bash,
        basicstyle=\ttfamily\scriptsize,
        frame=single,
        commentstyle=,
        backgroundcolor=\color{offwhite},
        showspaces=false,
        showstringspaces=false
        }


% a few macros
\newcommand{\bi}{\begin{itemize}}
\newcommand{\bbi}{\vspace{24pt} \begin{itemize} \itemsep8pt}
\newcommand{\ei}{\end{itemize}}
\newcommand{\be}{\begin{enumerate}}
\newcommand{\bbe}{\vspace{24pt} \begin{enumerate} \itemsep8pt}
\newcommand{\ee}{\end{enumerate}}
\newcommand{\ig}{\includegraphics}
\newcommand{\subt}[1]{{\footnotesize \color{subtitle} {#1}}}
\newcommand{\ttsm}{\tt \small}
\newcommand{\ttfn}{\tt \footnotesize}
\newcommand{\figh}[2]{\centerline{\includegraphics[height=#2\textheight]{#1}}}
\newcommand{\figw}[2]{\centerline{\includegraphics[width=#2\textwidth]{#1}}}


%%%%%%%%%%%%%%%%%%%%%%%%%%%%%%%%%%%%%%%%%%%%%%%%%%%%%%%%%%%%%%%%%%%%%%
% end of header
%%%%%%%%%%%%%%%%%%%%%%%%%%%%%%%%%%%%%%%%%%%%%%%%%%%%%%%%%%%%%%%%%%%%%%

\title{Building regression models}
\subtitle{Mapping multiple QTL}
\author{\href{https://kbroman.org}{Karl Broman}}
\institute{Biostatistics \& Medical Informatics, UW{\textendash}Madison}
\date{\href{https://kbroman.org}{\tt \scriptsize \color{foreground} kbroman.org}
\\[-4pt]
\href{https://github.com/kbroman}{\tt \scriptsize \color{foreground} github.com/kbroman}
\\[-4pt]
\href{https://twitter.com/kwbroman}{\tt \scriptsize \color{foreground} @kwbroman}
\\[-4pt]
{\scriptsize Course web: \href{https://kbroman.org/AdvData}{\tt kbroman.org/AdvData}}
}

\newcommand{\lod}{\text{LOD}}
\newcommand{\plod}{\text{pLOD}}
\newcommand{\M}{\text{M}}

\begin{document}

{
\setbeamertemplate{footline}{} % no page number here
\frame{
  \titlepage

\note{
  In this lecture, we'll look at building regression models, and
  particularly at the case of trying to map multiple QTL. Our interest
  here is not in prediction but rather in identifying the important
  variables.
}
} }




\begin{frame}[c]{LOD curves}

\figw{Figs/alod.pdf}{1.0}

\note{
  I've talked a lot about QTL mapping: seeking to identify the set of
  genetic loci in the genome that affect some quantitative trait, by
  doing an experimental cross and then performing ANOVA at each locus,
  one at a time.

  This basic analysis considers each predictor in isolation, though we
  expect there to be multiple loci that are influencing the trait.

  Today, I want to talk about efforts to go after multiple loci at
  once.
}
\end{frame}





\begin{frame}{Example}

\footnotesize
Sugiyama et al. Genomics 71:70-77, 2001

\bi
\item 250 male mice from the backcross (A $\times$ B) $\times$ B
\item Blood pressure after two weeks drinking water with 1\% NaCl
\ei

\figw{Figs/pheno.pdf}{1.0}

\note{
   As an example, I'll focus on this backcross of 250 mice, with the
   outcome being blood pressure. At each position, mice have one of
   two genotypes.
}
\end{frame}







\begin{frame}[c]{Genotype data}

\figw{Figs/genodata.pdf}{1.0}

\note{
  I like this example because the genotyping strategy makes the
  analysis quite difficult. There was a selective genotyping approach,
  in which only the 46 mice with highest blood pressure and the 46
  mice with lowest blood pressure were genotyped. But then they did
  an initial analysis and genotyped all the mice at regions of
  potential interest. And then they further genotyped apparent
  recombinant individuals in selected regions, to try to further hone
  in on QTL locations.
}
\end{frame}





\begin{frame}{Goals}

\bbi
\item Identify quantitative trait loci (QTL) \\[6pt]
   {\lolit   (and interactions among QTL)}
\item Interval estimates of QTL location
\item Estimated QTL effects
\ei

\note{
}
\end{frame}




\begin{frame}[c]{LOD curves}

\figw{Figs/alod.pdf}{1.0}

\note{
}
\end{frame}






\begin{frame}{Estimated effects}

\figw{Figs/meffects.pdf}{1.0}

\note{
}
\end{frame}






\begin{frame}{Modeling multiple QTL}

\bbi
\item Reduce residual variation $\longrightarrow$ increased power
\item Separate linked QTL
\item Identify interactions among QTL {\lolit (epistasis)}
\ei
\note{
}
\end{frame}






\begin{frame}[c]{Epistasis in BC}

\figw{Figs/epistasis_bc.pdf}{1.0}

\note{
}
\end{frame}






\begin{frame}[c]{Epistasis in F$_{\mathsf{2}}$}

\figw{Figs/epistasis_f2.pdf}{1.0}

\note{
}
\end{frame}











\begin{frame}[c]{Estimated effects}

\figw{Figs/ieffects.pdf}{1.0}

\note{
}
\end{frame}

















\begin{frame}[c]{Model selection}


  \begin{columns}
    \column{0.5\textwidth}

\bi
\item Class of models
    \bi
\item Additive models
\item + pairwise interactions
\item + higher-order interactions
\item Regression trees
\ei

\bigskip

\item Model fit
 \bi
\item Maximum likelihood
\item Haley-Knott regression
\item extended Haley-Knott
\item Multiple imputation
\item MCMC
\ei

\ei

\column{0.5\textwidth}

\bi
\item Model comparison
\bi
\item Estimated prediction error
\item AIC, BIC, penalized likelihood
\item Bayes
\item[]
\ei

\bigskip

\item Model search
\bi
\item Forward selection
\item Backward elimination
\item Stepwise selection
\item Randomized algorithms
\ei

\ei
\end{columns}

\note{
}
\end{frame}









\begin{frame}{Target}


\bbi

\item Selection of a model includes two types of errors:
\bi
\item Miss important terms (QTLs or interactions)
\item Include extraneous terms
\ei

\item Unlike in hypothesis testing, we can make {\hilit both errors} at
the same time.

\item {\hilit Identify as many correct terms as possible, while \\
{\vhilit controlling the rate of inclusion of extraneous terms}.}
\ei

\note{
}
\end{frame}









\begin{frame}{What is special here?}


\bbi

\item Goal: identify the major players
  \bi
  \item {\vhilit not} prediction
  \ei

\item A continuum of ordinal-valued covariates (the genetic loci)

\item Association among the covariates

\bi
\item Loci on different chromosomes are independent
\item Along chromosome, a very simple (and known) correlation
  structure
\ei

\ei

\note{
}
\end{frame}










\begin{frame}{Exploratory methods}


\bbi
\item Condition on a large-effect QTL


\bi
\item Reduce residual variation
\item Conditional LOD score:

\hspace{3em} $ \displaystyle{\mathsf{\lod(q_2 \ | \ q_1) = \text{log}_{10}
    \left\{\frac{\text{Pr}(\text{data} \ | \ q_1, q_2)}{
    \text{Pr}(\text{data} \ | \ q_1)}\right\} }}$
\ei


\bigskip

\item Piece together the putative QTL from the 1d and 2d scans

\bi
\item Omit loci that no longer look interesting (drop-one-at-a-time analysis)
\item Study potential interactions among the identified loci
\item Scan for additional loci (perhaps allowing interactions), conditional on these
\ei

\ei

\note{
}
\end{frame}






\begin{frame}[c]{Controlling for chr 4}

\figw{Figs/alod_c4.pdf}{1.0}

\note{
}
\end{frame}









\begin{frame}[fragile,c]{Drop-one-QTL table}

\begin{verbatim}
                                  df    LOD   %var
                1@68.3             1   6.30   11.0
                4@30.0             1  12.21   20.1
                6@61.0             2   7.93   13.6
                15@17.5            2   7.14   12.3
                6@61.0 : 15@17.5   1   5.68    9.9
\end{verbatim}

\note{
}
\end{frame}






\begin{frame}{Automation}
\bbi
\item Assistance to non-specialists

\item Understanding performance

\item Many phenotypes
\ei

\note{
}
\end{frame}






\begin{frame}{Additive QTL}


\vspace{-2mm}

\bi
\item[] \hspace{-2em} Simple situation:

\bi
\item Dense markers
\item Complete genotype data
\item No epistasis
\ei \ei

\bigskip \bigskip

\centerline{
$\mathsf{y  = \mu + \sum \beta_j \, q_j + \epsilon}$ \hspace{1cm}
       {\hilit which $\mathsf{\beta_j \ne 0}$?}
}

\bigskip \bigskip

{\hilit
$\mathsf{ \plod(\gamma) = \lod(\gamma) -
    {\hilit T} \, |\gamma| }$
}


\bigskip

\only<2>{
0 vs 1 QTL:

{\lolit
\qquad $\mathsf{\plod(\emptyset) = 0}$ \\
\qquad $\mathsf{\plod(\{\lambda\}) =
    \lod(\lambda) - {\hilit T}}$
} }


\only<3|handout 0>{
For the mouse genome: \\
\qquad $\mathsf{\hilit T}$ = {\lolit
  2.69} (BC) or {\lolit 3.52} (F$_{\mathsf{2}}$)
}

\note{
}
\end{frame}








\begin{frame}{Experience}

\bbi
\item Controls rate of inclusion of extraneous terms
\item Forward selection over-selects
\item {\vhilit Forward selection followed by backward elimination} works as well
  as MCMC
\item {\hilit Need to define performance criteria}
\item {\hilit Need large-scale simulations}
\ei

\bigskip \bigskip \bigskip

\footnotesize
\hfill Broman \& Speed, JRSS B 64:641-656, 2002 \\
\hfill \href{https://doi.org/10.1111/1467-9868.00354}{\tt 10.1111/1467-9868.00354}

\note{
}
\end{frame}







\begin{frame}{Epistasis}

\bigskip \bigskip \bigskip

\centerline{
$\mathsf{y  = \mu + \sum \beta_j \, q_j + \sum \gamma_{jk} \, q_j \,
    q_k + \epsilon}$
}

\bigskip \bigskip \bigskip

{\hilit
$\mathsf{ \plod(\gamma) = \lod(\gamma) -
    {\hilit T_m} \, |\gamma|_m - {\hilit T_i} \, |\gamma|_i }$
}


\bigskip \bigskip \bigskip

\hspace{3em} $\mathsf{\hilit T_m}$ = as chosen previously

\bigskip

\hspace{3em} $\mathsf{\hilit T_i}$ = ?

\note{
}
\end{frame}






\begin{frame}{Idea 1}


\hfill \begin{minipage}{10in}

Imagine there are two additive QTL and consider a 2d, 2-QTL scan.

\vspace{1cm}

\hspace*{0.5in} $\mathsf{\hilit T_i}$ = 95th percentile of the
  distribution of \\[6pt]
\hspace*{1.3in} {\lolit $\mathsf{ \text{max} \, \lod_f(s,t) -
    \text{max} \, \lod_a(s,t)}$}


\end{minipage}



\note{
}
\end{frame}







\addtocounter{page}{-1}

\begin{frame}{Idea 1}


\hfill \begin{minipage}{10in}

Imagine there are two additive QTL and consider a 2d, 2-QTL scan.

\vspace{1cm}

\hspace*{0.5in} $\mathsf{\hilit T_i}$ = 95th percentile of the
  distribution of \\[6pt]
\hspace*{1.3in} {\lolit $\mathsf{ \text{max} \, \lod_f(s,t) -
    \text{max} \, \lod_a(s,t)}$}


\vspace{2cm}

For the mouse genome: \\[12pt]
\hspace*{0.5in} $\mathsf{\hilit T_m}$ = {\lolit
  2.69} (BC) or {\lolit 3.52} (F$_{\mathsf{2}}$) \\[12pt]
\hspace*{0.5in} $\mathsf{\hilit T^H_i}$ = {\lolit
  2.62} (BC) or {\lolit 4.28} (F$_{\mathsf{2}}$)


\end{minipage}



\note{
}
\end{frame}







\begin{frame}{Idea 2}


\hfill \begin{minipage}{10in}

Imagine there is one QTL and consider a 2d, 2-QTL scan.

\vspace{1cm}

\hspace*{0.5in} $\mathsf{\hilit T_m + T_i}$ = 95th percentile of the
  distribution of \\[6pt]
\hspace*{2.0in} {\lolit $\mathsf{ \text{max} \, \lod_f(s,t) -
    \text{max} \, \lod_1(s)}$}


\end{minipage}



\note{
}
\end{frame}







\begin{frame}{Idea 2}


\hfill \begin{minipage}{10in}

Imagine there is one QTL and consider a 2d, 2-QTL scan.

\vspace{1cm}

\hspace*{0.5in} $\mathsf{\hilit T_m + T_i}$ = 95th percentile of the
  distribution of \\[6pt]
\hspace*{2.0in} {\lolit $\mathsf{ \text{max} \, \lod_f(s,t) -
    \text{max} \, \lod_1(s)}$}


\vspace{2cm}

For the mouse genome: \\[12pt]
\hspace*{0.5in} $\mathsf{\hilit T_m}$ = {\lolit
  2.69} (BC) or {\lolit 3.52} (F$_{\mathsf{2}}$) \\[12pt]
\hspace*{0.5in} $\mathsf{\hilit T^H_i}$ = {\lolit
  2.62} (BC) or {\lolit 4.28} (F$_{\mathsf{2}}$) \\[12pt]
\hspace*{0.5in} $\mathsf{\hilit T^L_i}$ = {\lolit
  1.19} (BC) or {\lolit 2.69} (F$_{\mathsf{2}}$)


\end{minipage}

\note{
}
\end{frame}





\begin{frame}[c]{Models as graphs}

\figw{Figs/models.pdf}{1.0}

\note{
}
\end{frame}





\begin{frame}[c]{Results}

\only<1>{\figw{Figs/hyper_models1.pdf}{1.0}}
\only<2|handout 0>{\figw{Figs/hyper_models2.pdf}{1.0}}
\only<3|handout 0>{\figw{Figs/hyper_models3.pdf}{1.0}}

\note{
}
\end{frame}



\begin{frame}{Profile LOD curves}

\figw{Figs/lod_profile.pdf}{1.0}

\note{
}
\end{frame}




\begin{frame}[fragile,c]{Drop-one-QTL table}

\begin{verbatim}
                                  df    LOD   %var
                1@68.3             1   6.30   11.0
                4@30.0             1  12.21   20.1
                6@61.0             2   7.93   13.6
                15@17.5            2   7.14   12.3
                6@61.0 : 15@17.5   1   5.68    9.9
\end{verbatim}

\note{
}
\end{frame}




\begin{frame}[c]{Add an interaction?}

\only<1>{\figw{Figs/hyper_models4.pdf}{1.0}}
\only<2|handout 0>{\figw{Figs/hyper_models5.pdf}{1.0}}
\only<3|handout 0>{\figw{Figs/hyper_models6.pdf}{1.0}}
\only<4|handout 0>{\figw{Figs/hyper_models7.pdf}{1.0}}
\only<5|handout 0>{\figw{Figs/hyper_models8.pdf}{1.0}}

\note{
}
\end{frame}



\begin{frame}[c]{Add another QTL?}

\only<1>{\figw{Figs/hyper_models9.pdf}{1.0}}
\only<2|handout 0>{\figw{Figs/hyper_models10.pdf}{1.0}}
\only<3|handout 0>{\figw{Figs/hyper_models11.pdf}{1.0}}
\only<4|handout 0>{\figw{Figs/hyper_models12.pdf}{1.0}}

\note{
}
\end{frame}







\begin{frame}{Summary}


\bbi
\item QTL mapping is a model selection problem
\item The problem is finding the major players, not minimizing
  prediction error
\item The criterion for comparing models is most important
\item We're focusing on a penalized likelihood method, with penalties
  derived from permutation tests with 1d and 2d scans
\item Manichaikul et al., Genetics 181:1077--1086, 2009 \\
  \href{https://doi.org/10.1534/genetics.108.094565}{\tt doi:10.1534/genetics.108.094565}
\ei

\note{
}
\end{frame}



\end{document}
