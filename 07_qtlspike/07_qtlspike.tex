\documentclass[aspectratio=169,12pt,t]{beamer}
\usepackage{graphicx}
\setbeameroption{hide notes}
\setbeamertemplate{note page}[plain]
\usepackage{listings}

\input{../LaTeX/header.tex}

%%%%%%%%%%%%%%%%%%%%%%%%%%%%%%%%%%%%%%%%%%%%%%%%%%%%%%%%%%%%%%%%%%%%%%
% end of header
%%%%%%%%%%%%%%%%%%%%%%%%%%%%%%%%%%%%%%%%%%%%%%%%%%%%%%%%%%%%%%%%%%%%%%

% title info
\title{The EM algorithm}
\subtitle{QTL mapping with a cure model}
\author{\href{https://kbroman.org}{Karl Broman}}
\institute{Biostatistics \& Medical Informatics, UW{\textendash}Madison}
\date{\href{https://kbroman.org}{\tt \scriptsize \color{foreground} kbroman.org}
\\[-4pt]
\href{https://github.com/kbroman}{\tt \scriptsize \color{foreground} github.com/kbroman}
\\[-4pt]
\href{https://twitter.com/kwbroman}{\tt \scriptsize \color{foreground} @kwbroman}
\\[-4pt]
{\scriptsize Course web: \href{https://kbroman.org/AdvData}{\tt kbroman.org/AdvData}}
}


\begin{document}

% title slide
{
\setbeamertemplate{footline}{} % no page number here
\frame{
  \titlepage

  \note{}

} }



\begin{frame}[c]{Intercross}
\figw{Figs/intercross.pdf}{1.0}
\end{frame}





\begin{frame}[c]{QTL mapping}

\vspace{5mm}
\figw{Figs/lodcurve_insulin_with_effects.pdf}{0.96}
\end{frame}




\begin{frame}{References}
  \bbi

\item Lander ES, Botstein D (1989) Mapping Mendelian factors
  underlying quantitative traits using RFLP linkage maps. Genetics
  121:185-199 \\
  \href{https://www.ncbi.nlm.nih.gov/pmc/articles/PMC1203601}{\footnotesize
    PMCID: PMC1203601}

\item Broman KW (2001) Review of statistical methods for QTL mapping
  in experimental crosses. Lab Animal 30(7):44-52 \\
  \href{https://www.ncbi.nlm.nih.gov/pubmed/11469113}{\footnotesize
    PMID: 11469113}

\item Broman KW (2003) Mapping quantitative trait loci in the case
  of a spike in the phenotype distribution. Genetics 163:1169-1175 \\
  \href{https://www.ncbi.nlm.nih.gov/pmc/articles/PMC1462498}{\footnotesize
    PMCID: PMC1462498}

  \ei

\end{frame}


\end{document}
