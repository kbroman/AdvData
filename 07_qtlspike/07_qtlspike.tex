\documentclass[aspectratio=169,12pt,t]{beamer}
\usepackage{graphicx}
\setbeameroption{hide notes}
\setbeamertemplate{note page}[plain]
\usepackage{listings}

% header.tex: boring LaTeX/Beamer details + macros

% get rid of junk
\usetheme{default}
\beamertemplatenavigationsymbolsempty
\hypersetup{pdfpagemode=UseNone} % don't show bookmarks on initial view


% font
\usepackage{fontspec}
\setsansfont
  [ ExternalLocation = ../fonts/ ,
    UprightFont = *-regular ,
    BoldFont = *-bold ,
    ItalicFont = *-italic ,
    BoldItalicFont = *-bolditalic ]{texgyreheros}
\setbeamerfont{note page}{family*=pplx,size=\footnotesize} % Palatino for notes
% "TeX Gyre Heros can be used as a replacement for Helvetica"
% I've placed them in fonts/; alternatively you can install them
% permanently on your system as follows:
%     Download http://www.gust.org.pl/projects/e-foundry/tex-gyre/heros/qhv2.004otf.zip
%     In Unix, unzip it into ~/.fonts
%     In Mac, unzip it, double-click the .otf files, and install using "FontBook"

% named colors
\definecolor{offwhite}{RGB}{255,250,240}
\definecolor{gray}{RGB}{155,155,155}
\definecolor{purple}{RGB}{177,13,201}
\definecolor{green}{RGB}{46,204,64}

\definecolor{background}{RGB}{255,255,255}
\definecolor{foreground}{RGB}{24,24,24}
\definecolor{title}{RGB}{27,94,134}
\definecolor{subtitle}{RGB}{22,175,124}
\definecolor{hilit}{RGB}{122,0,128}
\definecolor{vhilit}{RGB}{255,0,128}
\definecolor{codehilit}{RGB}{255,0,128}
\definecolor{lolit}{RGB}{95,95,95}
\definecolor{myyellow}{rgb}{1,1,0.7}
\definecolor{nhilit}{RGB}{128,0,128}  % hilit color in notes
\definecolor{nvhilit}{RGB}{255,0,128} % vhilit for notes

\newcommand{\hilit}{\color{hilit}}
\newcommand{\vhilit}{\color{vhilit}}
\newcommand{\nhilit}{\color{nhilit}}
\newcommand{\nvhilit}{\color{nvhilit}}
\newcommand{\lolit}{\color{lolit}}

% use those colors
\setbeamercolor{titlelike}{fg=title}
\setbeamercolor{subtitle}{fg=subtitle}
\setbeamercolor{institute}{fg=lolit}
\setbeamercolor{normal text}{fg=foreground,bg=background}
\setbeamercolor{item}{fg=foreground} % color of bullets
\setbeamercolor{subitem}{fg=lolit}
\setbeamercolor{itemize/enumerate subbody}{fg=lolit}
\setbeamertemplate{itemize subitem}{{\textendash}}
\setbeamerfont{itemize/enumerate subbody}{size=\footnotesize}
\setbeamerfont{itemize/enumerate subitem}{size=\footnotesize}

% page number
\setbeamertemplate{footline}{%
    \raisebox{5pt}{\makebox[\paperwidth]{\hfill\makebox[20pt]{\lolit
          \scriptsize\insertframenumber}}}\hspace*{5pt}}

% add a bit of space at the top of the notes page
\addtobeamertemplate{note page}{\setlength{\parskip}{12pt}}

% default link color
\hypersetup{colorlinks, urlcolor={hilit}}

\lstset{language=bash,
        basicstyle=\ttfamily\scriptsize,
        frame=single,
        commentstyle=,
        backgroundcolor=\color{offwhite},
        showspaces=false,
        showstringspaces=false
        }


% a few macros
\newcommand{\bi}{\begin{itemize}}
\newcommand{\bbi}{\vspace{24pt} \begin{itemize} \itemsep8pt}
\newcommand{\ei}{\end{itemize}}
\newcommand{\be}{\begin{enumerate}}
\newcommand{\bbe}{\vspace{24pt} \begin{enumerate} \itemsep8pt}
\newcommand{\ee}{\end{enumerate}}
\newcommand{\ig}{\includegraphics}
\newcommand{\subt}[1]{{\footnotesize \color{subtitle} {#1}}}
\newcommand{\ttsm}{\tt \small}
\newcommand{\ttfn}{\tt \footnotesize}
\newcommand{\figh}[2]{\centerline{\includegraphics[height=#2\textheight]{#1}}}
\newcommand{\figw}[2]{\centerline{\includegraphics[width=#2\textwidth]{#1}}}


%%%%%%%%%%%%%%%%%%%%%%%%%%%%%%%%%%%%%%%%%%%%%%%%%%%%%%%%%%%%%%%%%%%%%%
% end of header
%%%%%%%%%%%%%%%%%%%%%%%%%%%%%%%%%%%%%%%%%%%%%%%%%%%%%%%%%%%%%%%%%%%%%%

% title info
\title{The EM algorithm}
\subtitle{QTL mapping with a cure model}
\author{\href{https://kbroman.org}{Karl Broman}}
\institute{Biostatistics \& Medical Informatics, UW{\textendash}Madison}
\date{\href{https://kbroman.org}{\tt \scriptsize \color{foreground} kbroman.org}
\\[-4pt]
\href{https://github.com/kbroman}{\tt \scriptsize \color{foreground} github.com/kbroman}
\\[-4pt]
\href{https://twitter.com/kwbroman}{\tt \scriptsize \color{foreground} @kwbroman}
\\[-4pt]
{\scriptsize Course web: \href{https://kbroman.org/AdvData}{\tt kbroman.org/AdvData}}
}


\begin{document}

% title slide
{
\setbeamertemplate{footline}{} % no page number here
\frame{
  \titlepage

  \note{}

} }



\begin{frame}[c]{Intercross}
\figw{Figs/intercross.pdf}{1.0}
\end{frame}





\begin{frame}[c]{QTL mapping}

\vspace{5mm}
\figw{Figs/lodcurve_insulin_with_effects.pdf}{0.96}
\end{frame}




\begin{frame}[c]{Phenotype data}
\figw{Figs/pheno.pdf}{1.0}

\vspace{5mm}

{\lolit \footnotesize
Sugiyama et al.\ (2002) Physiol Genomics 10:5--12
}
\end{frame}


\begin{frame}[c]{Genotype data}
\figw{Figs/genodata.pdf}{1.0}
\end{frame}


\begin{frame}[c]{Genetic map}
\figw{Figs/geneticmap.pdf}{1.0}
\end{frame}



\begin{frame}[c]{ANOVA at marker loci}

\begin{columns}

\column{0.5\textwidth}
\bi
\item Also known as {\hilit marker regression}.
\item Split mice into groups according to genotype at a marker.
\item Do a t-test / ANOVA.
\item Repeat for each marker.
\ei

\column{0.5\textwidth}

\figw{Figs/anova.pdf}{1.0}

\end{columns}
\end{frame}



\begin{frame}{ANOVA at marker loci}

\begin{columns}
\column{0.5\textwidth}

{\hilit Advantages}

\bi
\item Simple.
\item Easily incorporates covariates.
\item Easily extended to more complex models.
\item Doesn't require a genetic map.
\ei


\column{0.5\textwidth}

{\hilit Disadvantages}

\bi
\item Must exclude individuals with missing genotype data.
\item Imperfect information about QTL location.
\item Suffers in low density scans.
\item {\vhilit Only considers one QTL at a time.}
\ei

\end{columns}

\end{frame}


\begin{frame}{References}
\vspace{-7mm}

  \bbi

\item Lander ES, Botstein D (1989) Mapping Mendelian factors
  underlying quantitative traits using RFLP linkage maps. Genetics
  121:185-199 \\
  \href{https://www.ncbi.nlm.nih.gov/pmc/articles/PMC1203601}{\footnotesize
    PMCID: PMC1203601}

\item Broman KW (2001) Review of statistical methods for QTL mapping
  in experimental crosses. Lab Animal 30(7):44-52 \\
  \href{https://www.ncbi.nlm.nih.gov/pubmed/11469113}{\footnotesize
    PMID: 11469113}

\item Boyartchuk VL, et al. (2001) Multigenic control of Listeria monocytogenes
  susceptibility in mice. Nat Genet 27:259-260 \\
  \href{https://doi.org/10.1038/85812}{\footnotesize doi:10.1038/85812}

\item Broman KW (2003) Mapping quantitative trait loci in the case
  of a spike in the phenotype distribution. Genetics 163:1169-1175 \\
  \href{https://www.ncbi.nlm.nih.gov/pmc/articles/PMC1462498}{\footnotesize
    PMCID: PMC1462498}

\ei


\end{frame}


\end{document}
