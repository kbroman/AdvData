\documentclass[aspectratio=169,12pt,t]{beamer}
\usepackage{graphicx}
\setbeameroption{hide notes}
\setbeamertemplate{note page}[plain]
\usepackage{listings}

% header.tex: boring LaTeX/Beamer details + macros

% get rid of junk
\usetheme{default}
\beamertemplatenavigationsymbolsempty
\hypersetup{pdfpagemode=UseNone} % don't show bookmarks on initial view


% font
\usepackage{fontspec}
\setsansfont
  [ ExternalLocation = ../fonts/ ,
    UprightFont = *-regular ,
    BoldFont = *-bold ,
    ItalicFont = *-italic ,
    BoldItalicFont = *-bolditalic ]{texgyreheros}
\setbeamerfont{note page}{family*=pplx,size=\footnotesize} % Palatino for notes
% "TeX Gyre Heros can be used as a replacement for Helvetica"
% I've placed them in fonts/; alternatively you can install them
% permanently on your system as follows:
%     Download http://www.gust.org.pl/projects/e-foundry/tex-gyre/heros/qhv2.004otf.zip
%     In Unix, unzip it into ~/.fonts
%     In Mac, unzip it, double-click the .otf files, and install using "FontBook"

% named colors
\definecolor{offwhite}{RGB}{255,250,240}
\definecolor{gray}{RGB}{155,155,155}
\definecolor{purple}{RGB}{177,13,201}
\definecolor{green}{RGB}{46,204,64}

\definecolor{background}{RGB}{255,255,255}
\definecolor{foreground}{RGB}{24,24,24}
\definecolor{title}{RGB}{27,94,134}
\definecolor{subtitle}{RGB}{22,175,124}
\definecolor{hilit}{RGB}{122,0,128}
\definecolor{vhilit}{RGB}{255,0,128}
\definecolor{codehilit}{RGB}{255,0,128}
\definecolor{lolit}{RGB}{95,95,95}
\definecolor{myyellow}{rgb}{1,1,0.7}
\definecolor{nhilit}{RGB}{128,0,128}  % hilit color in notes
\definecolor{nvhilit}{RGB}{255,0,128} % vhilit for notes

\newcommand{\hilit}{\color{hilit}}
\newcommand{\vhilit}{\color{vhilit}}
\newcommand{\nhilit}{\color{nhilit}}
\newcommand{\nvhilit}{\color{nvhilit}}
\newcommand{\lolit}{\color{lolit}}

% use those colors
\setbeamercolor{titlelike}{fg=title}
\setbeamercolor{subtitle}{fg=subtitle}
\setbeamercolor{institute}{fg=lolit}
\setbeamercolor{normal text}{fg=foreground,bg=background}
\setbeamercolor{item}{fg=foreground} % color of bullets
\setbeamercolor{subitem}{fg=lolit}
\setbeamercolor{itemize/enumerate subbody}{fg=lolit}
\setbeamertemplate{itemize subitem}{{\textendash}}
\setbeamerfont{itemize/enumerate subbody}{size=\footnotesize}
\setbeamerfont{itemize/enumerate subitem}{size=\footnotesize}

% page number
\setbeamertemplate{footline}{%
    \raisebox{5pt}{\makebox[\paperwidth]{\hfill\makebox[20pt]{\lolit
          \scriptsize\insertframenumber}}}\hspace*{5pt}}

% add a bit of space at the top of the notes page
\addtobeamertemplate{note page}{\setlength{\parskip}{12pt}}

% default link color
\hypersetup{colorlinks, urlcolor={hilit}}

\lstset{language=bash,
        basicstyle=\ttfamily\scriptsize,
        frame=single,
        commentstyle=,
        backgroundcolor=\color{offwhite},
        showspaces=false,
        showstringspaces=false
        }


% a few macros
\newcommand{\bi}{\begin{itemize}}
\newcommand{\bbi}{\vspace{24pt} \begin{itemize} \itemsep8pt}
\newcommand{\ei}{\end{itemize}}
\newcommand{\be}{\begin{enumerate}}
\newcommand{\bbe}{\vspace{24pt} \begin{enumerate} \itemsep8pt}
\newcommand{\ee}{\end{enumerate}}
\newcommand{\ig}{\includegraphics}
\newcommand{\subt}[1]{{\footnotesize \color{subtitle} {#1}}}
\newcommand{\ttsm}{\tt \small}
\newcommand{\ttfn}{\tt \footnotesize}
\newcommand{\figh}[2]{\centerline{\includegraphics[height=#2\textheight]{#1}}}
\newcommand{\figw}[2]{\centerline{\includegraphics[width=#2\textwidth]{#1}}}


%%%%%%%%%%%%%%%%%%%%%%%%%%%%%%%%%%%%%%%%%%%%%%%%%%%%%%%%%%%%%%%%%%%%%%
% end of header
%%%%%%%%%%%%%%%%%%%%%%%%%%%%%%%%%%%%%%%%%%%%%%%%%%%%%%%%%%%%%%%%%%%%%%

\title{KnitR + \LaTeX\/ $\rightarrow$ paper}
\author{\href{https://kbroman.org}{Karl Broman}}
\institute{Biostatistics \& Medical Informatics, UW{\textendash}Madison}
\date{\href{https://kbroman.org}{\tt \scriptsize \color{foreground} kbroman.org}
\\[-4pt]
\href{https://github.com/kbroman}{\tt \scriptsize \color{foreground} github.com/kbroman}
\\[-4pt]
\href{https://twitter.com/kwbroman}{\tt \scriptsize \color{foreground} @kwbroman}
\\[-4pt]
{\scriptsize Course web: \href{https://kbroman.org/AdvData}{\tt kbroman.org/AdvData}}
}

\begin{document}

{
\setbeamertemplate{footline}{} % no page number here
\frame{
  \titlepage

\note{
  This lecture is about how to create reproducible manuscripts, for
  journal articles. KnitR with R Markdown is great for informal
  reports. KnitR with AsciiDoc is great for somewhat fancier
  reports. There are a number of efforts, especially with Pandoc, to
  use R Markdown for journal articles. But if you want fine control
  over the appearance of a document, it's hard to beat \LaTeX, and so
  I'm just going to focus on that.

  I can't hope to explain \LaTeX\/ properly in just this one
  lecture. My goals are to give the general gist, indicate resources
  and options, and show how to use KnitR with \LaTeX.

  I also want to discuss some more general strategies for ensuring that
  the results described in a journal article are fully reproducible.
}
} }


\begin{frame}[c]{R Markdown for everything}

  \bbi
\item \href{https://bookdown.org/yihui/bookdown}{bookdown} for books {\lolit (or book-like things)}
\item \href{https://bookdown.org/yihui/blogdown}{blogdown} for websites
\item \href{https://pkgdown.r-lib.org}{pkgdown} for package websites
\item \href{https://slides.yihui.org/xaringan}{xaringan} for slides
\item \href{https://pagedown.rbind.io}{pagedown} for CVs, resumes, and letters
\item \href{https://github.com/brentthorne/posterdown}{posterdown} for posters
  \ei

  \note{
     You can now use R Markdown for everything: books, websites,
     package websites, slides, resumses, letters, and posters.

     But I still am using \LaTeX\/ for lots of things. I should probably
     figure out how to use R Markdown instead, but either for the fine
     control or just because I have so much experience to draw upon, I
     find it hard to leave \LaTeX\/ behind.
  }
\end{frame}


\begin{frame}[c,fragile]{\LaTeX}

\begin{lstlisting}
\documentclass[12pt]{article}

\usepackage{graphicx}

\title{An example document}
\author{Karl Broman}

\begin{document}

\maketitle
\thispagestyle{empty}

\section{A section}

This is a simple example of a \LaTeX\/ document for an article.
Here's some in-line math: $y = \beta_0 + \beta_1 x + \epsilon$.

And here's a display equation:

$$ \hat{\beta} = (X'X)^{-1} X'y $$

\end{document}
\end{lstlisting}

\note{
  \LaTeX\/ is like html or Markdown: plain text with special codes to
  indicate how things are to appear.

  A \LaTeX\/ document always starts with {\tt {\textbackslash}documentclass}, then a
  bunch of overall controlling information. The actual document is
  between {\tt {\textbackslash}begin\{document\}} and {\tt {\textbackslash}end\{document\}}.

  {\tt {\textbackslash}usepackage\{\}} is like {\tt library()} in R.

  Ideally, you focus on {\nhilit semantics} rather than {\nhilit
  style}: define the {\tt {\textbackslash}title\{\}} and {\tt {\textbackslash}author\{\}} and use
  {\tt {\textbackslash}maketitle} to have them included in the document, and
  indicate sections and subsections with {\tt {\textbackslash}section\{\}} and
  {\tt {\textbackslash}subsection\{\}}.

  For some reason, {\tt {\textbackslash}thispagestyle\{empty\}} (``don't
  show page number on this page'') needs
  to be placed {\nhilit after} {\tt {\textbackslash}maketitle}.

  A key feature of \LaTeX\/ is the mathematics typesetting. There's no
  better system. And your \LaTeX\/ skills can be immediately
  transferred to your Markdown documents, with MathJax.
}
\end{frame}


\begin{frame}[c,fragile]{What I actually do}

\begin{lstlisting}
\documentclass[12pt]{article}

\setlength{\headheight}{10pt}
\setlength{\headsep}{15pt}
\setlength{\topmargin}{-25pt}
\setlength{\topskip}{0in}
\setlength{\textheight}{8.7in}
\setlength{\footskip}{0.3in}
\setlength{\oddsidemargin}{0.0in}
\setlength{\evensidemargin}{0.0in}
\setlength{\textwidth}{6.5in}

\begin{document}
\begin{center}
\textbf{\large An example document}

\vspace{10mm}
Karl Broman
\end{center}

\vspace{30mm}
\textbf{\sffamily A section}
\end{lstlisting}

\note{
  In reality, for a paper, I don't use {\tt {\textbackslash}maketitle}
  or {\tt {\textbackslash}section}, but rather just muck about,
  hard-coding the placement of things.

  But mine is not the recommended approach. If, for some reason, you
  need to change the style, it's easier if your document is defined in
  terms of {\nhilit semantics}.
}
\end{frame}


\begin{frame}{Why \LaTeX?}

\bbi
\item Fine control of document appearance
\item Transparency of how that was achieved
\item Version control (diff/merge)
\item Typesetting equations
\item Markdown's not quite ready, or sufficiently rich
  \bi
  \item[] (but see the R package \href{https://github.com/rstudio/rticles}{rticles})
  \ei
\ei


\note{
  It's {\nhilit a lot} of work to learn \LaTeX, so we need to be clear
  about why we'd want to devote the effort to it.

  For reproducible research, we need some sort of code-based document
  system (i.e., {\nvhilit not Word!}), and \LaTeX\/ gives you the most
  fine-grained control, if you need it. Ultimately, I hope, Markdown
  will be sufficient, but for now, we often need \LaTeX.

  The code-based control makes what you're trying to do
  transparent.  And you should treat \LaTeX\/ like code: write clearly
  and simply, and comment the tricky bits.

  This sort of document also has the advantage of easy treatment of
  {\tt diff} and {\tt merge} in a version control system like git.

  The real power of \LaTeX\/ is in the typesetting of mathematical
  equations. And what you learn on that aspect can be transferred to
  your Markdown documents, using MathJax. (But I already said that,
  didn't I?)
}
\end{frame}


\begin{frame}[c]{}

\vspace{10mm}

\centerline{\Large simple \quad $\longleftrightarrow$ \quad flexible}

\vspace{10mm}

\onslide<2->{\centerline{\tt \scriptsize \lolit {\textbackslash}centerline\{{\textbackslash}Large simple
{\textbackslash}quad \${\textbackslash}longleftrightarrow\$ {\textbackslash}quad flexible\}}}

\note{
  \LaTeX\/ sits at the right of the simple-to-flexible spectrum.
}
\end{frame}



\begin{frame}[c]{}

\centering
Modify your desires to match the defaults.

\vspace{36pt}

Focus your compulsive behavior on things that matter.

\note{
  I've said this before, but I like to repeat it.

  Focus on the text and the figures before worrying too much about
  fine details of how they appear on the page.

  And consider which is more important: a manuscript, web page, blog,
  grant, course slides, course handout, report to collaborator,
  scientific poster.

  You can spend a ton of time trying to get things to look just
  right. Ideally, you spend that time trying to construct a general
  solution.  Or you can modify your desires to more closely match what
  you get without any effort.
}
\end{frame}


\begin{frame}[c,fragile]{Stuff I use a lot}

\begin{lstlisting}
% other fonts
\usepackage{palatino}
\usepackage{times}

\setlength{\rightskip}{0pt plus 1fil} % makes ragged right

\newcommand{\LOD}{\text{LOD}}

\usepackage{setspace}
\setstretch{2.0}

\addtocounter{framenumber}{-1}

% make figures S1, S2, ...
\renewcommand{\thefigure}{\textbf{S\arabic{figure}}}
\renewcommand{\figurename}{\textbf{Figure}}

% bigger space between rows in tables
\renewcommand{\arraystretch}{1.5}

% paragraphs not indented but have space between
\setlength{\parskip}{6pt}
\setlength{\parindent}{0pt}
\end{lstlisting}


\note{
  These are bits of \LaTeX\/ code that I use a lot.
}
\end{frame}





\begin{frame}<handout:0>[c,fragile]{KnitR + \LaTeX\/ $\rightarrow$ Rnw}

\begin{lstlisting}
\documentclass[12pt]{article}

\title{An example Rnw document}
\author{Karl Broman}

\begin{document}
\maketitle

<<load_library, echo=FALSE, results="hide">>=
library(broman) # used for myround()
@

<<example_chunk>>=
x <- rnorm(100)
y <- 5*x + rnorm(100)
lm.out <- lm(y ~ x)
plot(x,y)
abline(lm.out$coef)
@

The estimated slope is \Sexpr{myround(lm.out$coef[2], 1)}.
\end{document}
\end{lstlisting}
\end{frame}



\begin{frame}[c,fragile]{KnitR + \LaTeX\/ $\rightarrow$ Rnw}
\addtocounter{framenumber}{-1}

\begin{lstlisting}
\documentclass[12pt]{article}

\title{An example Rnw document}
\author{Karl Broman}

\begin{document}
\maketitle

<<load_library, echo=FALSE, results="hide">>=
library(broman) # used for myround()
@

<<example_chunk, out.width="0.8\\textwidth">>=
x <- rnorm(100)
y <- 5*x + rnorm(100)
lm.out <- lm(y ~ x)
plot(x,y)
abline(lm.out$coef)
@

The estimated slope is \Sexpr{myround(lm.out$coef[2], 1)}.
\end{document}
\end{lstlisting}

\note{
  KnitR works well with \LaTeX.

  Most of what you learned about KnitR with R Markdown transfers
  directly to working with \LaTeX.

  The main difference is the way in which code chunks are
  indicated. You use {\tt <<>>=} and {\tt @} for chunks, and
  {\tt {\textbackslash}Sexpr\{\}} for in-line code.

  KnitR basically does a search-and-replace for code
  chunks. Different patterns will be easier, depending on the nature
  of the surrounding code.

  The chunk options are the same. Here, I used
  {\tt out.width="0.8{\textbackslash}textwidth"}
  to make the figure appear as 80\% of the width of the page.

  {\tt out.width} and {\tt out.height} need units as in \LaTeX\/ (built into
  {\tt {\textbackslash\textbackslash}textwidth}; otherwise {\tt "in"}
  or {\tt "cm"} or {\tt "pt"} or whatever).

  {\tt fig.width} and {\tt fig.height} are as in R, with implied units.
}
\end{frame}

\begin{frame}{LyX}

\vspace{20pt}

\figh{Figs/lyx.png}{0.7}

\vspace{20pt}

\hfill \href{http://www.lyx.org/}{lyx.org}

\note{
  I create \LaTeX\/ documents in emacs. If you want something WYSIWYG,
  consider LyX. KnitR is built in, and Yihui Xie strongly endorses
  it. (LyX is not really ``WYSIWYG'' but rather ``WYSIWYM,'' but
  that's what you want most, anyway.)
}
\end{frame}


\begin{frame}{Also}

\bbi
\item \href{http://overleaf.com}{Overleaf}
\item \href{http://authorea.com}{Authorea}
\ei

\note{
  There are a few online tools for creating \LaTeX\/ documents,
  collaboratively.

  I have no experience with these, but I've heard good things about
  Overleaf.
}
\end{frame}




\begin{frame}{Flavors of \LaTeX}

\bbi
\item \href{http://www.latex-project.org/}{\LaTeX}
\item \href{http://en.wikipedia.org/wiki/PdfTeX}{pdflatex}
\item \href{http://en.wikipedia.org/wiki/XeTeX}{xelatex}
\item \href{http://www.luatex.org/}{lualatex}
\ei

\note{
  In addition to regular \LaTeX, there's pdflatex (which I mostly
  use). It has the advantage of being able to include pdf, jpg, and
  png figures, and produces a PDF file directly.

  XeLaTeX and LuaLaTeX are great for fonts and Unicode.

  I've not mentioned that behind the scenes is \TeX, which is the
  source of all of this. Believe or not, \LaTeX\/ exists because
  \TeX\/ is even harder. PdfLaTeX, XeLaTeX, and LuoLaTeX, really
  derive from PdfTeX, XeTex, and LuoTex.
}
\end{frame}



\begin{frame}{Getting help}

\bbi
\item Google
\item \href{tex.stackexchange.com}{tex.stackexchange.com}
\item Ask a friend
\item Look at others' documents
\item Resign yourself to something less-than-ideal
\ei

\note{
  There is {\nhilit a ton} of online information about \LaTeX. Start
  with google. It's highly unlikely that you have a completely unique
  question or problem.

  My last point here is basically that one way to help yourself is by
  learning to let things go.
}
\end{frame}



\begin{frame}[c,fragile]{Figure captions and floats}

\begin{lstlisting}
<<fig_with_caption, fig.cap="Scatterplot of $y$ vs $x$">>=
x <- rnorm(100)
y <- 5*x + rnorm(100)
lm.out <- lm(y ~ x)
plot(x,y)
abline(lm.out$coef)
@
\end{lstlisting}

\bigskip

\begin{lstlisting}
\begin{figure}[]
\includegraphics{figure/fig_with_caption}

\caption{Scatterplot of $y$ vs $x$\label{fig:fig_with_caption}}
\end{figure}
\end{lstlisting}


\note{
  If you use the chunk option {\tt fig.cap}, the figure will get a
  caption.

  But it will also be embedded within a {\tt figure} ``environment.''
  (That is, between {\tt {\textbackslash}begin\{figure\}} and
  {\tt {\textbackslash}end\{figure\}}.)

  This makes it a ``float.'' \LaTeX\/ decides where it's going to be
  placed. The placement of floats is {\nvhilit the biggest pain} in
  using \LaTeX.

  The figure also gets a label, from the chunk name. (The
  {\tt {\textbackslash}label\{\}} bit.) This allows you to
  cross-reference the figure, to have the figure number determined
  automatically.

  The cross reference would be with {\tt {\textbackslash}ref\{fig:fig\_with\_caption\}}.

  When you use cross references, you need to run \LaTeX\/ twice: once
  to establish where things will sit on the page and how they are
  numbered, and a second time to insert the cross references.
}
\end{frame}




\begin{frame}[fragile]{Tables in \LaTeX}

\vspace{18pt}

\begin{lstlisting}
\begin{tabular}{rrrrr} \hline
& Estimate & Std. Error & t value & Pr($>$$|$t$|$) \\  \hline
(Intercept) & 0.04 & 0.11 &  0.4 & 0.69 \\
      x     & 0.98 & 0.10 & 10.0 & 0.00 \\ \hline
\end{tabular}
\end{lstlisting}

\note{
  Tables in \LaTeX\/ are a pain, but they offer extremely fine
  control.

  But writing this sort of code ({\tt \&} indicates breaks between
  columns, {\tt {\textbackslash}{\textbackslash}} indicates the end of
  a row) {\nhilit reproducibly} is hard.
}
\end{frame}



\begin{frame}[fragile]{xtable}

\vspace{18pt}

\begin{lstlisting}
<<generate_and_fit>>=
x <- rnorm(100)
y <- x + rnorm(100)
lm.out <- lm(y ~ x)
@

<<table, results="asis">>=
library(xtable)
xtable(lm.out, digits=c(0,2,2,1,2))
@


% a non-floating version
<<table, results="asis">>=
library(xtable)
xtab <- xtable(lm.out, digits=c(0,2,2,1,2))
print(xtab, floating=FALSE)
@
\end{lstlisting}

\note{
  xtable is a superb R package for producing \LaTeX\/ tables.
  You don't have complete control, but you do have a ton of
  control. The xtableGallery vignette shows you much of what can be
  done.

  Note that a lot of the options are for {\tt print.xtable}, so look
  at the help files for both {\tt xtable} and {\tt print.xtable}.

  For example, if you {\nhilit don't} want a table to be ``floating,''
  (within a {\tt table} environment, between
  {\tt {\textbackslash}begin\{table\}} and
  {\tt {\textbackslash}end\{table\}}),
  you need to use {\tt print.table} with {\tt floating=FALSE}.
}
\end{frame}


\begin{frame}{Read page proofs carefully}

\bigskip

As submitted
\bigskip

\figw{Figs/rigenome_as_submitted.png}{0.5}

\bigskip

As printed
\bigskip

\figw{Figs/rigenome_error.png}{0.5}

\bigskip

\hfill
{\lolit \scriptsize Broman (2005) Genetics 169:1133{\textendash}1146}

\note{
  Some journals re-type a bunch of your manuscript, sometimes
  introducing errors.

  So read page proofs {\nhilit carefully}. (What pain!) And post a
  preprint, say to arXiv.org or bioRxiv.org.

  The above is the most important equation in the paper, and I missed
  that they'd introduced a mistake.
}
\end{frame}




\begin{frame}[c]{Re-type that!}

\vspace{18pt}

\figh{Figs/preCC_table.png}{0.65}

\vfill
\hfill
{\lolit \scriptsize Broman (2012) Genetics 190:403{\textendash}412}

\note{
  I have a few papers with {\nhilit a lot} of equations. I hope
  they're not trying to re-type these. I generated them from code.
}
\end{frame}


\begin{frame}[c,fragile]{BibTeX for bibliographies}

\begin{lstlisting}
%bibliography format
\usepackage[authoryear]{natbib}
\bibpunct{(}{)}{;}{a}{}{,}

A number of investigators have developed methods for identifying
such sample mix-ups \citep{Westra2011, Schadt2012, Lynch2012,
Ekstrom2012}, and a similar approach was applied by
\citet{Baggerly2008, Baggerly2009} in their forensic...

\bibliographystyle{genetics}
\renewcommand*{\refname}{\centerline{\normalsize\sffamily
   \textbf{Literature Cited}}}
\bibliography{samplemixups}
\end{lstlisting}

\bigskip

\begin{lstlisting}
@article{Baggerly2008,
author = {Baggerly, Keith A. and Coombes, Kevin R.},
journal = {J. Clin. Oncol.},
pages = {1186--1187},
title = {Run batch effects potentially compromise...},
volume = {26},
year = {2008} }
\end{lstlisting}


\note{
  References with \LaTeX\/ are via BibTeX, which is fabulous once you
  get used to it. Most software to track references will produce
  BibTeX files for you.

  The formatting of citations and the reference listings, to match
  what the journal wants, can be painful. But I've figured out how to
  produce what \emph{Genetics\/} wants, and I send all of my papers
  there.

  The first box is the sort of code that would appear
  in your \LaTeX\/ file: the bit at the top goes in the header (before
  {\tt {\textbackslash}begin\{document\}}). The bit in the middle
  shows how to cite papers: use {\tt {\textbackslash}citep} to get the
  whole thing in parentheses, and use {\tt {\textbackslash}citet} to
  get a reference like ``...applied by Baggerly and Coombes (2008, 2009)...''
  The last bit in the first box produces the actual list of references.

  The second box is the BibTeX format for a particular reference.

  When you use BibTeX, you tend to run {\tt pdflatex}, then {\tt
  bibtex}, and then {\tt pdflatex} a couple of more times.
}
\end{frame}


\begin{frame}{Organizing analyses}

  \vspace{-6mm}

\bbi
\item Directory for the main analysis project
  \bi
  \item[] {\tt {\textasciitilde}/Projects/Blah}
  \ei
\item Directory for a paper
  \bi
  \item[] {\tt {\textasciitilde}/Docs/Papers/Blah}
  \ei
\item Paper directory may have an analysis directory
  \bi
  \item[] {\tt {\textasciitilde}/Docs/Papers/Blah/Analysis}
  \ei
\item Symbolic links to {\tt .RData} files
  \bi
  \item[] {\tt ln -s {\textasciitilde}/Projects/Blah/DerivedData/blah.RData .}
  \ei
\item Each part well organized and fully reproducible.
\item R Markdown reports documenting different aspects.
\item Analysis with the paper may be re-done "properly."
\ei

\note{
  This is how I organize a paper related to a larger project.

  Some of the work in the main project may be re-done a bit
  differently (or cleaner) in the analysis with the paper.

  You don't want to re-do {\nhilit all} analyses for the paper, but
  it'd also be nice to have the data and code related to the paper be
  a bit more self-contained.

  And usually when you're sitting down to write the paper, you have
  better ideas about how to re-do things properly, and so it might be
  a good idea to go ahead and re-do things.

  Ideally, you'd separate out each aspect of the analysis: data
  manipulation, data cleaning, and different parts of the analysis.

  Have an R Markdown document describing each aspect, with the
  actual manuscript and its figures and tables drawing from the
  results of those R Markdown documents.
}
\end{frame}


\begin{frame}[c,fragile]{Make every number reproducible.}

\begin{lstlisting}
<<define_numbers, echo=FALSE>>=
numbers <- c("one", "two", "three", "four", "five",
             "six", "seven", "eight", "nine", "ten")
cap <- function(vec) paste0(toupper(substr(vec, 1, 1)),
                            substr(vec, 2, nchar(vec)))
Numbers <- cap(numbers)
n <- sample(1:10, 1)
@

Then if I want to talk about a number, like \Sexpr{n}, I can
refer to it by name: \Sexpr{numbers[n]}. And I can start a
sentence with it. \Sexpr{Numbers[n]} grasshoppers walked into a
bar\dots

But be careful about singular vs. plural, and so write
\Sexpr{Numbers[n]} grasshopper\Sexpr{ifelse(n>1, "s", "")}
walked\dots
\end{lstlisting}

\note{
  Every statistic, figure and table in your manuscript should be fully
  reproducible. So when you're citing statistics, use
  {\tt {\textbackslash}Sexpr\{\}} liberally.

  This should inhibit you from writing numbers as words, though the
  \LaTeX\/ code can get a bit ugly.

  There's a bit of fanciness here about capitalization and about
  ensuring that singular or plural nouns are correct. If
  {\tt {\textbackslash}Sexpr\{\}} produces a character string, it ends
  up as plain text in your document

  I'll use a lot of {\tt myround()} from my R/broman package, too.

  Long explanations or descriptions of figures can't be fully
  reproducible, but the figures themselves and any statistics you
  mention should be.
}
\end{frame}


\begin{frame}[c,fragile]{Keep the figures separate}

\begin{lstlisting}
# simple make file

mypaper.pdf: mypaper.tex Figs/fig1.pdf Figs/fig2.pdf
    pdflatex mypaper

Figs/fig1.pdf: R/fig1.R
    cd R;R CMD BATCH fig1.R fig1.Rout

Figs/fig2.pdf: R/fig2.R
    cd R;R CMD BATCH fig2.R fig2.Rout
\end{lstlisting}

\bigskip \bigskip


\begin{lstlisting}
\clearpage
\includegraphics{Figs/fig1.pdf}

\clearpage
\includegraphics{Figs/fig2.pdf}
\end{lstlisting}

\note{
  While you {\nhilit could} include all code in your {\tt .Rnw} file,
  I prefer to pull out the code for my figures as separate files, and
  then write a {\tt Makefile} for the manuscript construction and
  include them with {\tt {\textbackslash}includegraphics}.

  The advantage of this is the ability to reuse the figures in
  talks or whatever. Also, journals will generally want the figures as
  separate files. Finally, the code for my figures is often incredibly
  long and ugly, so it's best to separate it out.

  Ideally, the code for a figure would be structured as a function and
  then a function call.  Put a bit more effort into the code, so that
  you can reuse it later for a similar figure with different data.
  At the very least, you should write the repeated bits as functions.

  If your function takes arguments that define the placement of
  things (padding for text and so forth), then the fine adjustments of the
  figure appearance would be easier.
}
\end{frame}


\begin{frame}{Version Control}

\bbi
\item Your manuscript is under version control, right?
\onslide<2->{\item Local or private repository for the whole thing
  \bi
  \item including reviewers' reports and my response
  \item PDF of submitted and final manuscript
  \ei
\item Snapshot of the final version as a public repository
  \bi
  \item I don't really want to show the whole history
  \ei
}
\ei

\note{
  Git is as good for tracking manuscripts and data analyses as it is
  for tracking code. Use it!

  But I don't want to make {\nhilit everything} public, and I want to
  include private stuff in my repository.

  I've been using just a local repository, but I'm moving towards
  having a private repository hosted on BitBucket.

  I'll put a snapshot of the final version, and maybe a few final
  changes, on GitHub.
}
\end{frame}


\begin{frame}{Word}

\bbi
\item With papers led by a collaborator, I'm usually stuck with Word.
\item But my analyses and figures are fully reproducible.
\item Create an R Markdown document with the detailed results.
\ei

\note{
  Often, you'll be stuck with Word. And you can't reproducibly insert
  numbers into Word.

  So have a separate R Markdown report with the detailed results,
  including every statistic that will get inserted into Word.

  And take control of the figures and ensure that they are
  reproducible (and respectable).

  Teach your collaborators to at least have their figures be
  reproducible?
}
\end{frame}



\begin{frame}{Summary}

\vspace{-6mm}

\bbi
\item \LaTeX\/ is brilliant for fine control and for equations
\item Floating figures and tables can be a pain
\item You use KnitR with \LaTeX\/ much the same way as you'd used it
  with Markdown.
\item Ensure that every statistic, figure, and table in your paper are
  fully reproducible.
\item Use xtable to make tables.
\item Separate out the code for the figures.
\item Use version control!
\ei

\note{
  Summaries are helpful.
}
\end{frame}


\end{document}
